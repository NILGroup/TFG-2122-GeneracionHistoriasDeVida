% ----------------------------------------------------------------------
%
%                            TFMTesis.tex
%
%----------------------------------------------------------------------
%
% Este fichero contiene el "documento maestro" del documento. Lo único
% que hace es configurar el entorno LaTeX e incluir los ficheros .tex
% que contienen cada sección.
%
%----------------------------------------------------------------------
%
% Los ficheros necesarios para este documento son:
%
%       TeXiS/* : ficheros de la plantilla TeXiS.
%       Cascaras/* : ficheros con las partes del documento que no
%          son capítulos ni apéndices (portada, agradecimientos, etc.)
%       Capitulos/*.tex : capítulos de la tesis
%       Apendices/*.tex: apéndices de la tesis
%       constantes.tex: constantes LaTeX
%       config.tex : configuración de la "compilación" del documento
%       guionado.tex : palabras con guiones
%
% Para la bibliografía, además, se necesitan:
%
%       *.bib : ficheros con la información de las referencias
%
% ---------------------------------------------------------------------

\documentclass[11pt,a4paper,twoside]{book}
\usepackage{verse}
\usepackage{amsmath}
\usepackage{mathpazo}
\usepackage{domitian}
\usepackage[T1]{fontenc}
\usepackage{listings}
\renewcommand{\lstlistingname}{Código}

\renewcommand{\arraystretch}{1.7}
\usepackage{xcolor}
\usepackage{multirow}
%\definecolor{codegreen}{rgb}{0,0.6,0}
%\definecolor{codegray}{rgb}{0.5,0.5,0.5}
\definecolor{codepurple}{rgb}{0.58,0,0.82}
\definecolor{backcolour}{rgb}{0.95,0.95,0.92}

\lstdefinestyle{mystyle}{
	backgroundcolor=\color{backcolour},   
	%commentstyle=\color{codegreen},
	%keywordstyle=\color{magenta},
	%numberstyle=\tiny\color{codegray},
	%stringstyle=\color{codepurple},
	basicstyle=\ttfamily\footnotesize,
	breakatwhitespace=false,         
	breaklines=true,                 
	captionpos=b,                    
	keepspaces=true,                 
	numbers=left,                    
	numbersep=5pt,                  
	showspaces=false,                
	showstringspaces=false,
	showtabs=false,                  
	tabsize=2
}

\lstset{style=mystyle}


\let\oldstylenums\oldstyle

\usepackage[%
	a4paper,
	left=1.3in,
	right=1.3in,
	top=1.3in,
	bottom=1.3in
	]{geometry}
\renewcommand{\baselinestretch}{1.3}
\setlength{\parskip}{0.2ex}

%
% Definimos  el   comando  \compilaCapitulo,  que   luego  se  utiliza
% (opcionalmente) en config.tex. Quedaría  mejor si también se definiera
% en  ese fichero,  pero por  el modo  en el  que funciona  eso  no es
% posible. Puedes consultar la documentación de ese fichero para tener
% más  información. Definimos también  \compilaApendice, que  tiene el
% mismo  cometido, pero  que se  utiliza para  compilar  únicamente un
% apéndice.
%
%
% Si  queremos   compilar  solo   una  parte  del   documento  podemos
% especificar mediante  \includeonly{...} qué ficheros  son los únicos
% que queremos  que se incluyan.  Esto  es útil por  ejemplo para sólo
% compilar un capítulo.
%
% El problema es que todos aquellos  ficheros que NO estén en la lista
% NO   se  incluirán...  y   eso  también   afecta  a   ficheros  de
% la plantilla...
%
% Total,  que definimos  una constante  con los  ficheros  que siempre
% vamos a querer compilar  (aquellos relacionados con configuración) y
% luego definimos \compilaCapitulo.
\newcommand{\ficherosBasicosTeXiS}{%
TeXiS/TeXiS_pream,TeXiS/TeXiS_cab,TeXiS/TeXiS_bib,TeXiS/TeXiS_cover%
}
\newcommand{\ficherosBasicosTexto}{%
constantes,guionado,Cascaras/bibliografia,config%
}
\newcommand{\compilaCapitulo}[1]{%
\includeonly{\ficherosBasicosTeXiS,\ficherosBasicosTexto,Capitulos/#1}%
}

\newcommand{\compilaApendice}[1]{%
\includeonly{\ficherosBasicosTeXiS,\ficherosBasicosTexto,Apendices/#1}%
}

%- - - - - - - - - - - - - - - - - - - - - - - - - - - - - - - - - - -
%            Preámbulo del documento. Configuraciones varias
%- - - - - - - - - - - - - - - - - - - - - - - - - - - - - - - - - - -

% Define  el  tipo  de  compilación que  estamos  haciendo.   Contiene
% definiciones  de  constantes que  cambian  el  comportamiento de  la
% compilación. Debe incluirse antes del paquete TeXiS/TeXiS.sty
%---------------------------------------------------------------------
%
%                          config.tex
%
%---------------------------------------------------------------------
%
% Contiene la  definición de constantes  que determinan el modo  en el
% que se compilará el documento.
%
%---------------------------------------------------------------------
%
% En concreto, podemos  indicar si queremos "modo release",  en el que
% no  aparecerán  los  comentarios  (creados  mediante  \com{Texto}  o
% \comp{Texto}) ni los "por  hacer" (creados mediante \todo{Texto}), y
% sí aparecerán los índices. El modo "debug" (o mejor dicho en modo no
% "release" muestra los índices  (construirlos lleva tiempo y son poco
% útiles  salvo  para   la  versión  final),  pero  sí   el  resto  de
% anotaciones.
%
% Si se compila con LaTeX (no  con pdflatex) en modo Debug, también se
% muestran en una esquina de cada página las entradas (en el índice de
% palabras) que referencian  a dicha página (consulta TeXiS_pream.tex,
% en la parte referente a show).
%
% El soporte para  el índice de palabras en  TeXiS es embrionario, por
% lo  que no  asumas que  esto funcionará  correctamente.  Consulta la
% documentación al respecto en TeXiS_pream.tex.
%
%
% También  aquí configuramos  si queremos  o  no que  se incluyan  los
% acrónimos  en el  documento final  en la  versión release.  Para eso
% define (o no) la constante \acronimosEnRelease.
%
% Utilizando \compilaCapitulo{nombre}  podemos también especificar qué
% capítulo(s) queremos que se compilen. Si no se pone nada, se compila
% el documento  completo.  Si se pone, por  ejemplo, 01Introduccion se
% compilará únicamente el fichero Capitulos/01Introduccion.tex
%
% Para compilar varios  capítulos, se separan sus nombres  con comas y
% no se ponen espacios de separación.
%
% En realidad  la macro \compilaCapitulo  está definida en  el fichero
% principal tesis.tex.
%
%---------------------------------------------------------------------


% Comentar la línea si no se compila en modo release.
% TeXiS hará el resto.
% ¡¡¡Si cambias esto, haz un make clean antes de recompilar!!!
\def\release{1}


% Descomentar la linea si se quieren incluir los
% acrónimos en modo release (en modo debug
% no se incluirán nunca).
% ¡¡¡Si cambias esto, haz un make clean antes de recompilar!!!
%\def\acronimosEnRelease{1}


% Descomentar la línea para establecer el capítulo que queremos
% compilar

% \compilaCapitulo{01Introduccion}
% \compilaCapitulo{02EstructuraYGeneracion}
% \compilaCapitulo{03Edicion}
% \compilaCapitulo{04Imagenes}
% \compilaCapitulo{05Bibliografia}
% \compilaCapitulo{06Makefile}

% \compilaApendice{01AsiSeHizo}

% Variable local para emacs, para  que encuentre el fichero maestro de
% compilación y funcionen mejor algunas teclas rápidas de AucTeX
%%%
%%% Local Variables:
%%% mode: latex
%%% TeX-master: "./Tesis.tex"
%%% End:


% Paquete de la plantilla
\usepackage{TeXiS/TeXiS}

% Incluimos el fichero con comandos de constantes
%---------------------------------------------------------------------
%
%                          constantes.tex
%
%---------------------------------------------------------------------
%
% Fichero que  declara nuevos comandos LaTeX  sencillos realizados por
% comodidad en la escritura de determinadas palabras
%
%---------------------------------------------------------------------

%%%%%%%%%%%%%%%%%%%%%%%%%%%%%%%%%%%%%%%%%%%%%%%%%%%%%%%%%%%%%%%%%%%%%%
% Comando: 
%
%       \titulo
%
% Resultado: 
%
% Escribe el título del documento.
%%%%%%%%%%%%%%%%%%%%%%%%%%%%%%%%%%%%%%%%%%%%%%%%%%%%%%%%%%%%%%%%%%%%%%
\def\titulo{\textsc{TeXiS}: Una plantilla de \LaTeX\
  para Tesis y otros documentos}

%%%%%%%%%%%%%%%%%%%%%%%%%%%%%%%%%%%%%%%%%%%%%%%%%%%%%%%%%%%%%%%%%%%%%%
% Comando: 
%
%       \autor
%
% Resultado: 
%
% Escribe el autor del documento.
%%%%%%%%%%%%%%%%%%%%%%%%%%%%%%%%%%%%%%%%%%%%%%%%%%%%%%%%%%%%%%%%%%%%%%
\def\autor{Marco Antonio y Pedro Pablo G\'omez Mart\'in}

% Variable local para emacs, para  que encuentre el fichero maestro de
% compilación y funcionen mejor algunas teclas rápidas de AucTeX

%%%
%%% Local Variables:
%%% mode: latex
%%% TeX-master: "tesis.tex"
%%% End:


% Sacamos en el log de la compilación el copyright
%\typeout{Copyright Marco Antonio and Pedro Pablo Gomez Martin}

%
% "Metadatos" para el PDF
%
\ifpdf\hypersetup{%
    pdftitle = {\titulo},
    pdfsubject = {Plantilla de Tesis},
    pdfkeywords = {Plantilla, LaTeX, tesis, trabajo de
      investigación, trabajo de Master},
    pdfauthor = {\textcopyright\ \autor},
    pdfcreator = {\LaTeX\ con el paquete \flqq hyperref\frqq},
    pdfproducer = {pdfeTeX-0.\the\pdftexversion\pdftexrevision},
    }
    \pdfinfo{/CreationDate (\today)}
\fi


%- - - - - - - - - - - - - - - - - - - - - - - - - - - - - - - - - - -
%                        Documento
%- - - - - - - - - - - - - - - - - - - - - - - - - - - - - - - - - - -
\begin{document}

% Incluimos el  fichero de definición de guionado  de algunas palabras
% que LaTeX no ha dividido como debería
%----------------------------------------------------------------
%
%                          guionado.tex
%
%----------------------------------------------------------------
%
% Fichero con algunas divisiones de palabras que LaTeX no
% hace correctamente si no se le da alguna ayuda.
%
%----------------------------------------------------------------

\hyphenation{
% a
abs-trac-to
abs-trac-tos
abs-trac-ta
abs-trac-tas
ac-tua-do-res
a-gra-de-ci-mien-tos
ana-li-za-dor
an-te-rio-res
an-te-rior-men-te
apa-rien-cia
a-pro-pia-do
a-pro-pia-dos
a-pro-pia-da
a-pro-pia-das
a-pro-ve-cha-mien-to
a-que-llo
a-que-llos
a-que-lla
a-que-llas
a-sig-na-tu-ra
a-sig-na-tu-ras
a-so-cia-da
a-so-cia-das
a-so-cia-do
a-so-cia-dos
au-to-ma-ti-za-do
% b
batch
bi-blio-gra-fía
bi-blio-grá-fi-cas
bien
bo-rra-dor
boo-l-ean-expr
% c
ca-be-ce-ra
call-me-thod-ins-truc-tion
cas-te-lla-no
cir-cuns-tan-cia
cir-cuns-tan-cias
co-he-ren-te
co-he-ren-tes
co-he-ren-cia
co-li-bri
co-men-ta-rio
co-mer-cia-les
co-no-ci-mien-to
cons-cien-te
con-si-de-ra-ba
con-si-de-ra-mos
con-si-de-rar-se
cons-tan-te
cons-trucción
cons-tru-ye
cons-tru-ir-se
con-tro-le
co-rrec-ta-men-te
co-rres-pon-den
co-rres-pon-dien-te
co-rres-pon-dien-tes
co-ti-dia-na
co-ti-dia-no
crean
cris-ta-li-zan
cu-rri-cu-la
cu-rri-cu-lum
cu-rri-cu-lar
cu-rri-cu-la-res
% d
de-di-ca-do
de-di-ca-dos
de-di-ca-da
de-di-ca-das
de-rro-te-ro
de-rro-te-ros
de-sa-rro-llo
de-sa-rro-llos
de-sa-rro-lla-do
de-sa-rro-lla-dos
de-sa-rro-lla-da
de-sa-rro-lla-das
de-sa-rro-lla-dor
de-sa-rro-llar
des-cri-bi-re-mos
des-crip-ción
des-crip-cio-nes
des-cri-to
des-pués
de-ta-lla-do
de-ta-lla-dos
de-ta-lla-da
de-ta-lla-das
di-a-gra-ma
di-a-gra-mas
di-se-ños
dis-po-ner
dis-po-ni-bi-li-dad
do-cu-men-ta-da
do-cu-men-to
do-cu-men-tos
% e
edi-ta-do
e-du-ca-ti-vo
e-du-ca-ti-vos
e-du-ca-ti-va
e-du-ca-ti-vas
e-la-bo-ra-do
e-la-bo-ra-dos
e-la-bo-ra-da
e-la-bo-ra-das
es-co-llo
es-co-llos
es-tu-dia-do
es-tu-dia-dos
es-tu-dia-da
es-tu-dia-das
es-tu-dian-te
e-va-lua-cio-nes
e-va-lua-do-res
exis-ten-tes
exhaus-ti-va
ex-pe-rien-cia
ex-pe-rien-cias
% f
for-ma-li-za-do
% g
ge-ne-ra-ción
ge-ne-ra-dor
ge-ne-ra-do-res
ge-ne-ran
% h
he-rra-mien-ta
he-rra-mien-tas
% i
i-dio-ma
i-dio-mas
im-pres-cin-di-ble
im-pres-cin-di-bles
in-de-xa-do
in-de-xa-dos
in-de-xa-da
in-de-xa-das
in-di-vi-dual
in-fe-ren-cia
in-fe-ren-cias
in-for-ma-ti-ca
in-gre-dien-te
in-gre-dien-tes
in-me-dia-ta-men-te
ins-ta-la-do
ins-tan-cias
% j
% k
% l
len-gua-je
li-be-ra-to-rio
li-be-ra-to-rios
li-be-ra-to-ria
li-be-ra-to-rias
li-mi-ta-do
li-te-ra-rio
li-te-ra-rios
li-te-ra-ria
li-te-ra-rias
lo-tes
% m
ma-ne-ra
ma-nual
mas-que-ra-de
ma-yor
me-mo-ria
mi-nis-te-rio
mi-nis-te-rios
mo-de-lo
mo-de-los
mo-de-la-do
mo-du-la-ri-dad
mo-vi-mien-to
% n
na-tu-ral
ni-vel
nues-tro
% o
obs-tan-te
o-rien-ta-do
o-rien-ta-dos
o-rien-ta-da
o-rien-ta-das
% p
pa-ra-le-lo
pa-ra-le-la
par-ti-cu-lar
par-ti-cu-lar-men-te
pe-da-gó-gi-ca
pe-da-gó-gi-cas
pe-da-gó-gi-co
pe-da-gó-gi-cos
pe-rio-di-ci-dad
per-so-na-je
plan-te-a-mien-to
plan-te-a-mien-tos
po-si-ción
pre-fe-ren-cia
pre-fe-ren-cias
pres-cin-di-ble
pres-cin-di-bles
pri-me-ra
pro-ble-ma
pro-ble-mas
pró-xi-mo
pu-bli-ca-cio-nes
pu-bli-ca-do
% q
% r
rá-pi-da
rá-pi-do
ra-zo-na-mien-to
ra-zo-na-mien-tos
re-a-li-zan-do
re-fe-ren-cia
re-fe-ren-cias
re-fe-ren-cia-da
re-fe-ren-cian
re-le-van-tes
re-pre-sen-ta-do
re-pre-sen-ta-dos
re-pre-sen-ta-da
re-pre-sen-ta-das
re-pre-sen-tar-lo
re-qui-si-to
re-qui-si-tos
res-pon-der
res-pon-sa-ble
% s
se-pa-ra-do
si-guien-do
si-guien-te
si-guien-tes
si-guie-ron
si-mi-lar
si-mi-la-res
si-tua-ción
% t
tem-pe-ra-ments
te-ner
trans-fe-ren-cia
trans-fe-ren-cias
% u
u-sua-rio
Unreal-Ed
% v
va-lor
va-lo-res
va-rian-te
ver-da-de-ro
ver-da-de-ros
ver-da-de-ra
ver-da-de-ras
ver-da-de-ra-men-te
ve-ri-fi-ca
% w
% x
% y
% z
}
% Variable local para emacs, para que encuentre el fichero
% maestro de compilación
%%%
%%% Local Variables:
%%% mode: latex
%%% TeX-master: "./Tesis.tex"
%%% End:


% Marcamos  el inicio  del  documento para  la  numeración de  páginas
% (usando números romanos para esta primera fase).
\frontmatter
\pagestyle{empty}

%---------------------------------------------------------------------
%
%                          configCover.tex
%
%---------------------------------------------------------------------
%
% cover.tex
% Copyright 2009 Marco Antonio Gomez-Martin, Pedro Pablo Gomez-Martin
%
% This file belongs to the TeXiS manual, a LaTeX template for writting
% Thesis and other documents. The complete last TeXiS package can
% be obtained from http://gaia.fdi.ucm.es/projects/texis/
%
% Although the TeXiS template itself is distributed under the 
% conditions of the LaTeX Project Public License
% (http://www.latex-project.org/lppl.txt), the manual content
% uses the CC-BY-SA license that stays that you are free:
%
%    - to share & to copy, distribute and transmit the work
%    - to remix and to adapt the work
%
% under the following conditions:
%
%    - Attribution: you must attribute the work in the manner
%      specified by the author or licensor (but not in any way that
%      suggests that they endorse you or your use of the work).
%    - Share Alike: if you alter, transform, or build upon this
%      work, you may distribute the resulting work only under the
%      same, similar or a compatible license.
%
% The complete license is available in
% http://creativecommons.org/licenses/by-sa/3.0/legalcode
%
%---------------------------------------------------------------------
%
% Fichero que contiene la configuración de la portada y de la 
% primera hoja del documento.
%
%---------------------------------------------------------------------


% Pueden configurarse todos los elementos del contenido de la portada
% utilizando comandos.

%%%%%%%%%%%%%%%%%%%%%%%%%%%%%%%%%%%%%%%%%%%%%%%%%%%%%%%%%%%%%%%%%%%%%%
% Título del documento:
% \tituloPortada{titulo}
% Nota:
% Si no se define se utiliza el del \titulo. Este comando permite
% cambiar el título de forma que se especifiquen dónde se quieren
% los retornos de carro cuando se utilizan fuentes grandes.
%%%%%%%%%%%%%%%%%%%%%%%%%%%%%%%%%%%%%%%%%%%%%%%%%%%%%%%%%%%%%%%%%%%%%%
\tituloPortada{%
Generación de texto para historias de vida basadas en narrativa
}

%%%%%%%%%%%%%%%%%%%%%%%%%%%%%%%%%%%%%%%%%%%%%%%%%%%%%%%%%%%%%%%%%%%%%%
% Autor del documento:
% \autorPortada{Nombre}
% Se utiliza en la portada y en el valor por defecto del
% primer subtítulo de la segunda portada.
%%%%%%%%%%%%%%%%%%%%%%%%%%%%%%%%%%%%%%%%%%%%%%%%%%%%%%%%%%%%%%%%%%%%%%
\autorPortada{María Cristina Alameda Salas}

%%%%%%%%%%%%%%%%%%%%%%%%%%%%%%%%%%%%%%%%%%%%%%%%%%%%%%%%%%%%%%%%%%%%%%
% Fecha de publicación:
% \fechaPublicacion{Fecha}
% Puede ser vacío. Aparece en la última línea de ambas portadas
%%%%%%%%%%%%%%%%%%%%%%%%%%%%%%%%%%%%%%%%%%%%%%%%%%%%%%%%%%%%%%%%%%%%%%
\fechaPublicacion{\today}

%%%%%%%%%%%%%%%%%%%%%%%%%%%%%%%%%%%%%%%%%%%%%%%%%%%%%%%%%%%%%%%%%%%%%%
% Imagen de la portada (y escala)
% \imagenPortada{Fichero}
% \escalaImagenPortada{Numero}
% Si no se especifica, se utiliza la imagen TODO.pdf
%%%%%%%%%%%%%%%%%%%%%%%%%%%%%%%%%%%%%%%%%%%%%%%%%%%%%%%%%%%%%%%%%%%%%%
\imagenPortada{Imagenes/Vectorial/escudoUCM}
\escalaImagenPortada{.2}

%%%%%%%%%%%%%%%%%%%%%%%%%%%%%%%%%%%%%%%%%%%%%%%%%%%%%%%%%%%%%%%%%%%%%%
% Tipo de documento.
% \tipoDocumento{Tipo}
% Para el texto justo debajo del escudo.
% Si no se indica, se utiliza "TESIS DOCTORAL".
%%%%%%%%%%%%%%%%%%%%%%%%%%%%%%%%%%%%%%%%%%%%%%%%%%%%%%%%%%%%%%%%%%%%%%
\tipoDocumento{Trabajo de Fin de Grado}

%%%%%%%%%%%%%%%%%%%%%%%%%%%%%%%%%%%%%%%%%%%%%%%%%%%%%%%%%%%%%%%%%%%%%%
% Institución/departamento asociado al documento.
% \institucion{Nombre}
% Puede tener varias líneas. Se utiliza en las dos portadas.
% Si no se indica aparecerá vacío.
%%%%%%%%%%%%%%%%%%%%%%%%%%%%%%%%%%%%%%%%%%%%%%%%%%%%%%%%%%%%%%%%%%%%%%
\institucion{%
Grado en Ingeniería Informática\\[0.2em]
Facultad de Informática\\[0.2em]
Universidad Complutense de Madrid
}

%%%%%%%%%%%%%%%%%%%%%%%%%%%%%%%%%%%%%%%%%%%%%%%%%%%%%%%%%%%%%%%%%%%%%%
% Director del trabajo.
% \directorPortada{Nombre}
% Se utiliza para el valor por defecto del segundo subtítulo, donde
% se indica quién es el director del trabajo.
% Si se fuerza un subtítulo distinto, no hace falta definirlo.
%%%%%%%%%%%%%%%%%%%%%%%%%%%%%%%%%%%%%%%%%%%%%%%%%%%%%%%%%%%%%%%%%%%%%%
\directorPortada{Raquel Hervás Ballesteros\\Gonzalo Méndez Pozo}

%%%%%%%%%%%%%%%%%%%%%%%%%%%%%%%%%%%%%%%%%%%%%%%%%%%%%%%%%%%%%%%%%%%%%%
% Texto del primer subtítulo de la segunda portada.
% \textoPrimerSubtituloPortada{Texto}
% Para configurar el primer "texto libre" de la segunda portada.
% Si no se especifica se indica "Memoria que presenta para optar al
% título de Doctor en Informática" seguido del \autorPortada.
%%%%%%%%%%%%%%%%%%%%%%%%%%%%%%%%%%%%%%%%%%%%%%%%%%%%%%%%%%%%%%%%%%%%%%
\textoPrimerSubtituloPortada{%
\textbf{Trabajo de Fin de Grado en Ingeniería Informática}  \\ [0.3em]
\textbf{Departamento de XXXXXXXXXXXXX} \\ [0.3em]
}

%%%%%%%%%%%%%%%%%%%%%%%%%%%%%%%%%%%%%%%%%%%%%%%%%%%%%%%%%%%%%%%%%%%%%%
% Texto del segundo subtítulo de la segunda portada.
% \textoSegundoSubtituloPortada{Texto}
% Para configurar el segundo "texto libre" de la segunda portada.
% Si no se especifica se indica "Dirigida por el Doctor" seguido
% del \directorPortada.
%%%%%%%%%%%%%%%%%%%%%%%%%%%%%%%%%%%%%%%%%%%%%%%%%%%%%%%%%%%%%%%%%%%%%%
\textoSegundoSubtituloPortada{%
\textbf{Convocatoria: }\textit{Febrero/Junio/Septiembre \the\year} \\ [0.2em]
\textbf{Calificación: }\textit{Nota}
}

%%%%%%%%%%%%%%%%%%%%%%%%%%%%%%%%%%%%%%%%%%%%%%%%%%%%%%%%%%%%%%%%%%%%%%
% \explicacionDobleCara
% Si se utiliza, se aclara que el documento está preparado para la
% impresión a doble cara.
%%%%%%%%%%%%%%%%%%%%%%%%%%%%%%%%%%%%%%%%%%%%%%%%%%%%%%%%%%%%%%%%%%%%%%
\explicacionDobleCara

%%%%%%%%%%%%%%%%%%%%%%%%%%%%%%%%%%%%%%%%%%%%%%%%%%%%%%%%%%%%%%%%%%%%%%
% \isbn
% Si se utiliza, aparecerá el ISBN detrás de la segunda portada.
%%%%%%%%%%%%%%%%%%%%%%%%%%%%%%%%%%%%%%%%%%%%%%%%%%%%%%%%%%%%%%%%%%%%%%
%\isbn{978-84-692-7109-4}


%%%%%%%%%%%%%%%%%%%%%%%%%%%%%%%%%%%%%%%%%%%%%%%%%%%%%%%%%%%%%%%%%%%%%%
% \copyrightInfo
% Si se utiliza, aparecerá información de los derechos de copyright
% detrás de la segunda portada.
%%%%%%%%%%%%%%%%%%%%%%%%%%%%%%%%%%%%%%%%%%%%%%%%%%%%%%%%%%%%%%%%%%%%%%
\copyrightInfo{\autor}


%%
%% Creamos las portadas
%%
\makeCover

% Variable local para emacs, para que encuentre el fichero
% maestro de compilación
%%%
%%% Local Variables:
%%% mode: latex
%%% TeX-master: "../Tesis.tex"
%%% End:

%%\chapter*{Autorización de difusión}

   
El abajo firmante, matriculado en el Máster en Ingeniería en Informática de la Facultad de Informática, autoriza a la Universidad Complutense de Madrid (UCM) a difundir y utilizar con fines académicos, no comerciales y mencionando expresamente a su autor el presente Trabajo Fin de Máster: ``TITULO DEL TRABAJO'', realizado durante el curso académico CURSO bajo la dirección de DIRECTORES en el Departamento de XXXXXXXXXXXXXXXXXXXXXXXX, y a la Biblioteca de la UCM a depositarlo en el Archivo Institucional E-Prints Complutense con el objeto de incrementar la difusión, uso e impacto del trabajo en Internet y garantizar su preservación y acceso a largo plazo.

\vspace{5cm}

% +--------------------------------------------------------------------+
% | On the line below, replace "Enter Your Name" with your name
% | Use the same form of your name as it appears on your title page.
% | Use mixed case, for example, Lori Goetsch.
% +--------------------------------------------------------------------+
\begin{center}
	\large Nombre Del Alumno\\
	
	\vspace{0.5cm}
	
	% +--------------------------------------------------------------------+
	% | On the line below, replace Fecha
	% |
	% +--------------------------------------------------------------------+
	
	\today\\
	
\end{center}

% +--------------------------------------------------------------------+
% | Dedication Page (Optional)
% +--------------------------------------------------------------------+

\chapter*{Dedicatoria}


Texto de la dedicatoria...
% +--------------------------------------------------------------------+
% | Acknowledgements Page (Optional)                                   |
% +--------------------------------------------------------------------+

\chapter*{Agradecimientos}

Texto de los agradecimientos












\chapter*{Resumen}

Resumen en español del trabajo


\section*{Palabras clave}
   
\noindent Máximo 10 palabras clave separadas por comas

   



\begin{otherlanguage}{english}
\chapter*{Abstract}

Abstract in English.


\section*{Keywords}

\noindent 10 keywords max., separated by commas.




% Si el trabajo se escribe en inglés, comentar esta línea y descomentar
% otra igual que hay justo antes de \end{document}
\end{otherlanguage}

\ifx\generatoc\undefined
\else
%---------------------------------------------------------------------
%
%                          TeXiS_toc.tex
%
%---------------------------------------------------------------------
%
% TeXiS_toc.tex
% Copyright 2009 Marco Antonio Gomez-Martin, Pedro Pablo Gomez-Martin
%
% This file belongs to TeXiS, a LaTeX template for writting
% Thesis and other documents. The complete last TeXiS package can
% be obtained from http://gaia.fdi.ucm.es/projects/texis/
%
% This work may be distributed and/or modified under the
% conditions of the LaTeX Project Public License, either version 1.3
% of this license or (at your option) any later version.
% The latest version of this license is in
%   http://www.latex-project.org/lppl.txt
% and version 1.3 or later is part of all distributions of LaTeX
% version 2005/12/01 or later.
%
% This work has the LPPL maintenance status `maintained'.
% 
% The Current Maintainers of this work are Marco Antonio Gomez-Martin
% and Pedro Pablo Gomez-Martin
%
%---------------------------------------------------------------------
%
% Contiene  los  comandos  para  generar los  índices  del  documento,
% entendiendo por índices las tablas de contenidos.
%
% Genera  el  índice normal  ("tabla  de  contenidos"),  el índice  de
% figuras y el de tablas. También  crea "marcadores" en el caso de que
% se esté compilando con pdflatex para que aparezcan en el PDF.
%
%---------------------------------------------------------------------


% Primero un poquito de configuración...


% Pedimos que inserte todos los epígrafes hasta el nivel \subsection en
% la tabla de contenidos.
\setcounter{tocdepth}{2} 

% Le  pedimos  que nos  numere  todos  los  epígrafes hasta  el  nivel
% \subsubsection en el cuerpo del documento.
\setcounter{secnumdepth}{4} 


% Creamos los diferentes índices.

% Lo primero un  poco de trabajo en los marcadores  del PDF. No quiero
% que  salga una  entrada  por cada  índice  a nivel  0...  si no  que
% aparezca un marcador "Índices", que  tenga dentro los otros tipos de
% índices.  Total, que creamos el marcador "Índices".
% Antes de  la creación  de los índices,  se añaden los  marcadores de
% nivel 1.

\ifpdf
   \pdfbookmark{Índices}{indices}
\fi

% Tabla de contenidos.
%
% La  inclusión  de '\tableofcontents'  significa  que  en la  primera
% pasada  de  LaTeX  se  crea   un  fichero  con  extensión  .toc  con
% información sobre la tabla de contenidos (es conceptualmente similar
% al  .bbl de  BibTeX, creo).  En la  segunda ejecución  de  LaTeX ese
% documento se utiliza para  generar la verdadera página de contenidos
% usando la  información sobre los  capítulos y demás guardadas  en el
% .toc
\ifpdf
   \pdfbookmark[1]{Tabla de Contenidos}{tabla de contenidos}
\fi
\setcounter{tocdepth}{2}
\cabeceraEspecial{\'Indice}

\tableofcontents

\newpage 

% Índice de figuras
%
% La idea es semejante que para  el .toc del índice, pero ahora se usa
% extensión .lof (List Of Figures) con la información de las figuras.

\ifpdf
   \pdfbookmark[1]{Índice de figuras}{indice de figuras}
\fi

\cabeceraEspecial{\'Indice de figuras}

\listoffigures

\newpage

% Índice de tablas
% Como antes, pero ahora .lot (List Of Tables)

\ifpdf
   \pdfbookmark[1]{Índice de tablas}{indice de tablas}
\fi

\cabeceraEspecial{\'Indice de tablas}

\listoftables

\newpage

% Variable local para emacs, para  que encuentre el fichero maestro de
% compilación y funcionen mejor algunas teclas rápidas de AucTeX

%%%
%%% Local Variables:
%%% mode: latex
%%% TeX-master: "../Tesis.tex"
%%% End:

\fi

% Marcamos el  comienzo de  los capítulos (para la numeración  de las
% páginas) y ponemos la cabecera normal
\mainmatter

\pagestyle{fancy}
\restauraCabecera

%%%%%%%%%%%%%%%%%%%%%%%%%%%%%%%%%%%%%%%%%%%%%%%%%%%%%%%%%%%%%%%%%%%%%%%%%%%
% Si el TFM se escribe en ingles, comentar las siguientes líneas 
% porque no hace falta incluir nuevamente la Introducción en inglés
\begin{otherlanguage}{english}
\chapter{Introduction}
\label{cap:introduction}












\end{otherlanguage}
\addtocounter{chapter}{-1} 
%%%%%%%%%%%%%%%%%%%%%%%%%%%%%%%%%%%%%%%%%%%%%%%%%%%%%%%%%%%%%%%%%%%%%%%%%%%

\chapter{Introducción}
\label{cap:introduccion}

\chapterquote{Frase célebre dicha por alguien inteligente}{Autor}

\section{Contexto}
La enfermedad de Alzheimer es un trastorno neurológico caracterizado por cambios degenerativos en diferentes sistemas neurotransmisores que abocan finalmente a la muerte de las células nerviosas del cerebro. Por lo general, los síntomas iniciales de la enfermedad suelen atribuirse a un olvido puntual o la vejez, nada más lejos de la realidad. Según avanza la enfermedad, sus síntomas lo hacen con ella, agravándose y aumentando cada vez más hasta que el deterioro cognitivo ocasionado llega a afectar significativamente a las actividades de la vida diaria y finalmente a las necesidades fisiológicas básicas.

La evolución del Alzheimer se puede dividir en tres fases o etapas: Inicial, en la que se observa un deterioro cognitivo leve como puede ser la pérdida paulatina de la memoria episódica. Intermedia, con pérdidas de la memoria reciente asociada a un deterioro mayor. Avanzada, se va perdiendo progresivamente la memoria de los acontecimientos más antiguos, además también se observa un gran deterioro físico.

Los síntomas del Alzheimer son muy diversos, ya que no solo provoca problemas de memoria, sino también alteraciones en el estado de ánimo y la conducta, dificultad de toma de decisiones, desorientación, problemas del lenguaje, dificultad para comer y movilidad reducida, entre otros. Estos síntomas dependen de la fase evolutiva de la enfermedad.

En la actualizad, el Alzheimer es una enfermedad irreversible, no tiene cura. Sin embargo, existen diversos tratamientos que pretenden ralentizar el avance de la enfermedad y/o mejorar la calidad de vida de los pacientes. Podemos dividir estos tratamientos en dos ramas: tratamientos farmacológicos, que hacen uso de medicamentos, y tratamientos no farmacológicos o psicosociales, que no utilizan sustancias químicas. Ambos tratamientos son eficaces y de la combinación de ambos resulta el procedimiento más recomendado para tratar la enfermedad.

Existen una gran variedad de terapias no farmacológicas, algunas de las más utilizadas son el entrenamiento y estimulación cognitiva, ejercicio físico, musicoterapia, etc. Sin embargo, la técnica más utilizada dentro del contexto que aquí estudiamos es la reminiscencia.

La reminiscencia es el acto o proceso de recordar recuerdos del pasado. Esto puede implicar el recuerdo de episodios particulares o genéricos que pueden o no haber sido olvidados previamente, y que son acompañados por la sensación de que estos episodios son relatos verídicos de las experiencias originales. Esta técnica es utilizada para estimular la memoria episódica autobiográfica mediante el encadenamiento de recuerdos, que se agrupan en categorías y se archivan en el tiempo mediante la elaboración de la historia de vida.

La historia de vida es una técnica narrativa que se basa en organizar y estructurar recuerdos de una persona para componer una autobiografía.

\section{Motivación}
\subsection{Explicaciones adicionales}
Si quieres cambiar el \textbf{estilo del título} de los capítulos, abre el fichero \verb|TeXiS\TeXiS_pream.tex| y comenta la línea \verb|\usepackage[Lenny]{fncychap}| para dejar el estilo básico de \LaTeX.

Si no te gusta que no haya \textbf{espacios entre párrafos} y quieres dejar un pequeño espacio en blanco, no metas saltos de línea (\verb|\\|) al final de los párrafos. En su lugar, busca el comando  \verb|\setlength{\parskip}{0.2ex}| en \verb|TeXiS\TeXiS_pream.tex| y aumenta el valor de $0.2ex$ a, por ejemplo, $1ex$.

El siguiente texto se genera con el comando \verb|\lipsum[2-20]| que viene a continuación en el fichero .tex. El único propósito es mostrar el aspecto de las páginas usando esta plantilla. Quita este comando y, si quieres, comenta o elimina el paquete \textit{lipsum} al final de \verb|TeXiS\TeXiS_pream.tex|

\subsubsection{Texto de prueba}


\lipsum[2-20]
\chapter{Estado de la Cuestión}
\label{cap:estadoDeLaCuestion}
En este capítulo se realizará una aproximación a las historias de vida como terapia de reminiscencia de la enfermedad de Alzheimer. A continuación, se presentarán las técnicas de generación de lenguaje natural con mayor relevancia como método alternativo de composición de dichas historias de vida. Para ello, navegaremos a lo largo de la historia de la generación, concretamente del modelado de lenguaje como motor del sistema. Se  abordarán las técnicas más antiguas de la generación, como el análisis de la arquitectura tradicional de un sistema de generación de lenguaje genérico, hasta los modelos de lenguaje más novedosos.%TODO

\section{Alzheimer e historias de vida}
La pirámide poblacional modifica su estructura continuamente debido al progresivo envejecimiento generalizado de la población. Según proyecciones de la \cite{alzheimers-association-media-line-2021}, en el año 2050 las personas mayores de 65 años constituirán el 16 por ciento de la población mundial frente al 8 por ciento del año 2010. El aumento de la esperanza de vida en todo el mundo, principalmente en las sociedades más avanzadas, y la disminución de la natalidad, se encuentran entre las causas de la modificación de la distribución demográfica hacia edades más avanzadas. Este fenómeno es conocido como \textit{inversión de la pirámide poblacional} \citep{RCSP892}.

La realidad detrás de estas estadísticas: el incremento del número de personas de edad avanzada, y asociándose al envejecimiento la acumulación a lo largo del tiempo de una gran variedad de daños moleculares y celulares que lleva a un descenso gradual de las capacidades mentales y físicas, deriva en un mayor riesgo de determinadas enfermedades. 

La pérdida de la audición, las cataratas, la artritis y la artrosis son solo algunas de las enfermedades con mayor incidencia. Sin embargo, una de las dolencias más comunes y serias dentro de este rango de población es la enfermedad de Alzheimer, cuya prevalencia a nivel global se espera que supere todo dato conocido hasta ahora, dado que se estima que en el año 2050 se incremente el número de casos a 152,8 millones, sobrepasando considerablemente los 57,4 millones del año 2019 \citep{alzheimers_disease_international_2019}.


\subsection{Descripción general}
La enfermedad de Alzheimer es un trastorno neurológico caracterizado por cambios degenerativos en diferentes sistemas neurotransmisores que abocan finalmente a la muerte de las células nerviosas del cerebro encargadas del almacenamiento y procesamiento de la información. Las regiones del cerebro involucradas con la memoria y los procesos de aprendizaje, asociadas a los lóbulos temporal y frontal, reducen su tamaño como consecuencia de la degeneración de las sinapsis y la muerte de las neuronas \citep{romano2007enfermedad,mattson2004pathways}. En las etapas finales de esta patología, este proceso, también denominado \textit{atrofia cerebral} se extiende y provoca una pérdida significativa del volumen cerebral (figura~\ref{fig:CerebroPersonaAlzheimer} a).


\figura{Vectorial/02EstadoDeLaCuestion/CerebroPersonaAlzheimer}%
{width=.8\textwidth}%
{fig:CerebroPersonaAlzheimer}%
{Reducción del cerebro asociada al Alzheimer \citep{mattson2004pathways}}

En numerosas ocasiones son utilizadas imágenes similares a las mostradas en la figura~\ref{fig:CerebroPersonaAlzheimer} como indicativos de la enfermedad del Alzheimer. La figura~\ref{fig:CerebroPersonaAlzheimer} b  representa unas \textit{tomografías por emisión de positrones} o \textit{PET scans} en inglés. En ellas se reflejan los patrones de distribución espacial de la glucosa en el cerebro. En el cerebro de la persona con Alzheimer, el flujo glucolítico cerebral se minimiza provocando los síntomas de la enfermedad. Esta prueba se utiliza en el diagnóstico de la gravedad de la patología.


El proceso de detección de la enfermedad de Alzheimer es una tarea ardua de realizar dado que, por lo general, los síntomas iniciales de la enfermedad suelen atribuirse a un olvido puntual o la vejez. Nada más lejos de la realidad. Según avanza la enfermedad, sus síntomas lo hacen con ella, agravándose y aumentando cada vez más hasta que el deterioro cognitivo ocasionado llega a afectar significativamente a las actividades de la vida diaria y finalmente a las necesidades fisiológicas básicas.

La evolución del Alzheimer se puede dividir en tres fases o etapas. En una primera instancia, se comienza a observar un deterioro cognitivo leve como puede ser la pérdida paulatina de la memoria episódica, seguido de pérdidas de la memoria reciente asociadas a un deterioro mayor así como otras funciones mentales y de la personalidad. Para terminar, se produce una pérdida progresiva de la memoria referida a los acontecimientos más antiguos, acompañando además un importante deterioro físico.


\subsection{Síntomatología y pérdida de la memoria}
La amnesia o pérdida de la memoria es uno de los síntomas más representativos del Alzheimer. Sin embargo, se trata tan solo de la punta del iceberg debido a todos los desordenes que también se producen y que no son considerados o tenidos en cuenta por el personal no profesional: alteraciones del estado de ánimo y la conducta, dificultad de toma de decisiones, desorientación, problemas del lenguaje, dificultad para comer, movilidad reducida  y un largo etcétera son algunos de los síntomas que acompañan a esta enfermedad durante todo su camino. Todos estos síntomas dependen de la fase evolutiva de la enfermedad.

Podemos distinguir en cuanto a sintomatología dos fases marcadas por las alteraciones neurológicas: en una primera fase, conocida como fase predemencial, los signos de desordenes neurológicos todavía no se encuentran presentes; y la fase demencial, en la que se pueden observar grandes alteraciones motoras, cognitivas, sensoriales y emocionales.

En la etapa predemencial, durante la cual en numerosas ocasiones el paciente no se encuentra diagnosticado de la enfermedad, comienzan a producirse lesiones microscópicas en el cerebro. Sin embargo, no es hasta entre 10 y 20 años después que pueden aparecer las primeras alteraciones cognitivas. El conjunto de síntomas presentes en esta fase comprende principalmente alteraciones en la conducta como trastorno de la personalidad, apatía o cambios en el estado de ánimo; y deterioro gradual de la memoria, comenzando el paciente a olvidar pequeñas cosas hasta llegar a no ser capaz de recordar familia o amigos.

A medida que progresa el daño cerebral aparece progresivamente un deterioro más pronunciado del paciente, comenzando entonce la fase demencial de la enfermedad. En esta etapa comienzan a aparecer alteraciones neurológicas como pérdida del movimiento, temblores, alucinaciones,  trastornos en el lenguaje oral y escrito o alteraciones de la personalidad \citep{alberca-serrano-2010}.


\subsection{Tratamientos: historias de vida}
En la actualidad el Alzheimer es una enfermedad irreversible. Sin embargo, existen diversos tratamientos disponibles para ralentizar el avance de la enfermedad, así como mejorar la calidad de vida de los pacientes. Estos tratamientos se pueden dividir en dos ramas diferenciadas: tratamientos farmacológicos o farmacoterapia, que hacen uso de medicamentos; y tratamientos no farmacológicos o psicosociales, que no hacen uso de sustancias químicas. Ambos tipos de tratamientos resultan eficaces para tratar la enfermedad de Alzheimer. Sin embargo, de la combinación de ambos resulta el procedimiento más recomendado debido a su mayor efectividad. Esto es posible gracias a que ambos tipos de tratamientos no son mutuamente excluyentes \citep{romano2007enfermedad}.

Existen una gran variedad de terapias no farmacológicas. Algunas de las más utilizadas son el entrenamiento y estimulación cognitiva, ejercicio físico o musicoterapia. Además, en cada una de estas terapias podemos encontrar una enorme cantidad de técnicas, siendo la reminiscencia la más utilizada como terapia de estimulación cognitiva.

Según \cite{o2013cross}, la reminiscencia es el acto o proceso de recordar sucesos, eventos o información del pasado. Esto puede implicar el recuerdo de episodios particulares o genéricos que pueden o no haber sido olvidados previamente, y que son acompañados por la sensación de que estos episodios son relatos verídicos de las experiencias originales. Esta técnica es empleada en la estimulación del la memoria episódica autobiográfica mediante el encadenamiento de recuerdos, que se agrupan en categorías y se archivan en el tiempo mediante la elaboración de la \textit{historia de vida}.

La historia de vida es una técnica narrativa que se basa en organizar y estructurar recuerdos de una persona para componer una autobiografía. Según \cite{linde1993life}, una historia de vida debe cumplir dos criterios: primero, debe incluir algunos puntos de evaluación que comuniquen los valores morales de la persona; y segundo, los eventos incluidos en la historia de vida deben tener un significado especial y ser de importancia para ella. Estos eventos deben ser aspectos significativos de la vida pasada de la persona, su presente y su futuro.

Para componer la historia de vida de una persona con Alzheimer se recopilan historias a través de familiares u otras personas cercanas. Posteriormente, se documentan en forma de un libro o cuaderno, incluyendo experiencias y logros junto con fotografías y escritos sobre hechos importantes para la vida de la persona, a través de los cuales se muestra quién es esa persona.

Cada persona tiene su propia historia de vida única. Nuestras experiencias nos modelan y construyen la persona que somos. Las historias de vida ayudan a las personas con Alzheimer a conectar con su identidad recordando épocas felices. El miedo y la frustración provocados por el olvido de las tareas de la vida cotidiana, nombres y rostros, se mitigan recordando quiénes eran a través de estas historias. Les ayuda a ser conscientes de los momentos especiales que han marcado su vida, las personas que han conocido en su infancia o trabajo. También pueden ser utilizados por los cuidadores para comprender más sobre ellos, quiénes son, y ayudarles en la reminiscencia de recuerdos \citep{karlsson2014stories}.


Existen diferentes formatos en los que se pueden registrar estas experiencias de la persona. Ninguno de ellos es mejor o peor que otro, sino que lo ideal es utilizar aquel que mejor se adapte a la persona y a los hechos que se quieran transmitir.

Por una parte encontramos historias de vida más visuales, compuestas enteramente de imágenes (\textit{collages}) o videos, dirigidas especialmente a las personas con Alzheimer que se encuentran en una etapa tardía de la enfermedad. Otro formato se centra especialmente en textos. Los \textit{libros de vida}, destinados a los cuidadores y visitantes tanto como a la propia persona, combina las \textit{historias de vida}, en forma de texto claro y fácil de leer, con algunas imágenes. También nos encontramos los documentos de perfil personal que se centran en pequeñas versiones cortas de las historias de vida excluyendo las imágenes. Estos documentos son utilizados a menudo en hospitales y están diseñados para ayudar al personal a comprender las necesidades de la persona.

El contenido de una historia de vida es variable, aunque existen algunos temas básicos en los que se debe centrar: el perfil de la persona, incluyendo datos e información básica como es el nombre, edad, lugar de nacimiento o de residencia son esenciales para aproximarse de manera inicial a la persona. Otros temas como las relaciones significativas familiares y de amistad, infancia, lugares y eventos significativos y gustos o preferencias y aficiones son incluidos dentro de esta lista de posibles temas a tratar el la historia de vida \citep{thompsonlifestory}.

\section{Generación de lenguaje natural}

La Generación de Lenguaje Natural (GLN) se define como el ``subcampo de la inteligencia artificial y la lingüística computacional que se ocupa de la construcción de sistemas informáticos que pueden producir textos comprensibles en inglés u otros lenguajes humanos a partir de alguna representación no lingüística subyacente de la información''  \citep{reiter1997building}. Si bien esta definición estuvo generalmente aceptada como la más conveniente al hablar de generación de lenguaje natural durante muchos años, \cite{gatt2018survey} puntualizan que es una afirmación que solo engloba una parte de la generación de textos, ya que se refiere únicamente a aquellos sistemas cuya entrada es una ``representación no lingüística [...] de la información'' o datos, como veremos más adelante en el apartado \ref{cap:nglD2T}.

Desde hace muchos años, la GLN es empleada en numerosos proyectos de distinta naturaleza como la traducción de textos \citep{Cho2014LearningPR}, realización de resúmenes y fusión de documentos \citep{clarke2010discourse}, corrección automática de ortografía y gramática \citep{islam2018bangla}, redacción de noticias \citep{leppanen2017data}, informes meteorológicos \citep{sripada2014case} y financieros \citep{ren2021hybrid}, generación de resúmenes sobre la información de recién nacidos en un contexto clínico \citep{BabyTalk}... Todos estos sistemas tienen en común la generación de un texto (normalmente de una alta calidad) a partir de muy diferentes fuentes de información.

En los ejemplos de proyectos listados con anterioridad que emplean la generación de lenguaje natural para redactar distintos textos, los datos utilizados como fuente de información son muy dispares, no solo en su contenido sino también en el tipo de dato. Así, si para la traducción de textos se utiliza texto ya existente como entrada, en otros sistemas como en la generación de informes meteorológicos se emplean datos no lingüísticos. De esta manera, se consideran dos posibles enfoques en los sistemas GLN dependiendo del tipo de entrada: texto a texto (\textit{text-to-text}) y dato a texto (\textit{data-to-text}).

\subsection{Generación \textit{text-to-text} (T2T)}
Los sistemas de generación texto a texto, conocidos como \textit{text-to-text} en inglés o T2T por sus siglas, toman textos escritos en lenguaje natural como entrada y producen un texto nuevo y coherente como salida. La entrada de estos sistemas puede abarcar desde pequeñas oraciones a extensos escritos. Existen muchas aplicaciones en los sistemas GLN que utilizan T2T. Además de los mencionados anteriormente, pertenecen a este tipo la fusión de documentos y generación de resúmenes \citep{clarke2010discourse}, simplificación de textos complejos \citep{sulem2018simple}, o autocorrectores gramaticales \citep{Ge2019AutomaticGE}, entre otros. 

Sin embargo, el ejemplo más claro de este tipo de generación de lenguaje corresponde a un traductor automático. Este tipo de sistema ampliamente utilizado en la vida cotidiana toma una entrada textual correspondiente a un escrito en un idioma y genera un texto de salida en otro idioma. La traducción automática es un proceso muy complejo, puesto que no solamente tiene en cuenta el significado del corpus, sino que también hace falta interpretar y analizar de manera correcta todos los elementos del texto, así como comprender la influencia de unas palabras en otras con la finalidad de  generar un texto fluido y coherente. 

\subsection{Generación \textit{data-to-text} (D2T)}\label{cap:nglD2T}
Estos tipos de sistemas permiten la generación de texto como salida a partir de entradas no textuales. Además, el formato de los datos que pueden tomar como entrada son muy diversos. Aunque es muy común encontrar sistemas que parten de datos numéricos como hojas de cálculo, hay que considerar otros orígenes de datos de tipo estructurado tales como bases de datos, simulaciones de sistemas físicos o grafos de conocimientos. De manera general, podemos referirnos a la representación de la información de esta clase de sistema como datos estructurados o procesables.

Algunos autores prefieren emplear  el término \textit{concepto} en lugar de \textit{data}, motivo por el que algunos se refieren a este enfoque como generación concept-to-text (C2T) \citep{vicente2015generacion}.

\begin{figure}[t]
	\centering
	%
	\begin{SubFloat}
		{\label{fig:inputFoG}%
			Entrada del sistema FoG}%
		\includegraphics[width=0.7\textwidth]%
		{Imagenes/Bitmap/02EstadoDeLaCuestion/inputFoG}%
	\end{SubFloat}
	\qquad
	\begin{SubFloat}
		{\label{fig:outputFoG}%
			Salida del sistema FoG}%
		\includegraphics[width=0.7\textwidth]%
		{Imagenes/Bitmap/02EstadoDeLaCuestion/outputFoG}%
	\end{SubFloat}
	\caption{Sistema \textit{data-to-text} FoG%
		\label{fig:FoG}}
\end{figure}

Uno de los ejemplos más visuales que nos permite comprender este tipo de sistema sería el \textit{Forecast Generator}, sistema que forma parte del \textit{Forecaster's Production Assistant}, entorno desarrollado por \textit{CoGenTex} en 1992 para \textit{Environment Canada} con el fin ayudar a los meteorólogos a aumentar su productividad al redactar por ellos un informe meteorológico textual en inglés y en francés \citep{goldberg1994using}. En la figura  \ref{fig:inputFoG} se muestra el entorno sobre el que los meteorólogos modifican valores como la presión atmosférica, situación de frentes y otros datos (datos no textuales). Una vez se pulsa sobre \textit{Generar}, el sistema muestra el texto correspondiente al informe (figura \ref{fig:outputFoG}). 


En la figura \ref{fig:data-to-text}, explicada con más detalle en \cite{sai2020survey}, se muestran los datos de entrada y de salida de un sistema GLN D2T acercándonos a la generación de lenguaje desde una perspectiva distinta al ejemplo explicado anteriormente.  Los datos de entrada de este tipo de sistema toman la forma de grafo o cualquier otro tipo de datos semiestructurados como tablas (conjunto de tuplas del tipo [entidad, atributo, valor]). En la fila inferior, se muestran diferentes posibles soluciones como salida del sistema. Además, el autor introduce la necesidad de métodos de evaluación de la calidad del texto redactado ya que de las diferentes salidas, solo la tercera opción cubre toda la información de entrada y resulta ser fluida.

\figura{Bitmap/02EstadoDeLaCuestion/data-to-text}{width=.9\textwidth}{fig:data-to-text}%
{Ejemplo de D2T utilizado por \cite{sai2020survey}}

\section{Arquitectura tradicional y herramientas de un sistema GLN}
\label{sec:arquitectura_tradicional}
El objetivo final de un sistema de generación de lenguaje natural es mapear unos datos de entrada a un texto de salida \citep{reiter1997building}. Sin embargo, este proceso, aunque pueda parecer sencillo de entender, resulta complicado de llevar a cabo. Al principio del desarrollo de sistemas GLN, no había un consenso entre autores a la hora de establecer un proceso para construir este sistema. Finalmente, \cite{reiter1997building} propusieron una arquitectura asociada a una lista de tareas recomendables que se deben realizar a la hora de llevar a cabo dicha construcción. Esta arquitectura surgió de la observación de los diferentes sistemas que se habían llevado a cabo hasta la fecha.  Actualmente, es la solución más extendida y reconocida.

La arquitectura presentada por \cite{reiter1997building}, como se puede observar en la figura\ref{fig:arquitectura}, se divide en tres módulos: macroplanificación, microplanificación y realización. Además, cada módulo contiene una lista de tareas. Esta asignación tareas-módulo no es inamovible. Una tarea asociada a un módulo se puede realizar en otro si así se considera, incluso implementar su desarrollo a lo largo de varios módulos. Los módulos que se corresponden con las tareas iniciales suelen estar relacionados con adaptar datos o estructura al sistema de generación, mientras que los módulos finales corresponden a la transformación de los resultados intermedios en el texto final.

\figura{Bitmap/02EstadoDeLaCuestion/arquitectura}{width=1\textwidth}{fig:arquitectura}%
{Arquitectura de referencia para sistema GLN \citep{vicente2015generacion}}

\subsection{Macroplanificación}
Este es el primer módulo de un sistema de generación de lenguaje. Debe determinar qué decir, seleccionando para ello la información de entrada necesaria y organizarla en una estructura coherente, resultando de este proceso el plan del documento. Las tareas que intervienen se describen en los apartados siguientes. 

\subsubsection{Selección del contenido}
La selección o determinación del contenido puede definirse como el proceso de decidir qué información debe ser incluida en el texto generado y cuál no. Por lo general, la información de la que partimos contendrá más información de la que nos interesa, así debemos decidir qué información resulta innecesaria y por tanto tenemos que eliminar para la generación del texto final. También hay que tener en cuenta el público al que está dirigido el texto generado, ya que dependiendo de este podremos incluir cierta información de los datos entrantes o no.

Este proceso de selección de la información lleva a cabo el filtrado y resumen de esta en un conjunto de \textit{mensajes}. Cada uno de estos mensajes corresponde al significado de una palabra u oración y se le asigna una entidad, concepto o relación dominante.

\subsubsection{Estructuración del documento}
Definiendo el concepto \textit{texto} como ``unidad de comunicación completa, formada habitualmente por una sucesión ordenada de enunciados que transmiten un mensaje con las siguientes propiedades: adecuación, coherencia y cohesión'', podemos advertir que un texto no es un conjunto aleatorio de oraciones, sino que es necesaria la existencia de un orden en la presentación del texto final.

Dependiendo de la información que se comunique, este orden puede verse modificado o alterado. Es por ello que no hay una estructura fija, sino que hay que adecuarla al tipo de documento.

Una vez realizada la estructuración del texto, se obtiene un plan de discurso que corresponde a una representación estructurada y ordenada de los mensajes obtenidos en la tarea anterior.

\subsection{Microplanificación}
La microplanificación es el segundo módulo de la arquitectura. Parte del plan del documento resultante del módulo anterior para generar las oraciones evitando información redundante e innecesaria en el discurso. El resultado de este módulo es el plan de discurso. El proceso de generación de oraciones lo realiza mediante tres tareas.

\subsubsection{Agregación de oraciones}
La generación de una oración por cada uno de los mensajes puede resultar en la generación de un texto redundante y excesivamente estructurado. Una tarea en el proceso de construcción de un sistema GLN es la agregación de oraciones que pretender paliar este problema mediante la unión o agregación de contenidos de distintos mensajes en una sola oración. De esta manera los mensajes se combinan para obtener oraciones más largas y complejas, resultando en conjunto un texto menos estructurado y más fluido.

\subsubsection{Lexicalización}
En esta fase del proceso se empieza a generar el texto en lenguaje natural como tal. Para ello se debe decidir que palabras o estructuras sintácticas expresan mejor los conceptos y relaciones de las etapas anteriores. La dificultad de la generación en esta etapa reside en la gran cantidad de alternativas que encontramos para  expresar cada uno de estos conceptos o bloques de mensajes. 

Además, hace falta tener en cuenta un todavía mayor número de posibilidades ya que se debe considerar numerosas variables que podrían afectar al resultado final de la generación. Algunas de las variables a tener en cuenta pueden ser las necesidades o conocimiento de los usuarios, si el objetivo de la generación es generar textos con variaciones sintácticas o semánticas a lo largo del mismo, si es preferible un texto repetitivo y simple o diverso mediante la utilización de palabras sinónimas y para terminar, una apropiada selección de adjetivos.


\subsubsection{Generación de expresiones de referencia}
La diferenciación de unas entidades de otras para poder generar expresiones que se refieran a ellas es tratada en esta tarea con el objetivo de evitar la ambigüedad. Para realizar esta tarea se debe conseguir encontrar características particulares que contribuyan a diferenciar a una entidad del resto de entidades. Esta etapa está bastante consensuada en el campo GLN. 

La generación de expresiones de referencia (REG, por sus siglas en inglés) debe llevarse a cabo una vez que el plan del documento se haya generado y depende de este, esto implica que esta fase debe llevarse a cabo desde el primer momento después de que se hayan analizado los datos. Debemos adaptar el plan de documento del primer módulo a lo que necesita REG, es por ello que debemos tener conocimiento de ello desde el comienzo.

Un caso especialmente estudiado que aplica esta técnica es la descripción de imágenes, ya que debe tener en cuenta si un elemento se encuentra a la derecha de otro, detrás de otro, etc, para poder enriquecer el texto. Para ello es necesario reconocer y distinguir los elementos en escena unos de otros y así, obtener una descripción lo más fidedigna posible a la imagen real.

\subsection{Realización}
La realización constituye el último módulo de la arquitectura de un sistema GLN. El objetivo final corresponde en generar oraciones gramaticalmente correctas para comunicar mensajes. En este módulo deberán tenerse en cuenta reglas a cerca de la formación de verbos (elección del tiempo verbal adecuado y por tanto generación de las palabras correspondientes), reglas sobre concordancia de género y número entre palabras (\cite{reiter1997building} no tienen en cuenta el género de las palabras ya que focaliza la generación del lenguaje al inglés), generación de pronombres...
 
La entrada sobre la que se trabaja es el plan de discurso que contiene información sobre las oraciones generadas y la estructura utilizada en el texto final. En esta fase se traduce esta entrada en la salida que el usuario final recibirá.

Algunos autores consideran una única tarea de realización que engloba el convertir las especificaciones en oraciones y el dar un formato final al texto. Otros prefieren separar estas etapas para diferenciarlas y que sea más sencillo su estudio.

\subsubsection{Realización lingüística}
Con el objetivo de transformar las especificaciones de oraciones en las oraciones finales, en esta fase se ordenan los diferentes elementos constitutivos de una oración y se les asigna un formato correcto. Para elegir la forma morfológica correcta de una palabra se debe conjugar verbos, establecer concordancias de palabras, añadir formas pronominales en los lugares adecuados de las oraciones y establecer los signos de puntuación adecuados. 

\subsubsection{Realización de la estructura}
Esta etapa no está considerada por algunos autores como tal aunque aquí se muestra ya que puede ser relevante en ciertos contextos. 
En algunos documentos, es necesario añadir o modificar algunas líneas del texto para darle estructura al documento. Un ejemplo muy sencillo de entender es la generación de texto que utilice html o Latex como formato de salida. En ambos casos, la adición de etiquetas a lo largo del texto generado resulta crucial para un texto de cualquiera de estas naturalezas.

\subsection{Herramientas en la arquitectura tradicional}
Los módulos constituyentes de la anterior arquitectura presentada pueden ser complicados de construir. Debido a la existencia  de profesionales que quieren desarrollar un sistema de generación de lenguaje completo, basándose en esta arquitectura genérica de sistema NLG, o investigadores que se encuentren estudiando sus propias implementaciones de alguno de estos módulos y deseen enfocarse plenamente en él, se crearon una serie de herramientas que pueden ayudarlos con el arduo trabajo de implementación de las tareas que componen la arquitectura. De esta manera, algunas de las herramientas presentadas a continuación pueden utilizarse para tomar inspiración del proceso de desarrollo, mientras que otras se construyen con el objetivo de relegar por completo este proceso en la propia utilidad. En este apartado se realizará un breve análisis de las herramientas consideradas más relevantes, asociadas a cada uno de los módulos descritos. Cabe decir, que debido a la antigüedad de esta arquitectura, la mayoría de las herramientas encontradas son aproximadamente de entre diez y veinte años atrás, sin embargo, algunas de ellas todavía hoy en día conservan gran fama y usuarios.


En primer lugar, para la macroplanificación de la arquitectura tradicional de un sistema NLG, dedicado a  la selección de la información de entrada necesaria para el sistema junto a su posterior organización, se encontraron diferentes herramientas de diversa índole. Algunas como SPUR se construyeron expresamente para la realización de estas únicas tareas de macroplanificación. Concretamente, esta herramienta se desarrolló como parte de un sistema mayor, conocido como MATCH \citep{johnston-etal-2002-match}, construido con el objetivo de realizar comparaciones entre restaurantes en Nueva York y obtener recomendaciones personalizadas. De esta manera, SPUR recibía como entrada los atributos de las opciones a comparar entre restaurantes y la salida correspondía al plan del documento constituido por los atributos más importantes de cada una de las opciones. Otras proyectos como PESCaDO \citep{pescado} o NaturalOWL \citep{Androutsopoulos_2014} además de tareas de planificación también incluyen componentes para el proceso de microplanificación e incluso realización .


En segundo lugar, para el módulo de microplanificación se desarrollaron herramientas como SPaRKy, utilizada también en el sistema MATCH anteriormente mencionado. En este caso, la entrada a la herramienta corresponde al plan de documento de la herramienta SPUR y su salida corresponde al plan de discurso con las afirmaciones a incluir en la salida del sistema. Otros proyectos mucho mas grandes como BabyTALK \citep{gatt_2009_babytalk} o NaturalOWL también incluyen entre sus funcionalidades herramientas para el desarrollo de la microplanificación.

Para acabar, son muchos los proyectos mencionado anteriormente que incluyen herramientas internas que realizan funciones de realización, sin embargo, destaca la existencia de la herramienta SimpleNLG, que no está ligada al empleo de un proyecto en concreto sino que puede utilizarse con cualquier tipo de datos.

SimpleNLG \citep{simplenlg_gatt} es una API de Java que proporciona interfaces que ofrecen un control directo sobre la tarea de realización. Define un conjunto de tipos léxicos, correspondientes a las principales categorías gramaticales, así como formas de combinarlos y establecer valores de características. Está orientada a la generación de oraciones gramaticalmente correctas en sistemas \textit{data-to-text}. Aunque originalmente solo estaba disponible para textos de lengua inglesa, actualmente se encuentra versionado para muchos idiomas, entre ellos el español. La versión española de esta herramienta es conocida como SimpleNLG-ES \citep{aramossoto2017adapting} y realmente se trata de una adaptación bilingüe de la versión original en inglés.

Esta herramienta se basa en la flexibilidad a la hora de generar textos mediante la utilización de manera combinada de sistemas basados en esquemas y otros enfoques más avanzados; la robustez generando salidas (aunque en ocasiones incorrectas) pese a que las entradas sean erróneas o incompletas; y por último, la independencia entre las operaciones de decisión de morfología y sintácticas.

Para definir una oración con SimpleNLG, se parte de sus constituyentes sintácticos básicos: sujeto, verbo y objeto. Una vez definido cada uno de ellos, por ejemplo estableciendo el sujeto como ``Manuel'', el verbo como ``bake'' y el objeto como ``a cake'', puede generarse una oración simple, siendo la salida resultante ``Manuel bakes a cake''. Cabe destacar la existencia de métodos que permiten establecer el tiempo verbal: pasado, presente o futuro, así como modificar el tipo de oración: enunciativa (por defecto), negativa o interrogativa. En el caso de este último tipo de oración, también permite seleccionar el tipo de pregunta formalizando el complemento circunstancial (quién, dónde, cómo...) o si debe formularse para que la respuesta sea sí o no. También permite establecer complementos a la oración y modificadores verbales. En la figura~\ref{fig:simplenlg} se puede apreciar a través del ejemplo presentado, el resultado de todas estas funcionalidades.


\figura{Bitmap/02EstadoDeLaCuestion/simplenlg}{width=1\textwidth}{fig:simplenlg}%
{Entrada y salida de la herramienta SimpleNLG}

\section{Algoritmos y modelos para la generación de lenguaje}
\label{sec:modelos}
Cualquier sistema informático dedicado al tratamiento del lenguaje natural necesita un modelo que le permita caracterizar y representar la lengua que trata. En esta sección se realiza una aproximación a los algoritmos y modelos utilizados a lo largo de la historia para la generación de lenguaje. Se comenzará estudiando a través del tiempo las distintas aproximaciones mayormente empleadas durante la época y se estudiará de una manera más profunda aquellos enfoques más relevantes en la actualidad como son los modelos estadísticas basados en cadenas de Markov o N-Grams, así como enfoques neuronales a través del estudio de diferentes tipos de redes. Para terminar, se presentarán algunas imprecisiones de generación que comenten las redes neuronales al ser aplicadas a la generación de lenguaje.


\subsection{Historia}
Antes de comenzar a describir los distintos tipos de modelos utilizados en tareas de generación de lenguaje natural, es interesante conocer como ha ido evolucionando este campo a lo largo de la historia y los modelos más relevantes en cada una de las etapas. En general, la historia del procesamiento de lenguaje natural se divide en dos grandes períodos marcados por la aparición del aprendizaje profundo o \textit{Deep Learning} \citep{louis-2021}. 

La era \textit{pre Deep Learning} (figura~\ref{fig:erapreDL}) comienza aproximadamente en el 1949, momento en el que   Warren Weaver sugería en su memorando ``Translation'' \footnote{ Lectura disponible en https://web.stanford.edu/class/linguist289/weaver001.pdf} que la traducción automática computacional era posible. Esta fue la primera aproximación estadística al procesamiento y generación de lenguaje. Supuso una revolución e inspiró numerosos experimentos y proyectos que probaron que realmente esto era posible aunque a muy pequeña escala. Estos sistemas se basaban principalmente en la búsqueda en diccionarios de las palabras necesarias para la traducción y la posterior reordenación de las palabras para ajustarse a las reglas sintácticas del idioma destino de la traducción.

Después de una década de investigaciones para conseguir mejores resultados en este campo y de pérdida de financiación, ya que las soluciones encontradas hasta ahora conseguían resultados muy pobres, Surgieron nuevas \textit{teorías de la gramática} mucho más manejables computacionalmente, y más tarde las \textit{ontologías conceptuales} que estructuraban la información del mundo real en datos comprensibles por la computadora.

En la década de 1980, surgieron los \textit{Modelos Simbólicos} basados en reglas. Estos sistemas asignaban manualmente los significados de las palabras y de esta manera determinista se creaban oraciones. Debido a la complejidad de creación de estas reglas, ya que se debían crear a mano, estos modelos fueron ampliamente sustituidos por los\textit{ Modelos Estadísticos} que supusieron una revolución para el procesamiento de lenguaje en aquella época y que hoy en día todavía tienen una gran relevancia en la lingüística computacional. 


\begin{figure}[t]
	\centering
	%
	\begin{SubFloat}
		{\label{fig:erapreDL}%
			Era pre \textit{Deep Learning}}%
		\includegraphics[width=0.9\textwidth]%
		{Imagenes/Bitmap/02EstadoDeLaCuestion/erapreDL}%
	\end{SubFloat}
	\qquad
	\begin{SubFloat}
		{\label{fig:erapostDL}%
			Era \textit{Deep Learning}}%
		\includegraphics[width=0.9\textwidth]%
		{Imagenes/Bitmap/02EstadoDeLaCuestion/erapostDL}%
	\end{SubFloat}
	\caption{Etapas del Procesamiento de Lenguaje Natural%
		\label{fig:etapasDL}}
\end{figure}

Con el avance computacional de las \textit{Redes Neuronales}, comienzan a usarse en la década del 2000 en el modelado del lenguaje para la generación de textos y dan lugar a la era \textit{Deep Learning} (figura~\ref{fig:erapostDL}). \cite{bengio_2000} propuso el primer modelo de lenguaje neuronal utilizando una Red Neuronal Prealimentada  \textit{(FeedForward Neural Network)} de una capa oculta. Otros autores sustituyeron progresivamente esta arquitectura de red por Redes Neuronales Recurrentes (\textit{Recurrent Neural Networks}) y Redes de Memoria a Corto Plazo (\textit{Long Short-Term Memory}) aunque los componentes básicos de la arquitectura original se encuentran todavía en la mayoría de modelos de lenguaje neuronales.

Más tarde \cite{collobertWeston} introdujeron el \textit{Aprendizaje Multitarea} al procesamiento de lenguaje, utilizando una Red Neuronal Convolucional (\textit{Convolutional Neural Network}) para conseguir que varias tareas de aprendizaje se resolvieran de manera simultánea, resultando en una mejora de la eficiencia. 

Tras varios avances, como la introducción de modelos \textit{Word Embeddings} o la adopción general de redes neuronales para el modelado de lenguaje, surge la arquitectura Secuencia a Secuencia (\textit{Seq2Seq}). Estos sistemas estaban compuesto por dos componentes clave: el codificador o \textit{encoder} y el decodificador o \textit{decoder}. La revolución que supuso esta arquitectura fue significativa y todavía se siguen utilizando.

En 2014, \cite{Bahdanau} introduce los \textit{mecanismos de atención} que alivia el problema de cuello de botella de los modelos predecesores, los \textit{Seq2Seq}.

La última innovación en el mundo del Procesamiento de Lenguaje son los grandes Modelos de Lenguaje Preentrenados (\textit{Pretrained Moldels}). Debido a todo el esfuerzo computacional de días, semanas e incluso meses que supone entrenar un modelo de lenguaje, se proponen una serie de modelos que ya tienen realizado este entrenamiento. La finalidad de estos sistemas es el ajuste o \textit{fine-tune} de los mismos de acuerdo al objetivo que se quiera conseguir.

\subsection{Modelos estadísticos}
Estos tipos de modelos utilizan técnicas estadísticas y reglas lingüísticas para aprender las distribuciones de probabilidad de las palabras pertenecientes a un conjunto finito conocido como \textit{vocabulario} y, de esta manera, generar secuencias de palabras específicas. Entre las técnicas más utilizadas y que mejores resultados han arrojado en este ámbito encontramos los modelos basados en cadenas de Markov y en N-gramas.

\subsubsection{Modelo Markov Chain}

Este modelo, introducido por el matemático ruso Andrey Markov en 1913, es un modelo estocástico discreto que describe una secuencia de posibles eventos. Aplicado a la generación de texto, podemos resumirlo en un sistema que se basa en una distribución de probabilidades aleatorias para generar la siguiente palabra a un grupo de palabras o secuencia. 

Para que un proceso se considere Markov debe satisfacer una condición conocida como la \textit{propiedad de Markov}. Esta propiedad establece que la probabilidad del siguiente evento (en el caso de generación de lenguaje, la siguiente palabra) depende únicamente del evento actual. Si se considera, como es el caso, que las palabras futuras no dependen en absoluto de las palabras previas, la secuencia de palabras generada evoluciona de manera no determinista y entonces puede considerarse que la palabra $X_n$ es una variable aleatoria. La colección de variables aleatorias es la definición de \textit{proceso estocástico}, y sirve como modelo para representar la evolución aleatoria de una frase a lo largo del tiempo \citep{MoyotlHernndez2016MtodoPA}.

La fórmula \ref{eqn:markov_property} representa los fundamentos de la \textit{ propiedad de Markov}, donde \textit{X} es una variable aleatoria (palabra) que toma un valor en el espacio de estado dado \textit{s} (vocabulario); y \textit{n} es una variable discreta que representa el paso de tiempo \citep{howell_2022}. Como se puede observar, suponiendo que nos encontramos en el paso de tiempo \textit{n}, las probabilidades teniendo en cuenta los estados de los eventos anteriores y la probabilidad teniendo en cuenta únicamente el estado actual para modelar la siguiente palabra son equivalentes, por lo que la información anterior al estado actual carece de relevancia para el cálculo de la probabilidad del siguiente evento.

\begin{equation}
\label{eqn:markov_property}
\begin{aligned}
P(X_{n+1} = s_{n+1}|X_{n} = s_{n}, X_{n-1} = s_{n-1}, ..., X_{0} = s_{0}) &= P(X_{n+1} = s_{n+1}|X_{n} = s_{n})
\end{aligned}
\end{equation}

Viendo un ejemplo concreto, en la figura~\ref{fig:ex_markov} se puede apreciar una cadena de Markov aprendida del corpus ``A man, a plan, a canal: Panama! A dog, a panic in a pagoda!''. Esta cadena está formada por un conjunto de estados, cada uno representa una palabra única del corpus, unidas por transiciones, que simbolizan la probabilidad de pasar de un estado a otro. 

\begin{figure}[!h]
	\centering
	%
	\includegraphics[width=0.8\textwidth]%
	{Imagenes/Bitmap/02EstadoDeLaCuestion/ex_markov}%
	\caption{Ejemplo de cadena de Markov \citep{make-school-no-date}%
		\label{fig:ex_markov}}
\end{figure}

Para generar una secuencia de palabras con esta cadena, se comienza eligiendo aleatoriamente un estado inicial, supongamos ``A'' como elección. A continuación, se deben tomar una de las transiciones posibles en base a una elección aleatoria. Esta elección será ``man,'' o ``dog,'' con probabilidad 1/2; digamos que tomamos ``dog,''. Ahora debemos escoger ``a'' ya que no tenemos más posibilidades y seleccionar la siguiente palabra en base a las probabilidades. Tenemos cuatro opciones con similar probabilidad igual 1/4: ``plan,'',``canal,'', ``pagoda!'' y ``panic''. Suponiendo que escogemos ``pagoda!'', hasta ahora se ha formado la frase ``A dog, a pagoda!''. Este proceso se puede realizar indefinidamente hasta que se decida parar o se llegue a un estado sin transiciones de salida.




Debido a las características particulares de la propiedad de Markov, este modelo es un modelo sin memoria, ya que se desprecian todos los estados anteriores, por lo que no se retiene información relevante de un texto como las posiciones de las palabras en una oración o la relación entre palabras. Sin embargo, precisamente por carecer de memoria, es simple de comprender y rápido de ejecutar \citep{fumagalli_2020}.




\subsubsection{Modelo N-gram}

Otro modelo estadístico de gran transcendencia es el modelo \textit{N-Gram}. Este modelo va un poco más allá del modelo \textit{Markov Chain}, ya que se basa en realizar una predicción estadística de una secuencia de palabras teniendo en cuenta un conjunto de palabras anteriores. Esta secuencia de N palabras es representada mediante un N-grama.
Así, este modelo trata de predecir el N-grama más probable dentro de cualquier secuencia de palabras dado el historial de las N-1 palabras anteriores. Para su mejor comprensión, se expone un ejemplo ejemplo concreto a partir del siguiente extracto de ``La vida es sueño'' de Calderón de la Barca:


\begin{verse}
	¿Qué es la vida? Un frenesí.\\
	¿Qué es la vida? Una ilusión,\\
	una sombra, una ficción,\\
	y el mayor bien es pequeño;\\
	que toda la vida es sueño,\\
	y los sueños, sueños son.\\
\end{verse}

Podemos construir N-gramas a partir del texto anterior teniendo en cuenta que un N-grama está formado por N palabras que aparecen consecutivas en el corpus. A continuación se muestran unigramas, bigramas y trigramas extraídos del texto.

\begin{verse}
	\textbf{Unigramas} \\
	\{ qué, es, la vida, un frenesí, una, ilusión, sombra, ficción, y, el, mayor,... \}\\
	\textbf{Bigramas} \\
	\{ qué es, es la, la vida, vida un, un frenesí, frenesí qué, una ilusión,... \}\\
	\textbf{Trigramas} \\
	\{ qué es la, es la vida, la vida un, vida un frenesí, un frenesí qué,...\}
\end{verse}


Este proceso se realiza de manera iterativa hasta llegar al final del corpus teniendo en cuenta que no se pueden repetir dos N-gramas iguales dentro de un mismo conjunto. Una vez construidos los N-gramas se puede calcular la probabilidad condicional mediante las siguientes fórmulas dependientes de la ocurrencia de un sub n-gram dentro del conjunto.

\begin{equation}
	\label{eq:probabilidadesngram}
	\begin{aligned}
		\textrm{\textbf{Unigrama}}\:\:\:\:
		P_{(W_i)} & = \frac{C_{(W_i)}}{N}   \\     
		\\
		\textrm{\textbf{Bigrama}}\:\:\:\:
		P_{(W_i|W_{i-1})} & = \frac{C(W_{i-1}W_i)}{C({W_{i-1}})}\\
		\\
		\textrm{\textbf{Trigrama}}\:\:\:\:
		P_{(W_i|W_{i-2}W_{i-1})} & = \frac{C(W_{i-2}W_{i-1}W_i)}{C({W_{i-2}W_{i-1}})}
	\end{aligned}
\end{equation}

De esta manera, para un modelo N-gram, se calcula la probabilidad condicional a partir de las n-1 palabras anteriores. Volviendo al ejemplo de ``La vida es sueño'', para conocer la probabilidad de que a la secuencia \textit{la vida} le siga la secuencia \textit{es}, calculamos la probabilidad condicional \textit{es|la vida}. Según las fórmulas anteriores, esta probabilidad es igual al número de ocurrencias de \textit{la vida es} dividido por el número de ocurrencias de la secuencia \textit{la vida}.

\begin{equation}
	\label{eq:probabilidadesngram_ejemplo}
	\begin{aligned}
		P_{(es|la\;vida)} & = \frac{C_{(la\;vida\;es)}}{C_{(la\;vida)}}  &= \frac{1}{3}
	\end{aligned}
\end{equation}

Para conocer la probabilidad de una secuencia completa deberíamos multiplicar las probabilidades de manera iterativa. Para el ejemplo anterior, la $P(la\:vida\:es)= P(la)\:x\:P(vida|la)\:x\:P(es|la\:vida)$ y generalizando la fórmula anterior  $P(w1, w2, w3)= P(w1)\:x\:P(w2|w1)x\:P(w3|w2)$.

La realización de este proceso de manera iterativa nos lleva a la generación de oraciones, párrafos o libros completos. Sin embargo, la dependencia de generación de la palabra siguiente a una secuencia dada con respecto al conjunto de palabras generadas inmediatamente anteriores puede llevar a generaciones erróneas que no tengan en cuenta otros contextos anteriores.

\subsection{Algoritmos de \textit{Deep Learning} par la generación de texto}
\subsubsection{Redes Neuronales Recurrentes (RNNs)}
\label{sec:rnn}
Las redes neuronales recurrentes o \textit{Recurrent Neural Networks (RRN)} son una clase especial de red neuronal profunda que nos permite analizar datos tratando la dimensión ``tiempo''.
Aunque este tipo de red aparece por primera vez en el 1982 introducida por \cite{firstrnn}, debido a los requisitos computacionales que necesitaban no se pudieron llevar a la práctica hasta muchos años más tarde; cuando llegaron los avances necesarios para su puesta en marcha. La principal área de aplicación de este tipo de algoritmo de \textit{deep learning} es la resolución de problemas que involucran datos secuenciales (y por tanto, temporales) como traducción automática, procesamiento de lenguaje natural, descripción de imágenes o reconocimiento de voz. 

Teóricamente, una red neuronal recurrente está formada por \textit{neuronas recurrentes}. Mientras que otros tipos de redes utilizan como función de activación de la neurona una función que actúa en una sola dirección, desde la primera capa de entrada hasta la última capa de salida; este tipo de redes también incluyen conexiones hacia atrás, proporcionando al sistema cierta memoria. En cada instante de tiempo llamado \textit{timestep}, cada neurona de la red recibe como entrada la salida de la capa anterior así como su propia salida del instante de tiempo anterior. Este procedimiento se puede expresar con la notación de la ecuación \ref{eq:activation_rnn}, donde $x = (x_1,...,x_T)$ representa la secuencia de entrada procedente de la anterior capa, $y$ la secuencia de salida de la capa actual, $W_x$ los pesos a aplicar a los datos de entrada procedentes de la salida la capa anterior, $W_y$ los pesos que se aplican sobre los datos procedentes de la salida de la propia capa obtenidos en el anterior instante de tiempo y \textit{b} un bias a partir del cual centrar los datos.

\begin{equation}
	\label{eq:activation_rnn}
	\Large
	y_{(t)} = f_{activation}(W_{x}X_{(t)} + W_{y}Y_{(t-1)}+b)
\end{equation}

Otra forma más intuitiva a través de la cual comprender este proceso que en realidad no dista demasiado del algoritmo de una red neuronal convencional, es desarrollando esta neuronal a través de los pasos de tiempo \textit{t}. En la figura ~\ref{fig:rnn_neuron} se puede comprobar el proceso de este algoritmo desde un punto de vista más esquemático. A la izquierda, se muestra la neurona recurrente sin desarrollar y a la derecha, la neurona desplegada en el tiempo \citep{neuron}.

\figura{Bitmap/02EstadoDeLaCuestion/rnn_neuron}{width=1\textwidth}{fig:rnn_neuron}%
{Neurona recurrente desplegada en el tiempo}

La parte de la neurona donde se preserva un estado a través del tiempo se denomina \textit{memory cell}. La finalidad de este componente es recordar información relevante sobre un estado anterior que recibieron para poder realizar predicciones más precisas.

Mediante la unión y configuración en capas de varias neuronas de este tipo, pueden llegar a construirse grandes redes neuronales de tipo recurrente. Estas redes no solo modifican, con respecto a una red neuronal convencional, el tipo de neuronas y conexiones entre ellas; sino también algoritmos internos que permiten su adecuado funcionamiento. En concreto, el procedimiento de \textit{Backpropagation} convencional se sustituye por una versión del mismo dependiente de la dimensión ``tiempo'', conocido como \textit{Backpropagration Through Time (BTTT)}. Este algoritmo mantiene la función tradicional del mismo, que no es otra que ir hacia atrás en la red con el fin de encontrar las derivadas parciales del error con respecto a los pesos de las neuronas. Estas derivadas son utilizadas en el \textit{descenso de gradiente} para ajustar los pesos dependiendo del comportamiento de \textit{Loss}. Sin embargo, debido a la inclusión del ``tiempo'' en este algoritmo, el coste computacional aumenta haciendo a este modelo mucho más lento.

El problema de este tipo de redes es conocido como \textit{Vanishing Gradients}. Como mencionamos anteriormente, de la aplicación del \textit{Backpropagration} se obtenían unas derivadas parciales del error, cada una de estas derivadas es un \textit{gradiente}. El problema de desvanecimiento de gradiente ocurre porque el gradiente se reduce a medida que se propaga hacia atrás a través del tiempo. Cuando los valores de un gradiente son extremadamente pequeños, estos valores no contribuyen al aprendizaje perdiendo peso en el resultado. 

\subsubsection{\textit{Long Short-Term Memory (LSTM)}}
Estas redes, propuestas por \citep{lstm} en el año 1997, surgieron como una evolución de las redes neuronales recurrentes. Su principal objetivo es ampliar la memoria para poder recordar no solo información reciente sino datos producidos mucho más tiempo atrás, ya que las redes neuronales recurrentes convencionales no eran capaces de recordar información que se había producido hacía varios \textit{timestep}; llevando a una memoria limitada para recordar secuencias de entrada más largas. Este problema es resultado del \textit{Vanishing Gradients} de los gradientes más lejanos en el tiempo.

\figura{Bitmap/02EstadoDeLaCuestion/lstm}{width=1\textwidth}{fig:lstm}%
{Neurona LSTM}

Para solventar esta limitación, las redes \textit{Long Short-Term Memory} proponen una variación de las neuronas. Estas neuronas poseían una memoria o \textit{memory cell} donde almacenaban la información relevante de estados anteriores dependiendo de los pesos calculados. En cada neurona LSTM existen tres puertas a esta celda: la puerta de entrada (\textit{input gate}), la puerta de olvidar (\textit{forget gate}) y la puerta de salida (\textit{output gate}). Estas puertas regulan el flujo de información dentro y fuera de la celda. Deciden si se permite una nueva entrada a la memoria, si se elimina la información o si se deja que afecte a la salida del instante de tiempo actual. Estas puertas podemos codificarlas mediante una función de activación sigmoide, lo que hace posible incluirlas en la \textit{Backpropagation} solucionando el problema de \textit{Vanishing Gradients}. Toda esta explicación está representada en la figura .


\subsubsection{Modelos Seq2Seq y mecanismos de atención}
El modelo Sequence-to-Sequence (Seq2Seq) caracterizado por la utilización de una arquitectura especial de Red Neuronal Recurrente (RNN), ha alcanzado un gran éxito a la hora de resolver problemas complejos de Procesamiento de Lenguaje Natural, incluso llegando a superar a los modelos estadísticos de lenguaje en su efectividad \citep{analytics_vidhya_2020}. Esto se debe a que aproximaciones estadísticas como los \textit{N-grams} no eran capaces de capturar dependencias de palabras de corpus de gran tamaño, se necesitaría demasiado espacio y memoria RAM para poder guardar las probabilidades de todas posibles combinaciones de N-gramas. Sin embargo, las redes neuronales recurrentes que implementa este modelo no están limitadas a observar únicamente las palabras previas a una secuencia, sino que permiten propagar información desde el comienzo de una oración hasta el final consiguiendo mejores predicciones.

\paragraph{Arquitectura Encoder-Decoder}\hfill


Considerando los diferentes tipos de redes neuronales recurrentes descritas en apartados anteriores, se da paso a la explicación propia del modelo Seq2Seq. Desde un punto de vista muy general, podríamos representar este modelo como un sistema que toma una secuencia de elementos como entrada (input) y genera otra secuencia de elementos de salida (output). Como se muestra en la figura \ref{fig:basicseq2seq}, la arquitectura de este sistema sigue una arquitectura \textit{Encoder-Decoder}, compuesta internamente por dichos componentes, un \textit{encoder} y un \textit{decoder} que implementan redes neuronales recurrentes, concretamente LSTM o en menor número de casos GRU (\textit{Gated Recurrent Units}).


\figura{Bitmap/02EstadoDeLaCuestion/basicseq2seq}{width=1\textwidth}{fig:basicseq2seq}%
{Arquitectura de un sistema Seq2Seq}

La tarea del \textit{encoder} consiste en resumir la información de la secuencia que se introdujo como entrada en forma de un vector de estado oculto o \textit{context} y enviar los datos resultantes al \textit{decoder}. El objetivo principal de este vector es encapsular la información de todos los elementos de entrada para ayudar al \textit{decoder} a realizar predicciones precisas. Para calcular el estado oculto t-ésimo de la secuencia se utiliza la fórmula representada en la ecuación \ref{eq:encoder}, donde  $x_t$ corresponde a la secuencia de entrada en el instante de tiempo \textit{t} y \textit{W} representa la matriz de pesos a aplicar sobre los datos de entrada $W^{hx}$ y sobre la salida de la celda del instante anterior $W^{hh}$. Para cada una de las celdas del \textit{encoder} se calcula su vector de estado oculto, generando la última celda (en el instante de tiempo \textit{t}) el vector de estados finales.

\begin{equation}
	\label{eq:encoder}
	\Large
	h_t = f(W^{(hx)}x_t+W^{(hh)}h_{t-1})
\end{equation}


Por su parte, el \textit{decoder} utiliza como estado inicial la salida del \textit{encoder} correspondiente al vector de estados finales, calculando cada celda su estado oculto con la fórmula \ref{eq:decoder}. Una vez que se obtiene el estado oculto $h_{t}$, puede generarse la secuencia de palabras final aplicando al dataset de palabras junto con $h_{t}$ la función \textit{softmax}.

\begin{equation}
	\label{eq:decoder}
	\Large
	h_t = f(W^{(hh)}h_{t-1})
\end{equation}

Aunque esta aproximación parece solucionar muchos de los problemas de modelos anteriores, añade o mantiene limitaciones. Una de ellas es el cuello de botella que se genera en el último estado oculto del codificador ya que toda la información de la entrada debe atravesar el \textit{encoder} hasta este último punto para poder pasarle toda la información junta al \textit{decoder}. Además, ya que se intenta mapear una secuencia de longitud variable en una memoria de longitud fija y en el caso de textos largos podría perderse parte de la información. 


\paragraph{Mecanismos de atención}\hfill

Ante los problemas mencionados anteriormente propios de los modelos secuencia a secuencia, se plantea la utilización de mecanismos de atención que permiten que el \textit{decoder} no tenga que recibir toda la información del \textit{encoder}, sino que se fija en aquellas palabras más importante que producen los estados ocultos de codificador en cada uno de sus pasos. Estos mecanismos de atención fueron introducidos inicialmente por \cite{Bahdanau} para la traducción automática aunque posteriormente se ha aplicado a una multitud de áreas. 


Para conseguir estos beneficios del mecanismo de atención, se modifica ligeramente la arquitectura del sistema añadiendo una capa intermedia entre el codificador y decodificador que recibe los estados ocultos que se van generando en el \textit{encoder}. Sin embargo, no se espera a que todos los estados ocultos estén calculados sino solo los más importantes a los que se les establece un mayor peso. Para que el almacenamiento de los estados ocultos no sea ineficiente, los estados ocultos recibidos se combinan en un vector llamado \textit{vector de contexto} que contendrá más o menos información de las palabras dependiendo de su peso (figura ~\ref{fig:basicseq2seq_attention2}). Estos pesos se calculan comparando el último estado oculto del \textit{decoder} con cada uno de los estados del codificador determinando así las palabras más importantes.


\figura{Bitmap/02EstadoDeLaCuestion/basicseq2seq_attention2}{width=1\textwidth}{fig:basicseq2seq_attention2}%
{Arquitectura de un sistema Seq2Seq con mecanismo de atención}


%\subsubsection{Gated Recurrent Unit (GRU)}
%\subsubsection{BiDirectional RNNs (BRNNs)}

\subsubsection{Modelos preentrenados: Transformers}
Los modelos preentrenados surgen como una evolución de los modelos Seq2Seq. Es por esto que aún podemos encontrar características de estos últimos sistemas
en la arquitectura externa de este tipo de modelo como en la existencia de algunos de los dos componentes clave: el \textit{encoder} y el \textit{decoder}. 

La necesidad de este tipo de modelo en el modelado de lenguaje viene del requerimiento de grandes bases de datos para entrenar modelos de redes recurrentes junto con su consecuente enorme tiempo para entrenar todo un corpus en el modelo. Además, los requisitos computacionales precisados en para el entrenamiento de estas colecciones de datos haría imposible esta tarea por parte de pequeños investigadores que no dispusieran de estos medios.

Los modelos preentrenados se basan en crear un modelo previamente capacitado para la realización de ciertas tareas generales de procesamiento de texto y que permiten ser ajustados o \textit{finetune} sobre bases de datos de pequeña escala. De esta manera, no se debe construir un modelo desde cero para resolver problemas de procesamiento de lenguaje.

Dentro de los modelos preentrenados, la arquitectura Tranformers está en el centro de casi todos lo desarrollos recientes más importantes en el campo de procesamiento de lenguaje natural. El rendimiento de esta arquitectura es mejor que el de las redes neuronales recurrentes y redes neuronales convolucionales (estás últimas serán explicadas posteriormente en al apartado~\ref{sec:cnn}).

Es posible acceder a la arquitectura de los Transformers a través de la herramienta de Python de nombre homónimo.
Esta herramienta proporciona una serie de sistemas de propósito general para Natural Language Understading (NLU) y Natural Language Generation (NLG). Ofrece más de 32 modelos preentrenados en más de 100 idiomas, entre los que se encuentra el español. Entre los modelos más utilizados encontramos los famosos GPT-2, BERT y T5 junto con un gran número de variaciones de ellos dependiendo de los datos utilizados para su preentrenamiento.

%TODO - HABLAR DE MASKED Y CASUAL GENERATION.

\paragraph{GPT-2}\hfill

GPT-2 (\textit{Generative Pretrained Transformer}) es un modelo de generación de lenguaje natural presentado por OpenAI en el año 2019 basado en redes neuronales para secuencias, basadas en la autoatención%TODO - REVISAR LO DE MASKED PORQUE CREO QUE ES CASUAL
 enmascarada(\textit{masked self-attention}), y que ha sido construido sobre una arquitectura Transformer. El objetivo de este sistema es construir una distribución de probabilidad en la que para cada palabra posible a generar se le asigna una probabilidad en función del contexto anterior. Se trata de un modelo que ha sido preentrenado con un conjunto de datos correspondiente a las 8 millones de páginas web mejor valoradas en Reddit, lo que resulta en una gran base de conocimiento para generar textos automáticamente de manera muy correcta.

La potencia de este modelo es tal que sus creadores no quisieron en un primer momento publicar la versión completa por miedo de que se pudiera utilizar de manera ilícita. Según fueron pasando los años, se fueron liberando progresivamente diferentes versiones del modelo original ya que comenzaban a surgir otros proyectos con potencias igualmente competitivas. Estas diferentes versiones se diferenciaban en el número de parámetros que admitía la arquitectura y de esta manera se conseguía limitar su funcionamiento. La primera versión contaba con 117 miles de millones de parámetros mientras que la última versión, publicada en 2020, posee 1,5 billones.

GPT-2 únicamente está disponible en inglés aunque puede hacer uso de GoogleTranslate API para generar textos en otros idiomas. Es importante resaltar que al depender del traductor se puede ver disminuida la calidad de generación de lenguaje.


\paragraph{BERT}\hfill

BERT (Bidirectional Encoder Representations from Transformers) es un modelo NLP desarrollado por Google y publicado a finales de 2018 \citep{Devlin2019BERTPO}. Está basado redes neuronales bidireccionales que tratan de predecir las palabras perdidas (enmascaradas) en una oración y determinar si dos oraciones consecutivas son continuación lógica entre sí para determinar si están conectadas por su significado. Aunque originalmente no estaba destinado a la generación de textos, \cite{wang-cho-2019-bert} publicaron un método de utilización de este sistema para conseguir la generación de lenguaje que parece dar muy buenos resultados. De hecho, consiguió mejorar los resultados de la versión publicada en aquellos tiempos por GPT-2. 

De este modelo han surgido numerosas variaciones que se han publicado a lo largo de estos años. Como se puede apreciar en la figura \ref{fig:modelos_bert}, encontramos distintas ramificaciones que podemos dividir en dos grupos: modelos preentrenados con un corpus específico perteneciente a un dominio y modelos \textit{fine-tuned} que se ajustan a una tarea específica utilizando un modelo previamente entrenado \citep{rajasekharan_2019}.
Otras variaciones de BERT corresponden a los modelos construidos a partir de él pero entrenados en otros lenguajes para generar textos en otra lengua distinta al inglés, que es la original. Beto es la versión en Español de BERT \citep{CaneteCFP2020} y ha sido entrenado con una gran corpus en dicho idioma.

\figura{Bitmap/02EstadoDeLaCuestion/modelos_bert}{width=0.9\textwidth}{fig:modelos_bert}%
{Modelos surgidos a partir de BERT}





\subsection{Otras técnicas de \textit{Deep Learning} aplicadas al modelado de lenguaje}

En este apartado se van a describir algoritmos de redes neuronales que no fueron utilizados en un inicio a la generación de texto pero que actualmente se han estado aplicando al campo con la finalidad de obtener mejores resultados y estudiar como afectarían las ventajas de este tipo de redes a la generación de lenguaje. Entre todas las técnicas que se han encontrado, se han seleccionado las dos más relevantes: Redes Neuronales Convolucionales y Redes Neuronales Generativas Adversarias.

\subsubsection{Redes Neuronales Convolucionales (CNNs)}
\label{sec:cnn}
Las redes neuronales convolucionales o \textit{Convolutional Neural Neetworks} son uno de los algoritmos de mayor éxito en la visión artificial. Este tipo de redes se ha empleado en diversas áreas de inteligencia artificial con imágenes incluyendo la clasificación de fotografías o la detección y reconocimiento de objetos en imágenes. Debido al gran éxito que ha tenido este tipo de algoritmo en la visión artificial, se ha exportado a otros subcampos de la inteligencia artificial como es la generación de lenguaje natural, logrando resultados muy interesantes. 


Una red neuronal convolucional es un tipo de red neuronal profunda caracterizada por la utilización de \textit{convoluciones} en al menos una de sus capas ocultas. Este tipo de arquitectura red fue introducida por primera vez en 1990 por \citep{LeCun1989HandwrittenDR}, quien adaptó la operación de convolución para las redes neuronales con el objetivo de reconocer dígitos escritos a mano. Más tarde, modificó su sistema para aplicarlo a cualquier tipo de imagen y no solo reconocimiento de números \citep{cnn_lecun}.

Las redes convolucionales reciben, al igual que las redes neuronales convencionales, una entrada. En este tipo concreto de red, la entrada generalmente se tratará de una imagen o bajo el punto de vista del computador, una matriz de píxeles. Así, la altura y ancho de la matriz de entrada dependerá de la resolución de la imagen y la profundidad, de si se trata de una imagen en blanco y negro (con profundidad igual a uno) o si por el contrario, está compuesta por colores RGB, en cuyo caso la profundidad de la matriz será igual a tres.

Una vez preparados los datos de entrada, son procesados a lo largo de todas las capas que componen la red neuronal convolucional, que en este tipo de red corresponde a las conocidas como \textit{capas de convolución}. La función de esta capa es extraer las características de los datos presentados como entrada dependiendo del filtro seleccionado, comprimiéndolas con el objetivo de reducir su tamaño inicial. 

Una vez presentada la arquitectura general de este tipo de red, resulta interesante estudiar el funcionamiento de capa una de estas capas de convolución que la componen. Para el ejemplo de la figura ~\ref{fig:cnn_conv1} y~\ref{fig:cnn_conv2} , se supone como datos de entrada una imagen de dimensiones 28x28x1.  

Par realizar la primera convolución (primera capa de la red), se van tomando pequeños grupos de píxeles cercanos y se aplican operaciones matemáticas de baja computación (producto escalar) contra una pequeña matriz denominada \textit{kernel}. El resultado obtenido después de este proceso es una matriz denominada \textit{convolución del kernel}, matriz a la que posteriormente se le aplicará la función de activación de neuronas \textit{ReLu (Rectifier Linear Unit)} para obtener el mapa de detección de características o {feature map}. Sin embargo, no solo un único \textit{kernel} será aplicado a la matriz de entrada de la capa convolucional de la red, sino un conjunto de ellos. Este conjunto de \textit{kernels} que operan sobre la matriz de entrada recibe el nombre de \textit{filtro}.

Por cada uno de los \textit{kernels} aplicados a la matriz de entrada obtendremos un mapa de características diferente lo que se traduce en un incremento enorme en el número de neuronas utilizadas y que posteriormente deban ser procesadas por otras capas, teniendo que aumentar considerablemente el esfuerzo computacional requerido. Para solucionar este problema, se aplica al conjunto de mapas de características un método de muestreo conocido como \textit{Max-Pooling} que comprime el conjunto de matrices preservando las características más importantes detectadas. Este proceso se puede comprender de manera más gráfica en la figura~\ref{fig:cnn_conv1}.

\begin{figure}[!h]
	\centering
	%
	\includegraphics[width=0.8\textwidth]%
	{Imagenes/Bitmap/02EstadoDeLaCuestion/cnn_conv1}%
	\caption{Primera Convolución de Red Neuronal Convolucional%
		\label{fig:cnn_conv1}}
\end{figure}

La segunda convolución (así como el resto de convoluciones), tomaría como entrada una matriz con toda la información de salida de la anterior convolución y realizaría el mismo proceso descrito anteriormente (figura~\ref{fig:cnn_conv2}). Cabe decir, que a lo largo de una red convolucional podemos tener varias capa de convolución en la que cada una aplica un filtro diferente. Existen una multitud de filtros que se podrían aplicar a la matriz de entrada. Entre los filtros destacan aquellos que permiten detectar los bordes de los objetos en una imagen (filtro derivador) o formas geométricas. Otro filtros conocidos son filtros que permiten reducir el ruido de una fotografía. 

\begin{figure}[!h]
	\centering
	%
	\includegraphics[width=0.8\textwidth]%
	{Imagenes/Bitmap/02EstadoDeLaCuestion/cnn_conv2}%
	\caption{Segunda Convolución de Red Neuronal Convolucional%
		\label{fig:cnn_conv2}}
\end{figure}

Para finalizar, la salida de la última capa de convolución correspondiente a una matriz (potencialmente de profundidad igual a 3), se aplana para convertirla en una capa de neuronas tradicionales y se conecta a una capa oculta de neuronas de tipo \textit{feedforward}. Posteriormente, se le aplica la función \textit{softmax} a esta red neuronal tradicional, resultando en la capa de salida del sistema. Esta arquitectura completa se puede visualizar en la figura~\ref{fig:cnn_arq}.


\begin{figure}[t]
	\centering
	%
	\includegraphics[width=0.8\textwidth]%
	{Imagenes/Bitmap/02EstadoDeLaCuestion/cnn_arq}%
	\caption{Arquitectura de una CNN%
		\label{fig:cnn_arq}}
\end{figure}

En la aplicación de este tipo de red al procesamiento de lenguaje natural podríamos distinguir dos vertientes: sistemas \textit{text-to-text} en los que, a diferencia de los problemas de visión por ordenador en los que se utiliza como entrada la matriz de píxeles mencionada, las matrices estarían compuestas de palabras, oraciones o caracteres dependiendo del problema a resolver. Así, cada fila de la matriz representará a una palabra u oración. Estos sistemas suelen estar destinados a la clasificación de textos u oraciones  \citep{jacovi2018understanding,Lai2015RecurrentCN,Kim2014ConvolutionalNN}.

Otro campo de la generación de lenguaje en el que está incluyendo este tipo de arquitectura es la generación de descripciones de imágenes \citep{He2017DeepLF} que genera un texto en lenguaje natural representativo de la información contenida en una imagen dada como entrada. Estos últimos sistemas suelen estar compuestos por dos componentes, una red neuronal convolucional que extrae las características representativas de la imagen y una red neuronal recurrente vista en el apartado \ref{sec:rnn}, que transforma estas características a un texto en lenguaje natural. La entrada de este tipo de sistema estaría formada por una matriz de píxeles y la salida sería el texto descriptivo de la imagen.

\subsubsection{Redes Neuronales Generativas Adversarias (GANs)}
Las redes neuronales generativas adversarias o antagónicas son un sistema de aprendizaje no supervisado en el que dos inteligencias compiten entre sí para lograr un objetivo. Este tipo de red se ha convertido en un gran éxito en el mundo de la visión artificial, especialmente conocidas por la generación de imágenes hiperrealistas como el sistema \textit{StyleGAN}, que hace uso de este tipo de red para generar imágenes de alta resolución de rostros de personas \citep{Karras2019ASG}. Más reciente es su aplicación al mundo de la generación de lenguaje natural, empleándose especialmente en la generación de texto basada en estilos.

Las GANs internamente están formadas por dos componentes: el \textit{generador} cuya tarea es generar ejemplos ya sea creándolos él mismo (ejemplos falsos) o extrayéndolos de un conjunto de datos (ejemplos reales); y el \textit{discriminador} que clasifica los ejemplos recibidos en falsos o reales según se mencionó anteriormente.

Así, el objetivo del generador es producir ejemplo cercanos a los ejemplos reales de manera que pueda engañar al discriminador. Por otra parte, el discriminador debe clasificar de manera correcta estos datos producidos por el generador en los dos tipos de datos. Para que el generador mejore, produciendo cada vez ejemplos más cercanos a las soluciones reales, recibe continuamente retroalimentación desde el discriminador sobre qué tan bien las muestras generadas lograron engañarle.


Todo este procedimiento se basa en teoría de juegos, en el que la función objetivo es la función \textit{minimax}. Esta función es mostrada en la ecuación~\ref{eq:gan_minimax}, donde $D(x)$ representa la probabilidad de que $x$ sea real según el discriminador y $G(z)$ es una muestra producida por el generador. Intuitivamente se deduce que el discriminador trata de maximizar la función objetivo (que los datos reales se identifiquen como reales y los falsos como falsos), mientras que el generador trata de minimizarla (engañar al discriminador).

\begin{equation}
	\label{eq:gan_minimax}
	\underset{G}{\min}
	\underset{D}{\max}
	V(D,G) = E_{x~P_{data}(x)}[logD(x)] + E_{z~P_{z}(z)}[log(1-D(G(z)))]
\end{equation}


La utilización de este tipo de red para el modelado de lenguaje es un desafío debido a la naturaleza no derivable de los símbolos discretos empleados en este tipo de generación, mientras que en la visión artificial se emplean datos continuos como las imágenes. Algunos sistemas que tratan de resolver este obstáculo se basan en técnicas como la \textit{Maximun Likelihood Estimation} (MLE) \citep{li-etal-2017-adversarial}, el algoritmo \textit{Reinforce} o \textit{Policity Gradient} \citep{Yu2017SeqGANSG}, estos dos últimos basados en aprendizaje por refuerzo \citep{li-etal-2018-paraphrase, Shi2018TowardDT}.


\subsection{Efectos de las aplicación de redes neuronales a la generación de texto}

Con el avance de los modelos de generación de lenguaje natural, se ha empezado a prestar más atención a las limitaciones y riesgos potenciales de este tipo de sistemas. Los sistemas más modernos y en los que los investigadores fijan principalmente su atención son modelos de \textit{Deep Learning} basados esencialmente en redes neuronales profundas que han sido capaz de mejorar drásticamente la calidad de generación de lenguaje respecto a otros sistemas anteriores. Sin embargo, junto con estas mejoras, debido a las características intrínsecas de estos modelos computacionales, estos modelos son más propensos a fenómenos que conllevan una generación errónea de textos. Por una parte la llamada \textit{degeneración} produce salidas incoherentes o atascada en bucles repetitivos de palabras o expresiones. Otros modelos GLN en algunas ocasiones generan textos de salida sin sentido alguno o con datos para nada respaldados en la información introducida como entrada. Este fenómeno es conocido como \textit{alucinación} y perjudica seriamente la aplicabilidad de los modelos neuronales de generación de lenguaje en casos prácticos donde la precisión de la información es vital y el nivel de tolerancia hacia las alucinaciones es nulo.

\subsubsection{Alucinaciones}

Con \textit{alucinación} nos referimos al fenómeno en el que un modelo, especialmente de tipo neuronal ``\textit{end-to-end}'', produce información de salida que no es fiel a los datos provistos como entrada al sistema. 

%Ejemplos: Caso de imagenes, caso de texto
Este fenómeno se da en una diversidad de sistemas condicionales de generación de lenguaje. \cite{hallucinations_data2text} en su artículo, \textit{Controlling Hallucinations at Word Level in Data-to-Text Generation} destaca la existencia de alucinaciones en la generación \textit{data-to-text} en un modelo neuronal entrenado a partir de bases de datos como \textit{Totto} \citep{parikh-etal-2020-totto}. La entrada al sistema es una tabla. Una vez generado el texto de salida se puede comprobar que la palabra ``Italian'' a la que denomina \textit{enunciado divergente} no es respaldada por los datos de entrada (figura~\ref{fig:d2thallucinations}).

Por otro lado, \cite{rohrbach-etal-2018-object} subraya la existencia de alucinaciones en la generación de descripciones de imágenes. Estos tipos de sistemas se componen de dos modelos diferenciados. Por una parte, un modelo de predicción de imagen que trata de extraer los objetos de la misma y por otra, un modelo de predicción de lenguaje basado en la probabilidad de la siguiente palabra a generar. De esta forma, se analizaron las diferencias de predicción entre ambos modelos (figura~\ref{fig:imagehallucinations}) y llegaron a la conclusión de que en la mayoría de los casos la descripción generada se basaba principalmente en el modelo de lenguaje con el objetivo de conseguir una descripción más consistente semántica y sintácticamente. En el caso de estudio, la imagen sirve de entrada al sistema y se comprueba la predicción de ambos modelos nombrados anteriormente para la última palabra a generar. Mientras que el modelo de imagen predice palabras como ``bol'', ``brocoli'' o ``zanahoria'', el modelo de lenguaje propone ``tenedor'', ``cuchara'' o ``bol''. Finalmente, la descripción generada utiliza ``tenedor'' para completar la frase aunque no aparece en la imagen produciéndose una alucinación.


\begin{figure}[t]
	\centering
	%
	\begin{SubFloat}
		{\label{fig:d2thallucinations}%
			Alucinaciones en generación DT2}%
		\includegraphics[width=0.7\textwidth]%
		{Imagenes/Bitmap/04Analisis/alucinacion_d2t}%
	\end{SubFloat}
	\qquad
	\begin{SubFloat}
		{\label{fig:imagehallucinations}%
			Alucinaciones en generación de descripciones de imágenes}%
		\includegraphics[width=0.8\textwidth]%
		{Imagenes/Bitmap/04Analisis/alucinacion_image}%
	\end{SubFloat}
	\caption{Alucinaciones en distintos sistemas%
		\label{fig:hallucinations}}
\end{figure}


%Caso especial medico -> mucha precision y privacidad

%Tipos de alucinaciones: intrinsecas| extrinsecas

Aunque nos referimos a las alucinaciones de manera general como datos generados erróneamente. Atendiendo al resultado de la generación y por tanto a las consecuencias que puede tener esta generación, \citep{hallucination_survey} distingue dos tipos de alucinaciones.

Con \textit{alucinaciones intrínsecas} (figura~\ref{fig:inthallucinations})se refiere a la generación de textos de salida que contradicen los datos de entrada. Mientras que las \textit{alucinaciones extrínsecas} (figura~\ref{fig:exthallucinations}) son aquellas que generan una salida que no puede ser verificada a partir de los datos de entrada. Ambos tipos de alucinaciones generan datos no respaldados por la información que constituye los datos de entrada. Sin embargo, este último tipo de alucinaciones no siempre genera una salida errónea ya que no se puede asegurar que los datos generados sean incorrectos.


\begin{figure}[t]
	\centering
	%
	\begin{SubFloat}
		{\label{fig:inthallucinations}%
			Alucinación intrínseca}%
		\includegraphics[width=0.7\textwidth]%
		{Imagenes/Bitmap/04Analisis/alucinacion_intrinseca}%
	\end{SubFloat}
	\qquad
	\begin{SubFloat}
		{\label{fig:exthallucinations}%
			Alucinación extrínseca}%
		\includegraphics[width=0.7\textwidth]%
		{Imagenes/Bitmap/04Analisis/alucinacion_extrinseca}%
	\end{SubFloat}
	\caption{Tipos de alucinaciones%
		\label{fig:hallucinations_types}}
\end{figure}

%Tipos de alucinaciones: A partir de datos por los datos de entrenamiento e inferencia

Si nos preguntamos el porqué de la existencia de las alucinaciones cuando se introducen unos datos de entrada a un sistema entrenado, potencialmente bajo un modelo de red neuronal, de manera satisfactoria y haciendo uso de una base de datos que nos permite realizar la tarea que tenemos como objetivo; tenemos que tener en cuenta las posibles causas origen que generan este problema.

La generación errónea de los datos de salida, según \cite{hallucination_survey}, puede deberse a una divergencia en los datos utilizados para entrenar el modelo. Esta divergencia aparece cuando la relación entre los datos fuente-referencia está mal construida. Aunque el modelo base funcione correctamente, el modelo que ha sido entrenado bajo esta base de datos con divergencias puede alentar a generar una salida que no es fiel a los datos proporcionados como entrada. Otro escenario problemático emerge con la existencia de ejemplos de datos del conjunto duplicados y que han filtrados de una manera incorrecta. Cada vez más, los corpus de texto se incrementan en tamaño con el paso del tiempo y debido a la imposibilidad de revisión humana de todos estos grandes conjuntos de datos, pierden calidad respecto a los corpus más pequeños. \cite{lee2021deduplicating} afirma que el 10\% de los ejemplos de las bases de datos más empleadas en generación de lenguaje natural están repetidas en numerosas ocasiones. También destaca, que cuando estos ejemplos de datos duplicados pertenecientes a un conjunto se utilizan para entrenar un sistema, sesga el modelo para favorecer la generación de estas frases duplicadas. Si además también participaran de divergencias en entre la fuente y la referencia de los datos, la existencia de alucinaciones se multiplicaría.


Otras de las razones de la existencia de las alucinaciones, corresponde a las características propias del modelo de red neuronal. Aún partiendo de una base de datos perfecta, sin duplicados ni divergencias algunas, las opciones de entrenamiento y modelado de estos sistemas influirían generando textos de salida incorrectos. Por una parte, la incapacidad de comprensión del modelo de los datos de entrada debido a la generación de correlaciones incorrectas entre las diferentes partes de los datos de entrenamiento por parte del codificador, puede conllevar a un mal aprendizaje por parte del modelo. Así mismo, la estrategia de decodificación utilizada, correspondiendo en estos casos la elección a estrategias que añaden aleatoriedad o diversidad en la generación, están relacionadas directamente con el incremento de las alucinaciones.

\subsubsection{Degeneración}

Algunos modelos de redes neuronales conocidos, como GPT-2, se basan en la aleatoriedad de la salida como su objetivo principal frente a la maximización de la probabilidad. Esto se debe a que buscan la mayor similitud en la generación entre el procesamiento textual de la información por parte de un sistema y un humano. Para conseguir esta diversidad en la salida de la generación, se hace uso de estrategias de decodificación aleatorias, ya que las estrategias que buscan la maximización de la probabilidad para obtener mayores puntuaciones de similitud, especialmente en el caso de los textos largos, con frecuencia abocan a textos con repetitivos e incoherentes. Un ejemplo de estrategia de decodificación que alienta a degeneraciones es \textit{Beam Search}. En la figura~\ref{fig:degeneracion}, se muestra el texto de salida generado por GPT-2 que utiliza esta estrategia de decodificación. Destaca en color azul las repeticiones producidas en la salida, claro ejemplo de degeneración.

\figura{Bitmap/04Analisis/degeneracion}{width=.7\textwidth}{fig:degeneracion}%
{Ejemplo de degeneración con Beam Search }

\subsubsection{Falta de representación de los datos de entrada}

Esta limitación podría considerarse justo la contraria a las alucinaciones. Si en esta última se generaba mayor información de la proporcionada en los datos de entrada, también se da el caso de falta de representación de alguno o todos los datos de entrada. Se trata de un problema igual de grave que si se generaran datos erróneos (como las alucinaciones intrínsecas) en las que en el caso de una resolución médica, una mala interpretación de los datos de entrada podría llegar a poner en peligro la vida de una persona. En este caso, dependiendo del grado en que no aparezcan representados en la salida los datos introducidos como entrada, se tendría un impacto de diferente gravedad. Si unos datos muy importantes (en el ejemplo del caso médico) no se tuvieran en cuenta, podría llegarse también a un diagnóstico potencialmente diferente al que se generaría si se hubiera incluido este dato. Así mismo si el número de datos no introducidos fuera elevado. En el caso de no aparición en la salida de datos poco relevantes o de perder poca información, el impacto que supondría sería leve.







\section{Proyectos relacionados}
Muchos son los enfoques que se han estudiado para tratar de perfeccionar la generación de lenguaje natural. Los más tradicionales seguían la metodología presentada en el apartado \ref{sec:arquitectura_tradicional} de este documento. En ella dividían el problema principal en varios subproblemas o tareas. Entre ellas se incluía la selección de contenido, estructuración del texto, agregación, lexicalización, generación de expresiones de referencia y finalmente, la realización \citep{reiter1997building}. Sin embargo, en los últimos años ha crecido el interés por mirar más allá de aquella arquitectura. Los sistemas presentados en la sección \ref{sec:modelos} surgieron para romper con ella.

Todos estos sistemas presentados anteriormente tienen muy poca capacidad o poca calidad a la hora de generar textos de manera controlada. Algunos de ellos como GPT-2 generan textos a partir de una oración inicial, resultando su salida un texto de tamaño variable que da continuidad a la oración de entrada. Otros como BERT son capaces de generar palabras perdidas dentro de una oración. Sin embargo, pocos de ellos son capaces por sí mismos de generar contenido de manera controlada a partir de una información dada en todos los puntos de su generación.

 
Por todo esto, han surgido varios sistemas como DICE \citep{yang2020creative} que utilizarán estos modelos ajustándolos con el objetivo de controlar la generación del texto resultante. Como se muestra en la figura \ref{fig:DICE}, este sistema está compuesto por dos capas. La primera capa toma como entrada unas palabras clave introducidas por el usuario y forma un Grafo de Conocimiento. A continuación y para conectar ambas capas, utiliza tripletas como interfaz. Estas tripletas se pueden construir a partir del grafo o extraerse de un corpus de historias denominado ROCStory. Estas tripletas sirven como entrada a un sistema de generación de texto que utiliza GPT-2 para generar pequeñas historias.

\figura{Bitmap/02EstadoDeLaCuestion/DICE}{width=0.9\textwidth}{fig:DICE}%
{Arquitectura del sistema DICE}

Otro enfoque parecido utiliza una conocida dataset llamada WebNLG para entrenar un Modelo de Lenguaje. Esta base de datos contiene correspondencias entre textos y tripletas y es utilizada para ajustar el Modelo de Lenguaje T5. De esta manera, es capaz de generar textos a partir de tripletas introducidas como entrada al sistema. En la figura \ref{fig:T5}, podemos ver el formato de entrada.

\figura{Bitmap/02EstadoDeLaCuestion/T5}{width=0.9\textwidth}{fig:T5}%
{Entrada y salida del sistema T5}

Con respecto a propuestas dentro del ámbito de generación de lenguaje en terapias de reminiscencia, no son muchos los sistemas que encontramos. Por una parte, \citep{MoralesdeJess2020ACM} proponen desarrollar e implementar un modelo conversacional que pueda ayudar a los cuidadores y a sus propios pacientes con Alzheimer a realizar un mayor número de terapias de reminiscencia periódicas para así potenciar los beneficios de estas. Se centra en generar conversaciones personalizadas entre el prototipo del sistema conversacional y el paciente con el fin de recoger información relacionada con sus gustos, historial y estilo de vida. Su arquitectura, como se muestra en la figura \ref{fig:ConversationalModelArchitecture}, esta integrada por varios módulos: módulo de Reconocimiento Automático de Voz, Comprensión del Lenguaje Natural, Gestión de Diálogos, Modelo de Diálogos, Generación de Lenguaje Natural y Text-to-Speech.


\figura{Bitmap/02EstadoDeLaCuestion/ConversationalModelArchitecture}{width=.8\textwidth}{fig:ConversationalModelArchitecture}%
{Arquitectura modelo conversacional \citep{MoralesdeJess2020ACM}.}

En otra investigación (realizada por \cite{shi2012user}), se propone el desarrollo de un sistema computarizado llamado Life Story Book (LSB), que facilita el acceso y la recuperación de recuerdos almacenados que se utilizan como base para interacciones positivas entre ancianos y jóvenes, y especialmente entre personas con deterioro cognitivo y miembros de su familia o cuidadores. Para facilitar la gestión de la información y la generación dinámica de contenido, este artículo presenta un modelo semántico de LSB que se basa en el uso de ontologías y algoritmos avanzados para la selección de características y la reducción de dimensiones. Para terminar, propone un algoritmo llamado Onto-SVD que combina la selección de características semánticas y la ontología orientada al usuario con  la utilización de SVD como método de reducción de dimensiones para lograr la identificación de temas basada en la similitud semántica.


\chapter{Objetivos del trabajo y planteamiento de la solución}
\label{cap:objetivosDelTrabajoYPlanteamientoDeLaSolucion}

En el presente capítulo se detallarán los objetivos y requisitos clave del sistema a realizar en este trabajo junto con las restricciones a tener en cuenta. A continuación, se describirá el planteamiento inicial del sistema a través del estudio de una arquitectura general y la presentación de las herramientas que se van a necesitar durante el desarrollo. Para finalizar este punto, se realizará un exhaustivo estudio de las opciones disponibles para la implementación del sistema y se expondrán las cuestiones sobre las que se apoyan las decisiones de implementación tomadas.



\section{Objetivos y requisitos}
\label{sec:ObjetivosYRequisitos}
Pese a los avances informáticos de las últimas décadas, la composición de historias de vida sigue sustentando su proceso en medios rudimentarios.
La dificultad de composición es palpable debido a todas las tareas a realizar, generalmente, a través de medios humanos. Como paso previo a la redacción final de la historia de vida, debe realizarse una selección adecuada de los datos que se van a incluir sin dejar ninguna información relevante atrás. También hace falta concebir la estructura del escrito y que datos corresponden a cada una de dichas partes. Si se consigue hacer este proceso de manera satisfactoria, se puede comenzar con la escritura de la historia de vida. Quienes escriben estas historias suele ser personal terapéutico, y dado que escribir no es para nada una labor sencilla, pueden tener ciertas dificultades a la hora de buscar combinaciones perfectas de palabras que expresen ciertos conceptos, perder tiempo leyendo una y otra vez una misma oración que no parece convencerles o simplemente quedarse sin ideas. Si existiera un sistema de composición automático de estas historias, podrían ver su labor facilitada dado que la tarea de redacción humana sería sustituida por un procedimiento automático más rápido. Sin embargo, la existencia de sistemas de generación de historias de vida, o similares como notas biográficas, es prácticamente nula. Es por ello por lo que el sistema que se pretende desarrollar en este trabajo tratará de responder a este problema en particular.


Después de analizar y evaluar la problemática en cuestión, se puede resumir el objetivo del sistema como el desarrollo de un sistema de generación de lenguaje que tome como entrada unos datos correspondientes a información sobre la vida de una persona y sea capaz de transformarlos en un texto redactado en lenguaje natural. De la anterior descripción se derivan dos requisitos claros para nuestro sistema. Por una parte, la capacidad de conversión de datos a textos y por otra, la comprensión y representación de datos de tipo biográfico. Cabe mencionar que también deberán verse reflejados en el escrito final otros datos no puramente biográficos como pueden ser anécdotas personales, gustos, relatos de situaciones y un largo etcétera.

Finalmente, el sistema a desarrollar deberá estar libre de alucinaciones, de degeneraciones o, de cualquier otro tipo de fenómeno que perjudique la representación de los datos en lenguaje natural y la comprensión del escrito. También se considerará la generación de un texto fluido y coherente como resultado de la conversión de estos datos proporcionados como entrada a texto.

Para terminar esta sección, se plantean una serie de limitaciones con las que va a contar el sistema debido generalmente a restricciones ajenas al trabajo. Por una parte, como se comprobará más adelante, la consideración de generación de datos en castellano no se realizó debido a la inexistencia de estos, por lo que el sistema partirá y generará información en inglés, ya que en esta lengua el acceso a recursos es mucho más alcanzable. Otra limitación viene dada por la disponibilidad básica de recursos computacionales. En general, el entrenamiento de grandes cantidades de datos en un modelo neuronal necesita costosos computadores y una gran cantidad de tiempo que no se puede llegar a considerar en el ámbito de un trabajo de estas características. Por estas razones, algunas tecnologías deberán ser descartadas.



\section{Planteamiento inicial}
\label{sec:PlanteamientoInicial}

En este apartado se presenta toda aquella información imprescindible para comenzar a desarrollar el sistema de este trabajo. Primero, se expondrán las herramientas principales que se van a utilizar durante el desarrollo de este. Para finalizar, se considerará la estructura de la arquitectura del sistema general que va a estar compuesto necesariamente de varios componentes, cada uno dedicado a la realización de una tarea en concreto.


\subsection{Software y herramientas}

Antes de desarrollar el sistema, es conveniente presentar que software y herramientas van a ser necesarias durante el proceso de desarrollo del sistema.

Como primera consideración, la biblioteca de software \textit{HuggingFace} será utilizada para la implementación del modelo de lenguaje y la descarga de algunos de los conjuntos de datos utilizados, dado que pone a la disposición del usuario una gran cantidad de funcionalidades además de las ya mencionadas. Otro de los puntos a favor de usar esta biblioteca, es la integración de esta herramienta con las bibliotecas de aprendizaje profundo \textit{PyTorch} y \textit{TensorFlow} que nos permitirán realizar los cálculos del sistema a través de tensores en la GPU. Esta herramienta proporciona tecnologías de procesamiento de lenguaje natural de código abierto que giran especialmente en torno a la biblioteca \textit{Transformers}. Esta última funcionalidad permite el acceso a una gran cantidad de modelos pre-entrenados y sus variaciones en forma, tamaño y arquitectura para distintas tareas, facilitando la construcción del modelo.

Para finalizar, se utilizará el entorno virtual conocido \textit{Google Colaboratory} (\textit{Google Colab} en su forma abreviada), dada la facilidad de ejecución y de programación a través del navegador sin necesidad de configuración previa y
con acceso gratuito a GPUs. La versión utilizada de este software dispone de las siguientes especificaciones: procesador Intel(R) Xeon(R) CPU a 2.3GHz de frecuencia, 12 GB de memoria RAM y una GPU modelo Tesla T5 con 16 GB de memoria.



\subsection{Arquitectura del sistema}
Para construir el sistema completo que se pretende desarrollar en este trabajo, se han considerado las siguientes capas a implementar tal cual están representadas en la figura \ref{fig:arquitectura}.

\begin{figure}[!h]
	\centering%
	\includegraphics[width=1.0\textwidth]%
	{Imagenes/Bitmap/03Objetivos/arquitectura}%
	\caption{Arquitectura del sistema%
		\label{fig:arquitectura}}
\end{figure}



\begin{itemize}
\item \textbf{Capa de preprocesado de datos} \hfill

Dado que los datos de entrada que componen la información a representar en la historia de vida, por lo general, no se encuentran en un formato adecuado, se implementará esta capa cuyo objetivo será precisamente la conversión de esta información a datos manejables por el modelo de lenguaje. Entre las tareas de esta capa destaca la creación, a partir de los datos de entrada, de un grafo de conocimiento del que posteriormente se extraerán tripletas semánticas que representen su información. Otras tareas a realizar por esta capa será la conversión de estas tripletas a un formato adecuado para que el modelo de lenguaje pueda procesarlas.


\item \textbf{Modelo de lenguaje} \hfill

El modelo de lenguaje es el núcleo de nuestro sistema. Para que dicho modelo pueda desempeñar las tareas correspondientes de generación, se seleccionará un modelo pre-entrenado basado en redes neuronales idealmente alineado a los objetivos del trabajo y se ajustará bajo un conjunto de datos limpio y con un formato legible por el modelo. De esta manera, formaremos un modelo sobre el que se introducir unos datos de entrada y que potencialmente generará una información coherente y fluida en lenguaje natural.


\item \textbf{Capa de postprocesado de texto}\hfill

En esta capa se precisan las tareas de selección de los datos que conformarán la salida del modelo, así como el orden y estructura que deben seguir. Una vez generado el texto por el modelo de lenguaje, la información obtenida deberá ser transformada a un texto completo, coherente y legible por el usuario. Así, toda la información obtenida deberá unirse en una sola información a presentar si no estuviera ya estructurada de dicha manera. Esta capa podría contar de manera opcional con algún componente de traducción automática sencillo que tradujera los resultados al castellano, ya que el texto resultante de todo el proceso será inglés.

\end{itemize}
Atendiendo a los módulos que componen la arquitectura de nuestro sistema, el núcleo del trabajo gira en torno a la generación de texto a través del modelado de lenguaje. Como se expuso en las conclusiones generales (sección \ref{sec:2conclusionesgenerales}), la estructura más conveniente para nuestro modelo de lenguaje corresponde a implementaciones neuronales de modelos de lenguaje, concretamente modelos pre-entrenados \textit{Transformers}.  

La decisión de utilizar este tipo de modelo deriva esencialmente de las restricciones computacionales de las que partimos. Por una parte, si tuviéramos que entrenar una red neuronal bajo una gran cantidad de datos, el computador requerido necesitaría una serie de características computacionales enormes y no se dispone de ellas. Además, el tiempo necesario para llevar a cabo este entrenamiento sería considerable. Ejemplos de estas limitaciones podrían ser grandes modelos de lenguaje como \textit{GPT} que tomó un mes entero de entrenamiento sobre ocho GPUs. 

Seleccionado el tipo de modelo, componente núcleo de nuestro sistema para la generación de texto, debemos escoger los datos que utilizaremos para ajustarlo y el modelo concreto que utilizaremos. Para ello se realizaron varias pruebas y se tomó la decisión en base a aquel que en mejores resultados originara. Las decisiones tomadas después de la realización de diversas pruebas se presentan en las secciones siguientes.

\section{Conjuntos de datos y preparación para el entrenamiento}
\label{sec:ConjuntosDeDatosYPreparacionParaElEntrenamiento}

En este apartado se presentan soluciones de datos efectivas para cubrir los requisitos de generación del sistema. Entre estos requisitos se encuentra la generación de textos que representen textualmente información bibliográfica de una persona. Otra de las necesidades a cubrir es la generación textual a partir de unos datos precisados como entrada al sistema. 

Los modelos pre-entrenados si bien no necesitan enormes conjuntos de datos para realizar un entrenamiento desde cero, deben ser ajustados a tareas específicas haciendo uso de conjuntos de datos de tamaños mucho más pequeños.

Son muchas las \textit{datasets} existentes para realizar tareas de ajuste de modelos. Algunos corpus como \textit{News Aggregator}\footnote{Disponible en Kaggle https://www.kaggle.com/datasets/uciml/news-aggregator-dataset} pueden ser utilizados para la creación de noticias. Otra tarea puede ser la generación de recetas, utilizando para ello \textit{datasets} como \textit{recipe-box}\footnote{Disponible en GitHub https://github.com/rtlee9/recipe-box}.
Es importante realizar una correcta elección del conjunto de datos sobre el que se va a entrenar el modelo. La elección del conjunto de datos no es trivial, sino que obedece a las necesidades establecidas: en nuestro caso, generación de texto biográfico a partir de unos datos especificados como entrada. 
Así, se realizó una búsqueda de conjunto de datos apropiados a las características necesarias para construir el sistema propuesto en este trabajo, resultando esta búsqueda en tres grandes conjuntos de datos: \textit{Wiki2bio}, \textit{WebNLG} y \textit{KELM}, que serán descritos a continuación.



\subsection{Wiki2bio}

Wiki2bio es un conjunto de datos propuesto por \citep{lebret-etal-2016-neural}. Fue creado debido a la necesidad de existencia de una gran conjunto de datos que permitiera la redacción notas biográficas. Hasta entonces, las bases de datos con información bibliográfica eran demasiado pequeñas para entrenar un modelo de red neuronal, por lo que se ideó la construcción de un conjunto de una orden de magnitud superior. Compuesto por más de 700.000 ejemplos y un vocabulario de 400.000 palabras, extrajeron todos estos datos mapeando los datos contenidos en las tablas de información de Wikipedia con los textos descriptivos escritos en lenguaje natural.

Antes de realizar cualquier entrenamiento en el modelo bajo un conjunto de datos es necesario realizar un análisis y preprocesamiento previo que nos permita conocer más a fondo dicho conjunto. 

Existen diferentes métodos para obtener datos de Wiki2bio. La manera más sencilla para descargar el conjunto de datos completo es hacer uso de las API de \textit{HuggingFace} o \textit{TensorFlow}, ya que devuelven los datos encapsulados en un objeto procesable con un conjunto de herramientas adecuadas para la tarea. En este caso, nos quedaremos con la primera opción, ya que posteriormente emplearemos los modelos de lenguaje de esta misma API. 

El conjunto completo de datos se descarga finalmente a través de la biblioteca \textit{datasets} de HuggingFace\footnote{Disponible en https://huggingface.co/datasets/wiki\_bio} que devuelve un objetivo de tipo \textit{DatasetDict}. El archivo final ocupa 738.19 MB y está compuesto por un total de 728.321 muestras.
El conjunto de datos se divide aleatoriamente en tres subconjuntos: conjunto para entrenamiento (80\%), conjunto para validación (10\%) y conjunto para prueba (10\%), constando cada uno de ellos de 582.659, 72.831 y 72.831 ejemplos, respectivamente.

Un ejemplo de los cualesquiera presentes en este conjunto de datos es representado en la figura~\ref{fig:ex_wiki2bio}. Como se puede apreciar, la información de cada uno de los ejemplos se divide en dos grupos. Por una parte, el \textit{input\_text} o texto de entrada especifica el contexto del que se están proporcionando los datos y el cuadro de información de Wikipedia en formato tabular. Los datos contenidos en esta tabla establecen asignaciones a características del contexto como son la nacionalidad, fecha de nacimiento o trabajo. El otro grupo corresponde a la información de destino o \textit{target\_text} que representa en lenguaje natural la información contenida en el contexto y la tabla anterior.

\begin{figure}[!h]
	\centering%
	\includegraphics[width=0.9\textwidth]%
	{Imagenes/Bitmap/03Objetivos/wiki2bio}%
	\caption{Ejemplo de entrada de wiki2bio%
		\label{fig:ex_wiki2bio}}
\end{figure}

Después de descargar con éxito el \textit{dataset} completo, hemos realizado una limpieza de los datos que componen el conjunto ya que se pudo apreciar que, en numerosas ocasiones aparecen saltos de línea o caracteres como paréntesis y corchetes codificados como símbolos en mitad de palabras u oraciones y esto podría llevar a la generación de datos incorrectos e imprecisos después del entrenamiento que se va a realizar. Para realizar esta limpieza programamos un \textit{script} en \textit{Python} en el que, a través de la aplicación de operaciones de expresiones regulares, se consiguieron unos datos mucho más legibles y apropiados para utilizarlos en el entrenamiento. El resultado de esta limpieza de los datos llevada a cabo se muestra en la figura \ref{fig:ex_wiki2bio_clean}.

\begin{figure}[!h]
	\centering%
	\includegraphics[width=1.0\textwidth]%
	{Imagenes/Bitmap/03Objetivos/wiki2bio_limpieza}%
	\caption{Ejemplo de wiki2bio original y limpieza de datos
	\label{fig:ex_wiki2bio_clean}}
\end{figure}


\subsection{WebNLG}
Introducido por \citep{gardent2017creating}, el corpus WebNLG se compone de un conjunto de tripletas semánticas, que describen hechos en forma de entidades y relaciones entre ellas, y los eventos correspondientes presentados en lenguaje natural. Este conjunto de datos nace de la explosión de desarrollo de bases de datos RDF (\textit{Resouce Description Format}), cuya entidad atómica de datos es precisamente la tripleta semántica. Una tripleta está formada por un conjunto de tres entidades que codifica una relación entre datos semánticos, en el caso de WebNLG, expresada de la forma <sujeto - predicado - objeto>. Para la creación de este conjunto de datos se parte de la base de conocimiento DBPedia construida a partir de datos estructurados de Wikipedia. Según mencionan sus autores, este \textit{dataset} se crea con la finalidad de suplir las tareas de lexicalización, agregación de oraciones y realización de la arquitectura tradicional presentada en la sección~\ref{sec:arquitectura_tradicional}.

Este \textit{dataset}\footnote{Disponible en https://gitlab.com/shimorina/webnlg-dataset/-/tree/master/} está disponible en inglés, aunque la tercera versión del conjunto de datos también incluye una colección de datos en ruso (es necesario precisar que los conjuntos de datos en idiomas diferentes se encuentran separados). Un total de 45.050 oraciones componen WebNLG, dividiéndose a su vez en tres subconjuntos de datos: \textit{train}, \textit{dev}  y \textit{test}, conteniendo cada uno 35.426, 4.464 y 5.150 oraciones, respectivamente. 

\begin{figure}[!h]
	\centering%
	\includegraphics[width=1.0\textwidth]%
	{Imagenes/Bitmap/03Objetivos/webnlg_ex}%
	\caption{Ejemplo de WebNLG
	\label{fig:webnlg_ex}}
\end{figure}


El formato de un ejemplo del \textit{dataset} se encuentra en la figura~\ref{fig:webnlg_ex}. Como se puede comprobar, cada ejemplo está formado por diversos campos. En primer lugar, aparece una serie de información acerca de los datos que nos vamos a encontrar en el ejemplo como el id, categoría, forma, tipos de estructura de las tripletas y número de estas. La categoría representa el tipo de datos de la entidad representada en la DBPedia y el 'eid' un id único en el subconjunto de datos y categoría para el ejemplo. Además, se muestra el número de tripletas en el conjunto, representado por el campo 'size'. Por otra parte, se tiene en cuenta la forma del árbol formado por el conjunto de tripletas en el campo 'shape', que representa dicho árbol en forma de cadena formado por paréntesis anidados donde X es un nodo, este formato se basa en la representación de árboles de Newick \citep{cardona2008extended}.%%TERMINAR
En el campo 'shape\_type' se indica si el objeto de una tripleta es el sujeto de otra (\textit{chain}) si las tripletas tienen un tema compartido (\textit{siblings}) o si se dan ambos casos (\textit{mixed}). 



Finalmente, se muestra la información más interesante de este conjunto de datos. Junto con unos ids y la calidad de las lexicalizaciones (\textit{good} o \textit{bad} en el campo 'comment') aparecen diferentes oraciones (campo 'text') que representan en lenguaje natural la información dada por los conjuntos de tripletas ('modified\_triples\_sets' y 'original\_triple\_sets'). Es destacable la existencia de varias oraciones para representar la misma información de un solo conjunto de tripletas y de varios conjuntos de tripletas representantes todas de la misma información. 


Dado que el formato en el que descargamos los ficheros que componen esta \textit{dataset} es XML (Lenguaje de Marcado Extensible), se cargaron los datos y se pasaron a un objeto que pudiera entrenar el modelo, en este caso un \textit{DataFrame} que finalmente incluía 73.119 ejemplos para el entrenamiento. Posteriormente, se modificaron los datos para crear la entrada definitiva al modelo uniendo toda la información de entrada en una sola cadena de caracteres. Esta entrada, conformada por las tripletas, se estructura de la siguiente manera: objeto1 | relacion1 | atributo1 \&\& objeto2 | relacion2 | atributo2 ...; separando cada tripleta por un doble \textit{``et''} y cada entidad por una barra vertical. 

A continuación, se utilizaron expresiones regulares para limpiar cada uno de los ejemplos que componían el \textit{dataset} ya que incluía demasiados símbolos que podrían dificultar su comprensión. Las tareas de limpieza llevadas a cabo comprendieron la eliminación de símbolos tales como guiones bajos, acentos circunflejos, comillas dobles, entre otros, y la división de las relaciones según las letras mayúsculas y estableciéndolas en minúsculas. Este proceso será descrito en más profundidad en el apartado \ref{sec:webnlg_limpieza}. Después de procesar los datos para asignar a cada uno de los conjuntos de tripletas cada una de las soluciones en lenguaje natural, el resultado final se muestra en la figura~\ref{fig:webnlg_ex_own}.


\begin{figure}[!h]
	\centering
	%
	\includegraphics[width=1\textwidth]%
	{Imagenes/Bitmap/03Objetivos/ex_webnlg_own}%
	\caption{Ejemplo procesado de WebNLG%
		\label{fig:webnlg_ex_own}}
\end{figure}


\subsection{\textit{Knowledge-Enhanced Language Model} (KELM)}
KELM (\textit{Knowledge-Enhanced Language Model}) es un enorme corpus de datos en inglés construido específicamente para un sistema \textit{data-to-text} llamado TeKGen (\textit{Text from KG Generator}) propuesto por \cite{agarwal-etal-2021-knowledge}. Este sistema se basa en un modelo secuencia a secuencia para generar texto en lenguaje natural a partir de un grafo de conocimiento o \textit{Knowledge Graph} (KG) presentado a través de tripletas semánticas. La motivación de construcción de este conjunto de datos surge de la necesidad de variedad de entidades y relaciones que, según sus autores, WebNLG no tenía. De acuerdo con las cifras que mencionan, los datos de Wikidata (base de conocimiento con datos estructurados que sirve como conjunto de datos secundario para dar soporte a Wikipedia y como la que se ha construido el corpus KELM) podrían representarse en aproximadamente 6 millones de entidades y 1500 relaciones, frente a las 600 entidades y 20 relaciones de WebNLG.


Para crear este enorme corpus de datos, sus autores utilizaron supervisión a distancia para alinear las tripletas de Wikidata con el texto de Wikipedia. Para cada una de las entidades de Wikidata, se seleccionaron como oraciones candidatas aquellas que aparecen en la primera sección de su página de Wikipedia. A continuación, emparejaron con cada frase de esta sección las tripletas que contenían la entidad como sujeto. De esta manera, una tripleta puede alinearse con varias oraciones de la sección y a su vez, una sección puede alinearse con varias tripletas. Como observación, realizaron una heurística de sustitución de pronombres ya que en numerosas ocasiones las oraciones contenían este tipo de elemento sintáctico.
 
Finalmente, el conjunto de datos resultante está compuesto, según las cifras oficiales proporcionadas por sus creadores, por más de 18 millones de oraciones, 45 millones de tripletas y 1500 relaciones diferentes.


%Datasets such as WebNLG have instances with grouped triples that can be expressed as a fluent sentence. Such groups are not available for a large KG and using one triple at a time for inference would lead to hallucination as training uses multiple triples per example. Therefore, we develop a strategy to create entity subgraphs based on relation co-occurrence counts i.e. frequency of alignment of two relations to the same sentence in the training data. The algorithm is shown in Figure 4. It produces ∼18M entity subgraphs from ∼45M triples so the final corpus will have 18M generated sentences corresponding to each entity subgraph.


Al igual que las bases de datos anteriores, KELM se encuentra disponible para descargar desde la API de HuggingFace\footnote{Disponible en https://huggingface.co/datasets/kelm} y ocupa en total de 3.08 GB de tamaño y 7.964.073 ejemplos en total. Está dividida en 3 subconjuntos de datos: entrenamiento, prueba y validación. El número de ejemplos de cada uno de los subconjuntos mencionados anteriormente son 6.371.131, 796.471 y 796.471, que corresponde respectivamente al 80\%, 10\% y 10\% del número de ejemplos totales del conjunto de datos. 

\begin{figure}[!h]
	\centering
	%
	\begin{SubFloat}
		{\label{fig:kelm_ex1}%
			Ejemplo 1 de KELM}%
		\includegraphics[width=1\textwidth]%
		{Imagenes/Bitmap/03Objetivos/kelm_ex1}%
	\end{SubFloat}
	\qquad
	\begin{SubFloat}
		{\label{fig:kelm_ex2}%
			Ejemplo 2 de KELM}%
		\includegraphics[width=1\textwidth]%
		{Imagenes/Bitmap/03Objetivos/kelm_ex2}%
	\end{SubFloat}
	\caption{Diversos ejemplos del conjunto de datos KELM%
		\label{fig:kelm_ex}}
\end{figure}


En la figura \ref{fig:kelm_ex} se muestran dos ejemplos aleatorios del conjunto de datos KELM. Como se puede apreciar, cada ejemplo está formado por dos partes: la oración (\textit{sentence}), escrita en lenguaje natural; y la lista de tripletas (\textit{triple}), formada por las tripletas alineadas al texto de la oración separadas por una coma. Se muestran dos ejemplos diferentes ya que en el proceso de análisis de los datos se encontró que la lista de tripletas puede adoptar diversas estructuras. La primera forma, correspondiente al primer ejemplo mostrado (figura~\ref{fig:kelm_ex1}), es representada de la manera (entidad\_sujeto - nombre\_relacion1 - valor\_relacion1), (nombre\_relacion2 - valor\_relacion2) suponiendo que todas las tripletas se refieren a la entidad sujeto precisada en la primera de la lista. Otra estructura se muestra en la figura \ref{fig:kelm_ex2} en la que, debido a la diferencia de la entidad sujeto en las tripletas representadas, cada una de las tripletas es independiente de la otra adquiriendo la forma (entidad\_sujeto1 - nombre\_relacion1 - valor\_relacion1), (entidad\_sujeto2 - nombre\_relacion2 - valor\_relacion2)... Cabe destacar el formato adoptado en el conjunto de datos original no establece ninguna separación entre los elementos perteneciente a una misma tripleta. Debido a lo confuso que puede ser su comprensión se ha añadido a cada una de ellas una separación entre los elementos de la forma (entidad\_sujeto - nombre\_relacion - valor\_relacion) que no se encuentra en los datos originales.

Como se puede comprobar en los ejemplos presentados, este conjunto de datos no solo contiene información biográfica de personas, sino que también consta de conjuntos de ejemplos sobre otros temas. Es por esto por lo que puede resultar adecuada para el sistema que se pretende construir ya que podrá representar cualquier tipo de información que se desee sin limitarse al formato biográfico de Wiki2bio.


Con respecto al procesado de los datos, hay que destacar la limpieza de los datos originales. Únicamente hizo falta cambiar la codificación de los datos ya que se encontraban codificados en Windows-1252 por lo que algunos caracteres eran representados en un formato no apropiado para realizar el entrenamiento.


\section{Selección del modelo de lenguaje}
\label{sec:seleccion_modelo}

El núcleo del sistema a desarrollar en este trabajo es el modelo de lenguaje. Como hemos visto anteriormente, la cantidad de opciones para implementar su funcionamiento es inmensa. En esta sección, se muestra el resultado de la búsqueda de los modelos más apropiados para conseguir el objetivo que aquí se propone. 

De esta manera, se realizará un acercamiento a los modelos de lenguaje más empleados en la actualidad: GPT-2, BERT y T5. La selección de estos tres modelos no es una decisión arbitraria, sino que se escogieron aquellos modelos con mayor relevancia actualmente y que hubieran sido construidos con propósitos distintos de tal manera que se pueda discutir, según las características de cada uno de ellos, cuál/es serían los más apropiados para la tarea que se trata de realizar en este trabajo. Para poder considerar la adecuación del modelo se presentará el ajuste de cada uno de los modelos con respecto a un conjunto de datos de los presentados en la sección anterior \ref{sec:ConjuntosDeDatosYPreparacionParaElEntrenamiento}.



\subsection{Pre-entrenamiento del modelo}

Los modelos pre-entrenados han sido previamente capacitados para resolver una tarea mediante el aprendizaje de una serie de parámetros. Para realizar este pre-entrenamiento, generalmente, se utiliza aprendizaje auto-supervisado (\textit{Self-supervised Language Modelling}). Este proceso se realiza con los textos sin procesar, es decir, sin que los humanos etiqueten los datos de entrada de ninguna manera. La ventaja reside en que debido a la gran cantidad de datos que contienen los corpus utilizados en el entrenamiento, este procedimiento de etiquetado no se podría llevar a cabo de manera manual. %TODO: Contar un poco más - lo de que se sustituyen capas, etc ...


En los grandes modelos \textit{Transformers}, este proceso de entrenamiento es realizado en grandes computadores capaces de entrenar estos modelos sobre grandes conjuntos de datos. A continuación, vamos a centrarnos en los modelos de lenguaje que hemos considerado estudiar y extraeremos las principales claves de su pre-entrenamiento. También, se realizarán pruebas de funcionamiento de dichos modelos para comprobar sus resultados generales.



\subsubsection{GPT-2 (\textit{Generative Pretrained Transformer})}

Para comprender el proceso de generación de texto con este modelo se realizaron una serie de pruebas de funcionamiento básico. Para la obtención de todos los módulos necesario se utilizó la API \textit{Transformers} de la herramienta \textit{HuggingFace} que proporciona Python para la descarga y entrenamiento de modelos pre-entrenados. El modelo pre-entrenado utilizado es \textit{GPT2HeadModel}, una configuración del modelo GPT-2 preparado para modelado de lenguaje. Por otra parte, el \textit{tokenizer} empleado es \textit{GPT2Tokenizer} basado en el algoritmo \textit{Byte-Pair-Encoding} visto anteriormente.

Como apunte, la tokenización, en el campo del Procesamiento de Lenguaje Natural, se refiere al proceso de transformación de una secuencia de palabras o símbolos a \textit{tokens} para que la máquina pueda comprender el lenguaje humano y contexto detrás de él. El proceso de tokenización en GPT-2 se basa en la obtención de subpalabras mediante un algoritmo de codificación de pares de bytes (\textit{Byte Pair Encoding} o \textit{BPE}).

El algoritmo BPE \citep{gage1994new} es un algoritmo de tokenización basado en subpalabras. Su objetivo principal es la resolución de los problemas de otros tipos de tecnologías basadas en palabras o en caracteres mediante un enfoque intermedio. De manera teórica, BPE es una forma simple de comprensión de datos en el que el par más común de bytes de datos consecutivos se reemplaza con un byte que no aparece en esos datos. Esta idea garantiza que las palabras más comunes se representen en el vocabulario como un solo \textit{token}, mientras que las palabras menos habituales se dividen en dos o más \textit{tokens} de subpalabras.

Volviendo al tokenizador concreto utilizado, \textit{GPT2Tokenizer} tiene en cuenta los espacios y por tanto asignará diferentes \textit{tokens} teniendo en cuenta también dicho carácter. Se puntualiza que con el parámetro $add\_prefix\_space=True$ se puede sortear este comportamiento, aunque no es lo recomendable ya que el modelo no está pre-entrenado de esa manera y podría derivar en una disminución del rendimiento. El resultado de aplicar a un texto inicial el \textit{GPT2Tokenizer} se puede comprobar en el código \ref{lst:tokenizer-gpt2}.

\begin{lstlisting}[language=Python, caption=Ejemplo de uso de \textit{GPT2Tokenizer}, label={lst:tokenizer-gpt2}]
	tokenizer('I love Transformers', add_prefix_space=False)
	>> {'input_ids': [40, 1842, 39185], 'attention_mask': [1, 1, 1]}
	
	tokenizer(' I love you', add_prefix_space=False)
	>> {'input_ids': [314, 1842, 345], 'attention_mask': [1, 1, 1]}
\end{lstlisting} 

El resultado de aplicar la tokenización de \textit{GPT2Tokenizer} sobre una secuencia de palabras resulta en una lista denominada \textit{inputs\_ids} que asigna un número identificador a cada uno de los \textit{tokens} encontrados en dicha secuencia. En el ejemplo \ref{lst:enc_dec-gpt2} se puede comprobar el funcionamiento del proceso de codificar y decodificar. Dada una secuencia de entrada, en este caso \textit{'What is love?'}, se la pasamos al tokenizador y la codificamos. Este procedimiento devuelve los \textit{input\_its} en forma de objeto \textit{tensor} y a continuación decodificamos. La secuencia original y la resultante después de aplicar ambos procesos son similares.

\begin{lstlisting}[language=Python, caption= Encode y Decode, label={lst:enc_dec-gpt2}]
	>> original seq : What is love?
	>> input_ids    : tensor([[2061,  318, 1842,   30]])
	>> decoded seq  : What is love?
\end{lstlisting} 

A continuación, se describe el proceso completo de generación de texto con GPT-2 haciendo uso del tokenizador y del modelo. Para comenzar se codifica la secuencia de entrada al igual que se realizó en el ejemplo anterior. A continuación, se crea el modelo a partir del \textit{GPT2LMHeadModel} y se genera la salida con el método \textit{generate} (para generar el resultado final se emplearon los parámetros $max\_lenght=50$ para establecer una longitud máxima, $num\_beams=5$ y $no\_repeat\_ngram\_size=2$). Para finalizar, se decodifica la salida del generador, ya que el resultado es un objeto \textit{tensor} similar al producido en la codificación. En el ejemplo~\ref{lst:model-gpt2} mostrado se genera la continuación a la secuencia de entrada dada. El resultado es un texto coherente y cohesionado.

\begin{lstlisting}[language=Python, caption=Ejemplo de uso de \textit{GPT2LMHeadModel}, label={lst:model-gpt2}]
	>> What is love?
	Love is a word that has been around for a long time. It\'s a way of saying "I love you, but I don\'t know what it means to love someone else."	
\end{lstlisting} 

\subsubsection{BERT (\textit{Bidirectional Encoder Representations from Transformers})}

En este apartado, vamos a comprobar el funcionamiento del modelo BERT. Concretamente, se evaluará el proceso de predicción de la palabra más probable dada una secuencia con alguna palabra enmascarada con el \textit{token} [MASK].

Al igual que GPT-2, BERT necesita tokenizar los datos de entrada para poder manejarlos internamente. El tokenizador utilizado por este modelo de lenguaje es \textit{WordPiece} \citep{wordpiece} basado en subpalabras. Este algoritmo posee dos implementaciones: un enfoque ascendente de abajo hacia arriba y un enfoque descendente de arriba hacia abajo. El modelo BERT original utiliza el enfoque ascendente.

Este algoritmo no difiere demasiado del algoritmo BPE descrito anteriormente, ya que se trata de una versión modificada de dicho algoritmo. Sin embargo, \textit{WordPiece} trata de solucionar un problema común del BPE, limitado por la confusión de elección de un \textit{token} en el caso de las instancias que tiene más de una manera de ser codificadas. Debido a este problema, una misma entrada podría representarse mediante diferentes codificaciones pudiendo afectar a la precisión de las representaciones aprendidas.

Para realizar este proceso se utiliza el tokenizador \textit{BertTokenizer} que internamente
implementa el algoritmo \textit{WordPiece} descrito anteriormente. El primer ejemplo (código ~\ref{lst:tokenizer-bert}), muestra el resultado de tokenizar el texto de entrada. Se puede observar que la palabra ``thunderous'' no se encuentra en el vocabulario del tokenizador y por tanto la descompone en dos \textit{tokens}: ``thunder'' y ``\#\#ous''. Para indicar que estos \textit{tokens} no pertenecen a palabras separadas utiliza la doble almohadilla (\textit{\#\#}) como prefijo en el segundo \textit{token}.

\begin{lstlisting}[language=Python, caption=Resultado de aplicar \textit{BertTokenizer} a un texto de entrada, label={lst:tokenizer-bert}]
	sequence = "The thunderous roar of the jet overhead confirmed 
	her worst fears"
	
	>> ['The', 'thunder', '##ous', 'roar', 'of', 'the', 'jet', 
	'overhead', 'confirmed', 'her', 'worst', 'fears']
\end{lstlisting} 

A continuación, se muestra el proceso de predicción de una palabra enmascarada en una secuencia utilizada como entrada al modelo (código ~\ref{lst:model-bert}). El modelo utilizado es \textit{BertForMaskeLM}, una versión especial de BERT para la realización exclusiva de esta tarea. El proceso seguido para la predicción es muy sencillo: primero se obtienen los \textit{input\_ids} de los \textit{tokens} que conforman la entrada y se codifican. A continuación, se obtiene el índice de las palabras enmascaradas (en el ejemplo mostrado hay un solo \textit{token} [MASK]). Una vez obtenida la salida del modelo dada la secuencia de entrada, se aplica la función \textit{softmax} y finalmente se filtran las cinco palabras con mayor probabilidad. El resultado es una oración coherente mediante la generación de una palabra acorde con su entorno.

\begin{lstlisting}[language=Python, caption=Ejemplo de predicción de una palabra enmascarada en una secuencia, label={lst:model-bert}]
	text = "Every Monday, Mary goes to the " + tokenizer.mask_token + " to relax."
	
	>> Every Monday, Mary goes to the beach to relax.
	Every Monday, Mary goes to the library to relax.
	Every Monday, Mary goes to the bathroom to relax.
	Every Monday, Mary goes to the lake to relax.
	Every Monday, Mary goes to the gym to relax.
\end{lstlisting} 


\subsubsection{T5 (\textit{Text-to-Text Transfer Transformer})}


El funcionamiento básico de T5 es trivial habiendo estudiado el comportamiento de los anteriores modelos. 

Por una parte, mantiene la necesidad de tokenización de los datos de entrada, como cualquier otro sistema de este tipo ya que opera sobre valores numéricos. En el caso concreto de este modelo T5, se modifica el algoritmo de tokenización que utilizaban los anteriores modelos, basándose ahora en el algoritmo \textit{SentencePiece} que opera sobre regulación de subpalabras. Este algoritmo de tokenización, implementado en C++ es increíblemente rápido, lo que resulta en un entrenamiento y generación muchísimo más veloz en comparación con los tokenizadores utilizados en GPT-2 (BPE) o BERT (WordPiece). Otra de las ventajas de este tokenizador es que se utiliza directamente sobre los datos sin la necesidad de almacenar los datos tokenizados en discos, por lo que utiliza menos memoria en el proceso. Por otra parte, es agnóstico respecto a los espacios en blanco, confiriendo a idiomas que en ocasiones no hacen uso de ellos, como el chino o japones, la misma facilidad de tokenización que a cualquier otro lenguaje. En general, se basa en la idea de que la codificación de pares de bytes no es óptima para el entrenamiento previo del modelo de lenguaje \citep{bostrom-durrett-2020-byte}. Podemos acceder a este algoritmo de tokenización a través de la biblioteca \textit{HuggingFace}, concretamente lo implementa el módulo \textit{T5Tokenizer}. 

Una vez tokenizados los datos seguimos con la descarga del modelo para la prueba de funcionamiento. En este caso descargamos la versión más básica ('\textit{t5-base}') del módulo \textit{T5ForConditionalGeneration}. Para generar una salida, realizamos una llamada al método \textit{generate} con parámetros similares a los utilizados con GPT-2, (\textit{num\_beams} igual a 5 y \textit{no\_repeat\_ngram\_size} igual a 2). Los resultados de la prueba de generación se muestran en el código \ref{lst:model-t5}.


\begin{lstlisting}[language=Python, caption=Ejemplo de generación con el modelo pre-entrenado \textit{T5- base}, label={lst:model-t5}]
	sequence = '</s> Hello, my name'
	
	model.generate(sequence_tokenized['input_ids'], num_beams = 5 ,no_repeat_ngram_size = 2)
	
<< <pad> <extra_id_0> Hello, my name is Michael and I am looking for a new name.
\end{lstlisting} 


El código anterior muestra unos resultados coherentes en relación a la entrada dispuesta, en los que destaca la fluidez y la claridad de la redacción.




\subsection{\textit{Supervised fine-tuning}}
%https://s3-us-west-2.amazonaws.com/openai-assets/research-covers/language-unsupervised/language_understanding_paper.pdf


%%%%%%%%%%%%%%%%%%%%%%%%%%%%%%%%%%%%%%%%%%%%%%%%%%%%%


Como ya se ha comentado anteriormente, la aparición de los modelos \textit{Transformers} como evolución de los modelos basados en redes neuronales recurrentes para la generación de lenguaje natural supuso un cambio de paradigma en el modelado de lenguaje debido a la introducción de mecanismos de pre-entrenamiento y ajuste o \textit{finetune}. Después de un pre-entrenamiento sin supervisión del modelo bajo un gran conjunto de datos, dicho modelo puede ajustarse de manera mucho más rápida acorde a la tarea que se pretende lograr, utilizando para ello un conjunto de datos mucho más reducido y realizándose esta vez un entrenamiento supervisado. 

Dependiendo de la tarea que se pretenda conseguir con el ajuste del modelo, se debe realizar una correcta elección del conjunto de datos sobre el que se va a entrenar el modelo. La elección del conjunto de datos no es trivial, sino que obedece a las necesidades establecidas: en nuestro caso, generación de texto biográfico a partir de unos datos especificados como entrada. 

En los apartados siguientes se mostrarán los resultados del ajuste realizado a diferentes modelos de lenguaje bajo los conjuntos de datos presentados en el apartado \ref{sec:ConjuntosDeDatosYPreparacionParaElEntrenamiento}.

\subsubsection{Wiki2bio}

Para lograr un sistema de generación bibliográfica con esta \textit{dataset}, se escogió el modelo de lenguaje GPT-2 ya que el objetivo ideal de este \textit{dataset} es generar textos fluidos a partir de unos datos tabulares.

El proceso de ajuste de un modelo de lenguaje consta de una serie de pasos. En primer lugar, se debe escoger la versión adecuada del modelo de lenguaje seleccionado. En este caso, ya que se trata de una prueba se empleará la versión menos pesada (\textit{gpt2}) del modelo. Una vez seleccionada la versión se deben establecer los hiperparámetros que utilizará el modelo internamente para controlar el proceso de entrenamiento. La tasa de aprendizaje (\textit{learning rate}), el número de iteraciones de entrenamiento y el tamaño del lote (\textit{batch size}) son algunos ejemplos de hiperparámetros comunes. Los valores escogidos para estas variables tienen un impacto significativo en los parámetros aprendidos y, consecuentemente, en el rendimiento del modelo. 

A continuación, se creó una clase denominada \textit{MyDataset} que agrupa una serie de funcionalidades básicas para el manejo del conjunto de datos utilizado. Esta clase hereda de la clase abstracta \textit{Dataset} de la librería \textit{torch} utilizada para representar grandes corpus. Según se define en su documentación, las clases que hereden de \textit{Dataset} deben sobreescribir los métodos \textit{\_\_getitem\_\_(index)} que devuelve el elemento en la posición \textit{index} del conjunto de datos y \textit{\_\_len\_\_()} que retorna el tamaño de dicho conjunto. También se incluyeron métodos de conversión de la información procedente de \textit{Wiki2bio} en forma de diccionario con pares clave-valor a un formato comprensible por el modelo de lenguaje, en forma de texto.

Una vez realizados estos procedimientos básicos de gestión, se puede continuar el proceso de ajuste escogiendo un tokenizador adecuado. En este caso se utilizó \textit{GPT2-Tokenizer}, descrito anteriormente.

Para finalizar, se establece el modelo en modo de entrenamiento y se le asigna como datos de entrenamiento la porción de datos de \textit{MyDataset} establecida para ello. Para realizar esta prueba se realizará una selección de un subconjunto de los datos del corpus. Esta decisión viene motivada por los límites de Google Colab, si utilizamos una gran cantidad de datos la memoria de la GPU no lo soportaría y se provocaría una interrupción durante el entrenamiento; también se podrían modificar algunos parámetros que permitirían entrenar una mayor cantidad de datos, pero aumentando considerablemente el tiempo de ejecución. En nuestro caso seleccionamos 10.000 ejemplos del conjunto de datos original, de los cuales 8.000 se utilizarán para realizar el entrenamiento del modelo y 2.000 para la validación.

\begin{figure}[!h]
	\centering
	%
	\begin{SubFloat}
		{\label{fig:ex_gpt2_wiki2bio}%
			Ejemplo de generación de biografía con alucinaciones y degeneraciones}%
		\includegraphics[width=1\textwidth]%
		{Imagenes/Bitmap/05ModeladoDeLenguaje/ex_gpt2_wiiki2bio}%
	\end{SubFloat}
	\qquad
	\begin{SubFloat}
		{\label{fig:ex2_gpt2_wiki2bio}%
			Ejemplo de generación de biografía con pocas alucinaciones}%
		\includegraphics[width=0.95\textwidth]%
		{Imagenes/Bitmap/05ModeladoDeLenguaje/ex2_gpt2_wiiki2bio}%
	\end{SubFloat}
	\caption{Resultados de ajuste de GPT-2 en Wiki2bio%
		\label{fig:hallucinations_wiki2bio}}
\end{figure}


El funcionamiento interno del modelo en este modo de \textit{fine-tune} consiste en que para cada uno de los ejemplos establecidos como entrada, el modelo va aprendiendo a través de probabilidades cuales pueden ser las palabras siguientes a una secuencia. De esta manera se va ajustando el modelo a la tarea que se pretende conseguir, modificando los valores originales obtenidos después del pre-entrenamiento original.

Una vez realizado el entrenamiento que tomó en torno a dos horas, se puede comprobar con un ejemplo sencillo los resultados obtenidos. En la figura~\ref{fig:ex_gpt2_wiki2bio}, se muestra un ejemplo de resultado obtenido después de haber realizado el entrenamiento de ajuste. Como se puede apreciar algunos de los datos de salida son correctos y corresponden a los datos introducidos en la entrada. Sin embargo, se muestra un sobreajuste del modelo sobre los datos de entrenamiento ya que trata de generar un texto de longitud similar a los que han sido entrenados sin tener en cuenta el número de datos introducidos como entrada. 


Por otra parte, se puede observar que este entrenamiento cae en alucinaciones y degeneraciones (conceptos introducidos en la sección \ref{sec:efectos_alu_deg}) producidas en el momento en el que no sabe con qué información rellenar el texto. Estas alucinaciones se deben a que cuando se construyó este \textit{dataset}, los autores tomaron el cuadro de información de Wikipedia como fuente y la primera oración de la página de Wikipedia como referencia de verdad básica de texto de destino. Sin embargo, la primera oración del artículo de Wikipedia no es necesariamente equivalente al cuadro de información en términos de la información que contiene. De hecho \cite{dhingra2019handling}, señala que el 62 por ciento de las primeras frases de \textit{Wiki2bio} tienen información adicional no indicada en el \textit{infobox} correspondiente. Este desajuste entre el origen y destino en los conjuntos de datos puede hacer que los modelos entrenados alucinen como ocurre en este caso.

Para evitar en gran medida estas incorrecciones en la generación se puede limitar el número de palabras generadas dando como resultado la figura~\ref{fig:ex2_gpt2_wiki2bio} que muestra un texto mucho más natural, aunque todavía mostrando algunas alucinaciones. Sin embargo, tiene como desventaja la limitación en la longitud del texto a generar y la necesidad de aproximación de dicha longitud dependiendo de la información introducida.



\subsubsection{KELM}

Como ya se ha comentado, KELM es un gran corpus de datos que trata de generar texto a partir de un grafo de conocimiento representado en forma de tripletas semánticas. Este \textit{dataset} no solo contiene información en el formato propio de una biografía, sino que también contiene conjuntos de oraciones sobre otros temas. Es por esto por lo que puede resultar adecuada para el sistema que se pretende construir ya que podrá representar cualquier tipo de información que se desee sin limitarse al formato biográfico.

En este caso vamos a comprobar su funcionamiento bajo el modelo T5 descrito en el punto \label{sec:mtl}. El proceso de ajuste de este modelo es similar al utilizado para ajustar Wiki2bio sobre GPT-2. Una vez realizado todo el proceso de limpieza y preprocesado de los datos descargamos un modelo básico del \textit{Transformer} T5 denominado \textit{T5-small} para a continuación realizar un entrenamiento con supervisión bajo el conjunto de datos escogido. 

Para poder realizar el entrenamiento hace falta realizar una correcta selección de los hiperparámetros que se van a utilizar. El proceso de encontrar los mejores valores de hiperparámetros, que permite que el modelo descubra el mejor conjunto de parámetros para realizar una tarea determinada, recibe el nombre de \textit{optimización}. Para la optimización utilizaremos uno de los módulos disponibles a través de \textit{HuggingFace} conocido como \textit{AdaFactor}, un ``método de optimización estocástica basado en Adam que reduce el uso de la memoria al tiempo que conserva los beneficios empíricos de la adaptabilidad'' \citep{shazeer2018adafactor}.  Este método de optimización recibe una configuración inicial que hemos seleccionado según las recomendaciones de {HuggingFace} como se muestra en el código \ref{lst:adafactor}.

\begin{lstlisting}[language=Python, caption=on, caption=Configuración inicial del optimizador, label={lst:adafactor}]
optimizer = Adafactor(
    model.parameters(),
    lr = 1e-3,
    eps = (1e-30, 1e-3),
    clip_threshold = 1.0,
    decay_rate = -0.8,
    beta1 = None,
    weight_decay = 0.0,
    relative_step = False,
    scale_parameter = False,
    warmup_init = False,
)
\end{lstlisting} 

La entrada al modelo para realizar el entrenamiento está formada por ejemplos compuestos cada uno por su lista de tripletas separadas cada una de estas listas por una coma, un \textit{token} separador \textit{</s>} y la oración de destino. Así, cada uno de los ejemplos resultantes del proceso de pre-procesado del conjunto de datos se va introduciendo en el modelo durante el pre-entrenamiento produciéndose continuamente un ajuste de sus parámetros internos. Cabe decir que se escogieron 30.000 ejemplos para realizar el entrenamiento de prueba, aunque podrían haberse escogido más según se comprobó posteriormente ya que no nos acercamos al límite de uso de memoria de la GPU. Esto se debe a que T5 es un modelo que recorta bastante en el espacio requerido para el entrenamiento.

Ejemplos de los resultados obtenidos después de generar para diferentes tipos de datos las oraciones de salidas a través del modelo se encuentran en la figura \ref{fig:kelm_result}. Se han seleccionado dos ejemplos representativos para estudiar la adecuabilidad de utilización de este \textit{dataset}s. El primer ejemplo muestra una correcta generación de los datos proporcionados como entrada, al igual que el segundo de los ejemplos. Sin embargo, se puede apreciar una gran falta de representación de  los datos de entrada ya que en ambos ejemplos falta información dada en la entrada que no se muestra en la salida: el año de nacimiento en el primer ejemplo y este mismo año y la profesión en el segundo.

\begin{figure}[!h]%
	\centering%
	\includegraphics[width=1\textwidth]%
	{Imagenes/Bitmap/05ModeladoDeLenguaje/ex_t5_kelm}%
	\caption{Resultados de generación de KELM en el modelo T5%
		\label{fig:kelm_result}}
\end{figure}

\subsubsection{WebNLG}

WebNLG es un enorme corpus de datos con una estructura parecida a la anterior. Por una parte, proporcionaba una lista de tripletas, esta vez separando cada uno de los elementos de dicha tripleta, al contrario que hacía KELM. La ventaja de este conjunto de datos reside precisamente en esta separación ya que proporcionará al modelo un mayor conocimiento de la estructura de datos que se trata de pasar como entrada. De esta manera el modelo aprendería los ejemplos con más precisión. 

Para comprobar el funcionamiento de este conjunto de datos, se utilizaron dos modelos de los mencionados anteriormente: BERT y T5.

La elección de BERT como modelo de ajuste se realizó como alternativa a los modelos principalmente de generación casual que fueron entrenados en el resto de las pruebas. Ya que BERT es un modelo de lenguaje enmascarado utilizado especialmente para aprender una buena representación de los datos de entrada, pensamos que sería buena idea establecer otro planteamiento a la solución de modelado de lenguaje.

Para realizar el \textit{fine-tuning} de este modelo, una vez procesados todos los datos del conjunto WebNLG comenzamos descargando el modelo junto con el tokenizador. En nuestro caso hicimos uso, para ambos elementos, de la versión más básica: ``bert-base-uncased''. A continuación, transformamos el conjunto de datos en un objeto de tipo \textit{dataset} de \textit{HuggingFace} y manipulamos este objeto para componer la entrada al modelo. De esta manera pasamos de la estructura original de lista de tripletas (\textit{input\_text}) y la oración de destino (\textit{target\_text}) del conjunto de datos original a una secuencia de caracteres que contiene toda esta información y que es comprensible por el modelo.

Para componer esta secuencia de entrada, según se indica en las instrucciones del modelo, es necesario añadir una serie de \textit{tokens} a la entrada para que el modelo pueda distinguir las diferentes partes que la componen. Concretamente, se debe utilizar el \textit{token} [CLS] como \textit{token} de comienzo de oración (tradicionalmente expresado por [BOS]) y el \textit{token} separador [SEP] como fin de oración (tradicionalmente expresado por [EOS]. El resultado final de la entrada se muestra en la tabla \ref{tab:webnlg_bert}.

\begin{table}[h!]
    \begin{center}
    \begin{tabular}{p{0.2\linewidth} | p{0.7\linewidth}}
    
    \textit{input\_text} & albany, \_ oregon | ispartof | oregon \&\& albany, \_ oregon | country | united \_ states \&\& united \_ states | ethnicgroup | african \_ americans \\ \hline
    \textit{target\_text} & albany, oregon is in the united states, where african americans are an ethnic group \\ \hline
    Entrada final & [CLS] albany, \_ oregon | ispartof | oregon \&\& albany, \_ oregon | country | united \_ states \&\& united \_ states | ethnicgroup | african \_ americans [SEP] albany, oregon is in the united states, where african americans are an ethnic group. [SEP] \\
    
    \end{tabular}
    \caption{Conversión de los datos de entrada para BERT}
    \label{tab:webnlg_bert}
    \end{center}
\end{table}



A continuación, se extrajeron los identificadores de \textit{tokens} a través del tokenizador. Un conjunto de 10.000 ejemplos de este \textit{dataset} tokenizado conformará la entrada final del modelo. La elección de este número de ejemplos viene motivada por el mismo principio de GPT-2, su alto coste de memoria.
 
Para terminar, se establecieron ciertos hiperparámetros básicos para el entrenamiento  y se ajustó el modelo frente a los datos de entrada pasándole como parámetro un objeto de tipo \textit{DataCollator} que enmascarará un porcentaje (15\%) de los \textit{tokens} de entrada para que el modelo aprenda su tarea de MLM.


Para entrenar el modelo sobre WebNLG, se comenzó estableciendo el número de iteraciones o \textit{epochs} a dos. Sin embargo, los resultados obtenidos no eran suficientes como para ser comparados con el resto de modelos. Entonces, se volvió a entrenar de incrementando el número de iteraciones de dos en dos hasta obtener unos resultados decentes. Finalmente, el entrenamiento completo tomó doce iteraciones y teniendo en cuenta que cada dos iteraciones se empleaba una hora y media para realizar el entrenamiento, el tiempo de entrenamiento total asciende a en torno nueve horas, una cantidad considerablemente importante en comparación con otros modelos.

Para comprobar el resultado de los entrenamientos del modelo, se construyó un ejemplo de entrada correspondiente a los datos básicos biográficos de una persona. Esta secuencia comienza con el \textit{token} [CLS] y utiliza el \textit{token} [SEP] para marcar el final de la oración. Además, enmascaramos la palabra sobre la que se desea que se realice la predicción con el \textit{token} [MASK]. En la tabla siguiente se muestran los resultados de generación de lenguaje enmascarado de diferentes iteraciones realizadas durante el entrenamiento del modelo con la finalidad de comparar la evolución de aprendizaje a partir de la secuencia de entrada mostrada en la primera línea de la tabla \ref{tab:webnlg_bert_resultados}. Para cada una de las iteraciones se exponen las cinco palabras con mayor probabilidad, junto con el valor numérico de esta, y se muestra el resultado final de la oración sustituyendo el \textit{token} [MASK] por la palabra de mayor probabilidad.


\begin{table}[h!]
    \begin{center}
    \begin{tabular}{p{0.25\linewidth} | p{0.65\linewidth}}
    \multicolumn{2}{p{0.9\linewidth}}{\textbf{Texto de entrada:}} \\ 
    \multicolumn{2}{p{0.9\linewidth}}{[CLS] Alan Shepard | birth place | New Hampshire [SEP] Alan Shepard was [MASK] in New Hampshire [SEP]} \\ \hline

    \multirow{2}{*}{Segunda iteración} 
    &\textbf{Top tokens:} \& (0.58) - united (0.02) - states (0.01) - , (0.01) - new (0.01) \\ 
    &\textbf{Resultado:} alan shepard was \& in new hampshire \\ \hline
    
    \multirow{2}{*}{Cuarta iteración} 
    &\textbf{Top tokens:} | (0.05) - in (0.03) - of (0.03) - , (0.03) - birth (0.03) \\ 
    &\textbf{Resultado:} alan shepard was in in new hampshire \\ \hline
    
    \multirow{2}{*}{Octava iteración} 
    &\textbf{Top tokens:} born (0.16)- was (0.11) - in (0.06) - birth (0.03) - place (0.03) \\ 
    &\textbf{Resultado:} alan shepard was born in new hampshire \\ \hline
    
    \multirow{2}{30mm}{Décimo segunda iteración} 
    &\textbf{Top tokens:} born (0.66) - birth (0.09) - was (0.03)- died (0.02) - alma (0.02) \\ 
    &\textbf{Resultado:} alan shepard was born in new hampshire \\ \hline
    
    \end{tabular}
    \caption{Resultados de distintas \textit{epochs} con BERT}
    \label{tab:webnlg_bert_resultados}
    \end{center}
\end{table}


Como se puede observar el los resultados anteriores, los \textit{tokens} generados por los entrenamientos con menor número de iteraciones no tenían nada que ver con el posible resultado final. En general, el modelo no había aprendido bien del conjunto de datos resultando en la generación especialmente de símbolos tales como comas, ``et'' o preposiciones. Es de suponer que esta generación se debe a que estos \textit{tokens} son comúnmente repetidos a lo largo del conjunto de datos asignándoles así una mayor probabilidad. Por el contrario, los entrenamientos de ocho y doce iteraciones consiguieron unos mejores resultados, generando como \textit{tokens} de mayor probabilidad palabras acorde con el texto de entrada. Sin embargo, si tenemos que escoger uno, este sería el de doce iteraciones debido a que la probabilidad de escoger la palabra ``\textit{born}'' (resultado correcto, dicho sea) es bastante superior en este modelo que en el de ocho iteraciones, 66\% frente a 16\% respectivamente.


Para finalizar, tratamos de pasar de generación de palabras enmascaradas a generación de lenguaje casual ya que se puede comprobar que generar una única palabra en una oración no parece una solución demasiado buena para generar historias de vida completas. Para realizar este proceso, se comienza definiendo una secuencia de entrada, en este caso utilizaremos una parecida a la del ejemplo anterior, y se añade un \textit{token} de enmascaramiento ``[MASK]'' al final de la secuencia (antes del \textit{token} [SEP]). A continuación, se pasa al modelo esta secuencia de entrada y se obtiene el \textit{token} de mayor probabilidad para esa palabra enmascarada. Para finalizar, se añade este \textit{token} a la secuencia de entrada y se realiza este proceso de manera iterativa hasta que se obtenga una secuencia de la longitud deseada (tabla \ref{tab:webnlg_bert_resultados_CLM}).


\begin{table}[h!]
    \begin{center}
    \begin{tabular}{p{0.25\linewidth} | p{0.65\linewidth}}
    \multicolumn{2}{p{0.9\linewidth}}{\textbf{Texto de entrada:}} \\ 
    \multicolumn{2}{p{0.9\linewidth}}{[CLS] Alan Shepard | birth date | 1923-11-18 [SEP] Alan} \\ \hline

    \multirow{2}{*}{Primera iteración} 
    &\textbf{Top token:} shepard (0.24) \\ 
    &\textbf{Resultado:} alan shepard \\ \hline
    
    \multirow{2}{*}{Segunda iteración} 
    &\textbf{Top token:} shepard (0.16) \\ 
    &\textbf{Resultado:} alan shepard shepard\\ \hline
    
    \multirow{2}{*}{Tercera iteración} 
    &\textbf{Top token:} shepard (0.07) \\ 
    &\textbf{Resultado:} alan shepard shepard shepard\\ \hline
    
    \multirow{2}{*}{Cuarta iteración} 
    &\textbf{Top token:} in (0.05) \\ 
    &\textbf{Resultado:} alan shepard shepard shepard in\\ \hline
    
    \multirow{2}{*}{Quinta iteración} 
    &\textbf{Top token:} 1923 (0.04) \\ 
    &\textbf{Resultado:} alan shepard shepard shepard in 1923 \\ \hline
    
    \multirow{2}{*}{Sexta iteración} 
    &\textbf{Top token:} - (0.08) \\ 
    &\textbf{Resultado:} alan shepard shepard shepard in 1923 - \\ \hline
    
    \multirow{2}{*}{Séptima iteración} 
    &\textbf{Top token:} 01 (0.05) \\ 
    &\textbf{Resultado:} alan shepard shepard shepard in 1923 - 01\\ \hline
    
    \multirow{2}{*}{Décima iteración} 
    &\textbf{Top token:} - (0.03) \\ 
    &\textbf{Resultado:} alan shepard shepard shepard in 1923 - 01 - 01 -\\ \hline

    
    \end{tabular}
    \caption{Ejemplo de CLM con BERT}
    \label{tab:webnlg_bert_resultados_CLM}
    \end{center}
\end{table}

En el ejemplo anterior, se puede comprobar la evolución en la generación con este modelo. Aunque los resultados son ligeramente acertados (nombra a Alan Shepard y realiza un intento de poner la fecha de nacimiento de los datos de entrada), no son aceptables ya que el resultado no es un texto coherente y fluido, además de que no se muestra toda la información de la entrada. Debido a las repeticiones de la palabra ``shepard'' y de la secuencia ``- 01'', se puede apreciar la presencia bastante común de las degeneraciones estudiadas en la sección \ref{sec:efectos_alu_deg}. Otra de las limitaciones destacables, es la necesidad de establecer exactamente la longitud del texto a generar debido a que se realiza por iteraciones. Podrían considerarse decisiones de parada de generación, como la generación en algunas de la iteraciones del \textit{token}  punto (.). Sin embargo, esto llevaría a la representación de los datos de entrada siempre en una única oración, pudiendo dejar de lado datos de entrada que no tuvieran cabida en esa oración específica.

Debido al mal funcionamiento del entrenamiento del conjunto de datos sobre este modelo BERT recurrimos a entrenar de nuevo los datos de Webnlg sobre el modelo t5 ya que anteriormente nos había proporcionado un buen resultado sobre otro conjunto de datos. 

El proceso de entrenamiento de Webnlg sobre este modelo es parecido al de KELM. Una vez descargados y limpiados los datos del conjunto, se procedió a la composición de los datos de entrada y de destino. Al contrario que la generación con BERT, los datos de entrada están compuestos únicamente por los datos de las tripletas (\textit{input\_text}) y los datos de destino por la oración en la que se convertirán estas tripletas en lenguaje natural (\textit{target\_text}).

A continuación, se crean el tokenizador y el modelo de lenguaje con la versión ``t5-base'' para ambos componentes y se definen los hiperparámetros a utilizar a través del módulo \textit{AdaFactor}. Finalmente se entrenó el modelo T5 bajo 10.000 ejemplos escogidos de manera aleatoria del conjunto de datos. 
Entrenando los datos en una sola iteración (15 minutos), el resultado fue el expuesto en la tabla \ref{tab:webnlg_t5_resultados}.



\begin{table}[h!]
    \begin{center}
    \begin{tabular}{p{0.47\linewidth} | p{0.47\linewidth}}
    
    \textbf{Datos de entrada} & \textbf{Salida generada} \\ \hline
    
    Mary | hometown | London \&\& Mary | birthdate | 03-05-1950 </s> & Mary was born in London on May 1950. \\ \hline
    David | occupation | football player \&\& David | player at | Bayern Munich \&\& David | started playing | 2005 </s> & David began playing football in 2005 and played for Bayern Munich. \\ \hline
    
    Mario | hobby | exercise \&\& Mario | win | bodybuilding awards & Mario has won bodybuilding awards and is an exponent of exercise.
    \end{tabular}
    \caption{Resultados generados por T5 ajustado bajo WebNLG }
    \label{tab:webnlg_t5_resultados}
    \end{center}
\end{table}

Como se puede apreciar en los resultados anteriores que muestran ejemplos de generación de T5 ajustado bajo WebNLG, aún escogiendo una cantidad idéntica de ejemplos del conjunto de entrenamiento (10.000 ejemplos) que en el caso del ajuste de BERT y pese a que el número de iteraciones o \textit{epochs} (y por tanto, tiempo total) de entrenamiento es considerablemente menor, los resultados obtenidos son muchísimo mejores. Por una parte, hay que destacar la inclusión en los textos de salida de prácticamente todos los datos proporcionados en la entrada, aún tratando de complicarla al añadir más tripletas y la ausencia de alucinaciones, de degeneraciones y cualquier otro fenómeno propio de la generación de texto por redes neuronales. 

\section{Preprocesado y postprocesado de datos}

Habiendo observado los distintos conjuntos de datos posibles para ajustar el modelo de lenguaje se puede llegar a la conclusión de que la entrada en la que dispongamos de los datos, ya no en la etapa de ajuste sino una vez completada en la etapa de generación de los datos que el usuario desee transformar a lenguaje natural, no puede ser una entrada compacta con todos la información a generar, pero de igual manera tampoco puede resultar en entradas totalmente independientes. En el primero de los casos, introducción de toda la información de una historia de vida directamente al modelo, se puede afirmar que gran cantidad de la información a representar en la salida no se generará. Por el contrario, si introducimos la información de manera independiente una de otra (en el caso de las tripletas de WebNLG O KELM, tripleta a tripleta) el texto resultante podría perder bastante fluidez.

Frente a estas limitaciones, hemos llegado a la conclusión de que esta información va a tener que ser agrupada de alguna manera, de tal forma que cuando se realice el proceso de generación, la vayamos introduciendo agrupada en pequeñas porciones de un tamaño adecuado. 

La división de estas agrupaciones se puede hacer de diversas formas. En general, existen dos opciones contrapuestas para realizar este proceso: manual, el usuario realiza esta división; o automática, diseñando un algoritmo que realice este procedimiento.

Aquí, se decidió por una aproximación intermedia en la que el usuario puede establecer en alto nivel la agrupación de los datos principales mediante adición de una etiqueta a los datos introducidos y a bajo nivel, aquellos datos que no se puedan introducir debido a la gran cantidad de información en el subconjunto de datos podrán ser seleccionados en grupos mediante un algoritmo de \textit{Clustering}.

Además, se concederá al usuario la capacidad de ordenamiento de estos datos agrupados con el objetivo de que guíe la narrativa de la generación.



\section{Conclusión final}

Habiendo comprobado los resultados de diferentes combinaciones de bases de datos y modelos, se ha llegado a la conclusión de que la mejor combinación resultante ha sido la del ajuste del modelo T5 bajo el corpus WebNLG, conclusión trivial después de observar los resultados de la generación. 

Considerando todo el proceso de búsqueda de la combinación más adecuada, partimos del ajuste de GPT-2 con el conjunto de datos Wiki2bio. El problema de este conjunto era que debido al tamaño de sus ejemplos el modelo trata de generar textos de igual magnitud y si los datos proporcionados como entrada no son suficientes, caía en alucinaciones y degeneraciones imposibles de tratar posteriormente. 

Por otra parte, el corpus KELM conseguía la resolución de los problemas anteriores, aunque introducía algunos nuevos como la falta de representación de los datos de entrada en bastantes casos de los mostrados. Esto podría llevar a que datos importantes necesarios de ser representados en la salida no se tuvieran en cuenta. 

Posteriormente, el modelo BERT de enmascaramiento de lenguaje se descartó al comprobar que si se traba de realizar tareas de generación de lenguaje casual,
el pre-conocimiento de la longitud de los datos de salida era importante para poder generar las oraciones y este pre-conocimiento en algunas ocasiones podría ser difícil de determinar. Otras limitaciones de este modelo ajustado eran la generación de degeneraciones y la falta de representación de la mayoría de datos de entrada, además de la falta de coherencia y fluidez de los resultados.

Finalmente, el ajuste del modelo T5 para la generación de lenguaje bajo WebNLG resultó en muy buenos resultados sin ninguno de los efectos perjudiciales que en ocasiones las redes neuronales pueden llegar a provocar y pudiendo conseguir estos resultados en muchísimo menor tiempo que todos los anteriores modelos ajustados.

Así pues, la selección del modelo y del conjunto de datos adecuados para realizar el ajuste, nos lleva a que el formato de los datos de entrada será en forma de tripleta. Además, muestra la necesidad planteada de agrupar los datos en subconjuntos de longitud máxima de siete tripletas ya que el conjunto de datos WebNLG entrenará el modelo con ejemplos que utilizan entre una y siete tripletas. La adición de un mayor número de datos a la entrada podría conllevar la no inclusión de muchos de ellos en la salida.

Además, según se decidió, se emplearán métodos manuales y automáticos para la generación de estas agrupaciones de datos de entrada, al igual que se dispondrán de los métodos necesarios para la ordenación de los mismos en la salida.
\chapter{Sistema final}
\label{cap:sistemafinal}

\section{Descripción del sistema}
%Descripción del sistema final

\section{Agrupación de la información}
%Hablamos de webnlg y algoritmo de clustering para agrupar las tripletas

\section{Modelo de Lenguaje}
%Hablamos de T5 y como entrenamos el modelo

\section{Como agrupamos los párrafos... }


\section{Resultados finales}
\chapter{Interfaz y resultados}
\label{cap:GUI_resultados}
\chapter{Conclusiones y Trabajo Futuro}
\label{cap:conclusiones}

Conclusiones del trabajo y líneas de trabajo futuro.







%%%%%%%%%%%%%%%%%%%%%%%%%%%%%%%%%%%%%%%%%%%%%%%%%%%%%%%%%%%%%%%%%%%%%%%%%%%
% Si el TFM se escribe en inglés, comentar las siguientes líneas 
% porque no es necesario incluir nuevamente las Conclusiones en inglés
\setcounter{chapter}{\thechapter-1} 
\begin{otherlanguage}{english}
\chapter{Conclusions and Future Work}
\label{cap:conclusions}

Conclusions and future lines of work.



\end{otherlanguage}
%%%%%%%%%%%%%%%%%%%%%%%%%%%%%%%%%%%%%%%%%%%%%%%%%%%%%%%%%%%%%%%%%%%%%%%%%%%


% Apéndices
\appendix
%\chapter{Título}
\label{Appendix:Key1}

Contenido del apéndice
%\chapter{Título}
\label{Appendix:Key2}

%\include{Apendices/appendixC}
%\include{...}
%\include{...}
%\include{...}
\backmatter

%
% Bibliografía
%
% Si el TFM se escribe en inglés, editar TeXiS/TeXiS_bib para cambiar el
% estilo de las referencias
%---------------------------------------------------------------------
%
%                      configBibliografia.tex
%
%---------------------------------------------------------------------
%
% bibliografia.tex
% Copyright 2009 Marco Antonio Gomez-Martin, Pedro Pablo Gomez-Martin
%
% This file belongs to the TeXiS manual, a LaTeX template for writting
% Thesis and other documents. The complete last TeXiS package can
% be obtained from http://gaia.fdi.ucm.es/projects/texis/
%
% Although the TeXiS template itself is distributed under the 
% conditions of the LaTeX Project Public License
% (http://www.latex-project.org/lppl.txt), the manual content
% uses the CC-BY-SA license that stays that you are free:
%
%    - to share & to copy, distribute and transmit the work
%    - to remix and to adapt the work
%
% under the following conditions:
%
%    - Attribution: you must attribute the work in the manner
%      specified by the author or licensor (but not in any way that
%      suggests that they endorse you or your use of the work).
%    - Share Alike: if you alter, transform, or build upon this
%      work, you may distribute the resulting work only under the
%      same, similar or a compatible license.
%
% The complete license is available in
% http://creativecommons.org/licenses/by-sa/3.0/legalcode
%
%---------------------------------------------------------------------
%
% Fichero  que  configura  los  parámetros  de  la  generación  de  la
% bibliografía.  Existen dos  parámetros configurables:  los ficheros
% .bib que se utilizan y la frase célebre que aparece justo antes de la
% primera referencia.
%
%---------------------------------------------------------------------


%%%%%%%%%%%%%%%%%%%%%%%%%%%%%%%%%%%%%%%%%%%%%%%%%%%%%%%%%%%%%%%%%%%%%%
% Definición de los ficheros .bib utilizados:
% \setBibFiles{<lista ficheros sin extension, separados por comas>}
% Nota:
% Es IMPORTANTE que los ficheros estén en la misma línea que
% el comando \setBibFiles. Si se desea utilizar varias líneas,
% terminarlas con una apertura de comentario.
%%%%%%%%%%%%%%%%%%%%%%%%%%%%%%%%%%%%%%%%%%%%%%%%%%%%%%%%%%%%%%%%%%%%%%
\setBibFiles{%
nuestros,latex,otros%
}
\typeout{}
\bibliography{otros}

%%%%%%%%%%%%%%%%%%%%%%%%%%%%%%%%%%%%%%%%%%%%%%%%%%%%%%%%%%%%%%%%%%%%%%
% Definición de la frase célebre para el capítulo de la
% bibliografía. Dentro normalmente se querrá hacer uso del entorno
% \begin{FraseCelebre}, que contendrá a su vez otros dos entornos,
% un \begin{Frase} y un \begin{Fuente}.
%
% Nota:
% Si no se quiere cita, se puede eliminar su definición (en la
% macro setCitaBibliografia{} ).
%%%%%%%%%%%%%%%%%%%%%%%%%%%%%%%%%%%%%%%%%%%%%%%%%%%%%%%%%%%%%%%%%%%%%%
\setCitaBibliografia{
\begin{FraseCelebre}
\begin{Frase}
  Y así, del mucho leer y del poco dormir, se le secó el celebro de
  manera que vino a perder el juicio.
\end{Frase}
\begin{Fuente}
  Miguel de Cervantes Saavedra
\end{Fuente}
\end{FraseCelebre}
}

%% Creamos la bibliografia
%%
\makeBib

% Variable local para emacs, para  que encuentre el fichero maestro de
% compilación y funcionen mejor algunas teclas rápidas de AucTeX

%%%
%%% Local Variables:
%%% mode: latex
%%% TeX-master: "../Tesis.tex"
%%% End:

%
% Índice de palabras
%

% Sólo  la   generamos  si  está   declarada  \generaindice.  Consulta
% TeXiS.sty para más información.

% En realidad, el soporte para la generación de índices de palabras
% en TeXiS no está documentada en el manual, porque no ha sido usada
% "en producción". Por tanto, el fichero que genera el índice
% *no* se incluye aquí (está comentado). Consulta la documentación
% en TeXiS_pream.tex para más información.
\ifx\generaindice\undefined
\else
%%---------------------------------------------------------------------
%
%                        TeXiS_indice.tex
%
%---------------------------------------------------------------------
%
% TeXiS_indice.tex
% Copyright 2009 Marco Antonio Gomez-Martin, Pedro Pablo Gomez-Martin
%
% This file belongs to TeXiS, a LaTeX template for writting
% Thesis and other documents. The complete last TeXiS package can
% be obtained from http://gaia.fdi.ucm.es/projects/texis/
%
% This work may be distributed and/or modified under the
% conditions of the LaTeX Project Public License, either version 1.3
% of this license or (at your option) any later version.
% The latest version of this license is in
%   http://www.latex-project.org/lppl.txt
% and version 1.3 or later is part of all distributions of LaTeX
% version 2005/12/01 or later.
%
% This work has the LPPL maintenance status `maintained'.
% 
% The Current Maintainers of this work are Marco Antonio Gomez-Martin
% and Pedro Pablo Gomez-Martin
%
%---------------------------------------------------------------------
%
% Contiene  los  comandos  para  generar  el índice  de  palabras  del
% documento.
%
%---------------------------------------------------------------------
%
% NOTA IMPORTANTE: el  soporte en TeXiS para el  índice de palabras es
% embrionario, y  de hecho  ni siquiera se  describe en el  manual. Se
% proporciona  una infraestructura  básica (sin  terminar)  para ello,
% pero  no ha  sido usada  "en producción".  De hecho,  a pesar  de la
% existencia de  este fichero, *no* se incluye  en Tesis.tex. Consulta
% la documentación en TeXiS_pream.tex para más información.
%
%---------------------------------------------------------------------


% Si se  va a generar  la tabla de  contenidos (el índice  habitual) y
% también vamos a  generar el índice de palabras  (ambas decisiones se
% toman en  función de  la definición  o no de  un par  de constantes,
% puedes consultar modo.tex para más información), entonces metemos en
% la tabla de contenidos una  entrada para marcar la página donde está
% el índice de palabras.

\ifx\generatoc\undefined
\else
   \addcontentsline{toc}{chapter}{\indexname}
\fi


% Generamos el índice
\printindex

% Variable local para emacs, para  que encuentre el fichero maestro de
% compilación y funcionen mejor algunas teclas rápidas de AucTeX

%%%
%%% Local Variables:
%%% mode: latex
%%% TeX-master: "./tesis.tex"
%%% End:

\fi

%
% Lista de acrónimos
%

% Sólo  lo  generamos  si  está declarada  \generaacronimos.  Consulta
% TeXiS.sty para más información.


\ifx\generaacronimos\undefined
\else
%---------------------------------------------------------------------
%
%                        TeXiS_acron.tex
%
%---------------------------------------------------------------------
%
% TeXiS_acron.tex
% Copyright 2009 Marco Antonio Gomez-Martin, Pedro Pablo Gomez-Martin
%
% This file belongs to TeXiS, a LaTeX template for writting
% Thesis and other documents. The complete last TeXiS package can
% be obtained from http://gaia.fdi.ucm.es/projects/texis/
%
% This work may be distributed and/or modified under the
% conditions of the LaTeX Project Public License, either version 1.3
% of this license or (at your option) any later version.
% The latest version of this license is in
%   http://www.latex-project.org/lppl.txt
% and version 1.3 or later is part of all distributions of LaTeX
% version 2005/12/01 or later.
%
% This work has the LPPL maintenance status `maintained'.
% 
% The Current Maintainers of this work are Marco Antonio Gomez-Martin
% and Pedro Pablo Gomez-Martin
%
%---------------------------------------------------------------------
%
% Contiene  los  comandos  para  generar  el listado de acrónimos
% documento.
%
%---------------------------------------------------------------------
%
% NOTA IMPORTANTE:  para que la  generación de acrónimos  funcione, al
% menos  debe  existir  un  acrónimo   en  el  documento.  Si  no,  la
% compilación  del   fichero  LaTeX  falla  con   un  error  "extraño"
% (indicando  que  quizá  falte  un \item).   Consulta  el  comentario
% referente al paquete glosstex en TeXiS_pream.tex.
%
%---------------------------------------------------------------------


% Redefinimos a español  el título de la lista  de acrónimos (Babel no
% lo hace por nosotros esta vez)

\def\listacronymname{Lista de acrónimos}

% Para el glosario:
% \def\glosarryname{Glosario}

% Si se  va a generar  la tabla de  contenidos (el índice  habitual) y
% también vamos a  generar la lista de acrónimos  (ambas decisiones se
% toman en  función de  la definición  o no de  un par  de constantes,
% puedes consultar config.tex  para más información), entonces metemos
% en la  tabla de contenidos una  entrada para marcar  la página donde
% está el índice de palabras.

\ifx\generatoc\undefined
\else
   \addcontentsline{toc}{chapter}{\listacronymname}
\fi


% Generamos la lista de acrónimos (en realidad el índice asociado a la
% lista "acr" de GlossTeX)

\printglosstex(acr)

% Variable local para emacs, para  que encuentre el fichero maestro de
% compilación y funcionen mejor algunas teclas rápidas de AucTeX

%%%
%%% Local Variables:
%%% mode: latex
%%% TeX-master: "../Tesis.tex"
%%% End:

\fi

%
% Final
%
%---------------------------------------------------------------------
%
%                      fin.tex
%
%---------------------------------------------------------------------
%
% fin.tex
% Copyright 2009 Marco Antonio Gomez-Martin, Pedro Pablo Gomez-Martin
%
% This file belongs to the TeXiS manual, a LaTeX template for writting
% Thesis and other documents. The complete last TeXiS package can
% be obtained from http://gaia.fdi.ucm.es/projects/texis/
%
% Although the TeXiS template itself is distributed under the 
% conditions of the LaTeX Project Public License
% (http://www.latex-project.org/lppl.txt), the manual content
% uses the CC-BY-SA license that stays that you are free:
%
%    - to share & to copy, distribute and transmit the work
%    - to remix and to adapt the work
%
% under the following conditions:
%
%    - Attribution: you must attribute the work in the manner
%      specified by the author or licensor (but not in any way that
%      suggests that they endorse you or your use of the work).
%    - Share Alike: if you alter, transform, or build upon this
%      work, you may distribute the resulting work only under the
%      same, similar or a compatible license.
%
% The complete license is available in
% http://creativecommons.org/licenses/by-sa/3.0/legalcode
%
%---------------------------------------------------------------------
%
% Contiene la última página
%
%---------------------------------------------------------------------


% Ponemos el marcador en el PDF
\ifpdf
   \pdfbookmark{Fin}{fin}
\fi

\thispagestyle{empty}\mbox{}

\vspace*{4cm}

\small

\hfill \emph{--¿Qué te parece desto, Sancho? -- Dijo Don Quijote --}

\hfill \emph{Bien podrán los encantadores quitarme la ventura,}

\hfill \emph{pero el esfuerzo y el ánimo, será imposible.}

\hfill 

\hfill \emph{Segunda parte del Ingenioso Caballero} 

\hfill \emph{Don Quijote de la Mancha}

\hfill \emph{Miguel de Cervantes}

\vfill%space*{4cm}

\hfill \emph{--Buena está -- dijo Sancho --; fírmela vuestra merced.}

\hfill \emph{--No es menester firmarla -- dijo Don Quijote--,}

\hfill \emph{sino solamente poner mi rúbrica.}

\hfill 

\hfill \emph{Primera parte del Ingenioso Caballero} 

\hfill \emph{Don Quijote de la Mancha}

\hfill \emph{Miguel de Cervantes}


\newpage
\thispagestyle{empty}\mbox{}

\newpage

% Variable local para emacs, para  que encuentre el fichero maestro de
% compilación y funcionen mejor algunas teclas rápidas de AucTeX

%%%
%%% Local Variables:
%%% mode: latex
%%% TeX-master: "../Tesis.tex"
%%% End:

%\end{otherlanguage}
\end{document}
