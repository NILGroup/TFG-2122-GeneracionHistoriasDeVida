\chapter{Introducción}
\label{cap:introduccion}

\chapterquote{Frase célebre dicha por alguien inteligente}{Autor}

\section{Contexto}
La enfermedad de Alzheimer es un trastorno neurológico caracterizado por cambios degenerativos en diferentes sistemas neurotransmisores que abocan finalmente a la muerte de las células nerviosas del cerebro. Por lo general, los síntomas iniciales de la enfermedad suelen atribuirse a un olvido puntual o la vejez, nada más lejos de la realidad. Según avanza la enfermedad, sus síntomas lo hacen con ella, agravándose y aumentando cada vez más hasta que el deterioro cognitivo ocasionado llega a afectar significativamente a las actividades de la vida diaria y finalmente a las necesidades fisiológicas básicas.

La evolución del Alzheimer se puede dividir en tres fases o etapas: Inicial, en la que se observa un deterioro cognitivo leve como puede ser la pérdida paulatina de la memoria episódica. Intermedia, con pérdidas de la memoria reciente asociada a un deterioro mayor. Avanzada, se va perdiendo progresivamente la memoria de los acontecimientos más antiguos, además también se observa un gran deterioro físico.

Los síntomas del Alzheimer son muy diversos, ya que no solo provoca problemas de memoria, sino también alteraciones en el estado de ánimo y la conducta, dificultad de toma de decisiones, desorientación, problemas del lenguaje, dificultad para comer y movilidad reducida, entre otros. Estos síntomas dependen de la fase evolutiva de la enfermedad.

En la actualizad, el Alzheimer es una enfermedad irreversible, no tiene cura. Sin embargo, existen diversos tratamientos que pretenden ralentizar el avance de la enfermedad y/o mejorar la calidad de vida de los pacientes. Podemos dividir estos tratamientos en dos ramas: tratamientos farmacológicos, que hacen uso de medicamentos, y tratamientos no farmacológicos o psicosociales, que no utilizan sustancias químicas. Ambos tratamientos son eficaces y de la combinación de ambos resulta el procedimiento más recomendado para tratar la enfermedad.

Existen una gran variedad de terapias no farmacológicas, algunas de las más utilizadas son el entrenamiento y estimulación cognitiva, ejercicio físico, musicoterapia, etc. Sin embargo, la técnica más utilizada dentro del contexto que aquí estudiamos es la reminiscencia.

La reminiscencia es el acto o proceso de recordar recuerdos del pasado. Esto puede implicar el recuerdo de episodios particulares o genéricos que pueden o no haber sido olvidados previamente, y que son acompañados por la sensación de que estos episodios son relatos verídicos de las experiencias originales. Esta técnica es utilizada para estimular la memoria episódica autobiográfica mediante el encadenamiento de recuerdos, que se agrupan en categorías y se archivan en el tiempo mediante la elaboración de la historia de vida.

La historia de vida es una técnica narrativa que se basa en organizar y estructurar recuerdos de una persona para componer una autobiografía.

\section{Motivación}
\subsection{Explicaciones adicionales}
Si quieres cambiar el \textbf{estilo del título} de los capítulos, abre el fichero \verb|TeXiS\TeXiS_pream.tex| y comenta la línea \verb|\usepackage[Lenny]{fncychap}| para dejar el estilo básico de \LaTeX.

Si no te gusta que no haya \textbf{espacios entre párrafos} y quieres dejar un pequeño espacio en blanco, no metas saltos de línea (\verb|\\|) al final de los párrafos. En su lugar, busca el comando  \verb|\setlength{\parskip}{0.2ex}| en \verb|TeXiS\TeXiS_pream.tex| y aumenta el valor de $0.2ex$ a, por ejemplo, $1ex$.

El siguiente texto se genera con el comando \verb|\lipsum[2-20]| que viene a continuación en el fichero .tex. El único propósito es mostrar el aspecto de las páginas usando esta plantilla. Quita este comando y, si quieres, comenta o elimina el paquete \textit{lipsum} al final de \verb|TeXiS\TeXiS_pream.tex|

\subsubsection{Texto de prueba}


\lipsum[2-20]